% Options for packages loaded elsewhere
\PassOptionsToPackage{unicode}{hyperref}
\PassOptionsToPackage{hyphens}{url}
%
\documentclass[
  letterpaper,
]{book}

\usepackage{amsmath,amssymb}
\usepackage{iftex}
\ifPDFTeX
  \usepackage[T1]{fontenc}
  \usepackage[utf8]{inputenc}
  \usepackage{textcomp} % provide euro and other symbols
\else % if luatex or xetex
  \usepackage{unicode-math}
  \defaultfontfeatures{Scale=MatchLowercase}
  \defaultfontfeatures[\rmfamily]{Ligatures=TeX,Scale=1}
\fi
\usepackage{lmodern}
\ifPDFTeX\else  
    % xetex/luatex font selection
\fi
% Use upquote if available, for straight quotes in verbatim environments
\IfFileExists{upquote.sty}{\usepackage{upquote}}{}
\IfFileExists{microtype.sty}{% use microtype if available
  \usepackage[]{microtype}
  \UseMicrotypeSet[protrusion]{basicmath} % disable protrusion for tt fonts
}{}
\makeatletter
\@ifundefined{KOMAClassName}{% if non-KOMA class
  \IfFileExists{parskip.sty}{%
    \usepackage{parskip}
  }{% else
    \setlength{\parindent}{0pt}
    \setlength{\parskip}{6pt plus 2pt minus 1pt}}
}{% if KOMA class
  \KOMAoptions{parskip=half}}
\makeatother
\usepackage{xcolor}
\ifLuaTeX
  \usepackage{luacolor}
  \usepackage[soul]{lua-ul}
\else
  \usepackage{soul}
  
\fi
\setlength{\emergencystretch}{3em} % prevent overfull lines
\setcounter{secnumdepth}{5}
% Make \paragraph and \subparagraph free-standing
\makeatletter
\ifx\paragraph\undefined\else
  \let\oldparagraph\paragraph
  \renewcommand{\paragraph}{
    \@ifstar
      \xxxParagraphStar
      \xxxParagraphNoStar
  }
  \newcommand{\xxxParagraphStar}[1]{\oldparagraph*{#1}\mbox{}}
  \newcommand{\xxxParagraphNoStar}[1]{\oldparagraph{#1}\mbox{}}
\fi
\ifx\subparagraph\undefined\else
  \let\oldsubparagraph\subparagraph
  \renewcommand{\subparagraph}{
    \@ifstar
      \xxxSubParagraphStar
      \xxxSubParagraphNoStar
  }
  \newcommand{\xxxSubParagraphStar}[1]{\oldsubparagraph*{#1}\mbox{}}
  \newcommand{\xxxSubParagraphNoStar}[1]{\oldsubparagraph{#1}\mbox{}}
\fi
\makeatother


\providecommand{\tightlist}{%
  \setlength{\itemsep}{0pt}\setlength{\parskip}{0pt}}\usepackage{longtable,booktabs,array}
\usepackage{calc} % for calculating minipage widths
% Correct order of tables after \paragraph or \subparagraph
\usepackage{etoolbox}
\makeatletter
\patchcmd\longtable{\par}{\if@noskipsec\mbox{}\fi\par}{}{}
\makeatother
% Allow footnotes in longtable head/foot
\IfFileExists{footnotehyper.sty}{\usepackage{footnotehyper}}{\usepackage{footnote}}
\makesavenoteenv{longtable}
\usepackage{graphicx}
\makeatletter
\newsavebox\pandoc@box
\newcommand*\pandocbounded[1]{% scales image to fit in text height/width
  \sbox\pandoc@box{#1}%
  \Gscale@div\@tempa{\textheight}{\dimexpr\ht\pandoc@box+\dp\pandoc@box\relax}%
  \Gscale@div\@tempb{\linewidth}{\wd\pandoc@box}%
  \ifdim\@tempb\p@<\@tempa\p@\let\@tempa\@tempb\fi% select the smaller of both
  \ifdim\@tempa\p@<\p@\scalebox{\@tempa}{\usebox\pandoc@box}%
  \else\usebox{\pandoc@box}%
  \fi%
}
% Set default figure placement to htbp
\def\fps@figure{htbp}
\makeatother
% definitions for citeproc citations
\NewDocumentCommand\citeproctext{}{}
\NewDocumentCommand\citeproc{mm}{%
  \begingroup\def\citeproctext{#2}\cite{#1}\endgroup}
\makeatletter
 % allow citations to break across lines
 \let\@cite@ofmt\@firstofone
 % avoid brackets around text for \cite:
 \def\@biblabel#1{}
 \def\@cite#1#2{{#1\if@tempswa , #2\fi}}
\makeatother
\newlength{\cslhangindent}
\setlength{\cslhangindent}{1.5em}
\newlength{\csllabelwidth}
\setlength{\csllabelwidth}{3em}
\newenvironment{CSLReferences}[2] % #1 hanging-indent, #2 entry-spacing
 {\begin{list}{}{%
  \setlength{\itemindent}{0pt}
  \setlength{\leftmargin}{0pt}
  \setlength{\parsep}{0pt}
  % turn on hanging indent if param 1 is 1
  \ifodd #1
   \setlength{\leftmargin}{\cslhangindent}
   \setlength{\itemindent}{-1\cslhangindent}
  \fi
  % set entry spacing
  \setlength{\itemsep}{#2\baselineskip}}}
 {\end{list}}
\usepackage{calc}
\newcommand{\CSLBlock}[1]{\hfill\break\parbox[t]{\linewidth}{\strut\ignorespaces#1\strut}}
\newcommand{\CSLLeftMargin}[1]{\parbox[t]{\csllabelwidth}{\strut#1\strut}}
\newcommand{\CSLRightInline}[1]{\parbox[t]{\linewidth - \csllabelwidth}{\strut#1\strut}}
\newcommand{\CSLIndent}[1]{\hspace{\cslhangindent}#1}

\usepackage{makeidx}
\makeindex
% \makeindex[intoc=true, columns=3, columnseprule=true, options=-s latex/indexstyles.ist]
\makeatletter
\@ifpackageloaded{bookmark}{}{\usepackage{bookmark}}
\makeatother
\makeatletter
\@ifpackageloaded{caption}{}{\usepackage{caption}}
\AtBeginDocument{%
\ifdefined\contentsname
  \renewcommand*\contentsname{Table of contents}
\else
  \newcommand\contentsname{Table of contents}
\fi
\ifdefined\listfigurename
  \renewcommand*\listfigurename{List of Figures}
\else
  \newcommand\listfigurename{List of Figures}
\fi
\ifdefined\listtablename
  \renewcommand*\listtablename{List of Tables}
\else
  \newcommand\listtablename{List of Tables}
\fi
\ifdefined\figurename
  \renewcommand*\figurename{Figure}
\else
  \newcommand\figurename{Figure}
\fi
\ifdefined\tablename
  \renewcommand*\tablename{Table}
\else
  \newcommand\tablename{Table}
\fi
}
\@ifpackageloaded{float}{}{\usepackage{float}}
\floatstyle{ruled}
\@ifundefined{c@chapter}{\newfloat{codelisting}{h}{lop}}{\newfloat{codelisting}{h}{lop}[chapter]}
\floatname{codelisting}{Listing}
\newcommand*\listoflistings{\listof{codelisting}{List of Listings}}
\makeatother
\makeatletter
\makeatother
\makeatletter
\@ifpackageloaded{caption}{}{\usepackage{caption}}
\@ifpackageloaded{subcaption}{}{\usepackage{subcaption}}
\makeatother

\usepackage{bookmark}

\IfFileExists{xurl.sty}{\usepackage{xurl}}{} % add URL line breaks if available
\urlstyle{same} % disable monospaced font for URLs
\hypersetup{
  pdftitle={Snap! Reference Manual},
  pdfauthor={Brian Harvey \& Jens Mönig},
  hidelinks,
  pdfcreator={LaTeX via pandoc}}


\title{Snap! Reference Manual}
\author{Brian Harvey \& Jens Mönig}
\date{2024-11-12}

\begin{document}
\frontmatter
\maketitle

\renewcommand*\contentsname{Table of contents}
{
\setcounter{tocdepth}{2}
\tableofcontents
}

\mainmatter
\bookmarksetup{startatroot}

\chapter*{The Snap! Reference Manual}\label{the-snap-reference-manual}
\addcontentsline{toc}{chapter}{The Snap! Reference Manual}

\markboth{The Snap! Reference Manual}{The Snap! Reference Manual}

\index{home}

\bookmarksetup{startatroot}

\chapter{Acknowledgements}\label{acknowledgements}

We have been extremely lucky in our mentors. Jens cut his teeth in the
company of the Smalltalk pioneers: Alan Kay{[}{]}\{index=``Kay, Alan''\}
, Dan Ingalls{[}{]}\{index=``Ingalls, Dan''\} , and the rest of the gang
who invented personal computing and object oriented programming in the
great days of Xerox PARC{[}{]}\{index=``Xerox PARC''\} . He worked with
John Maloney{[}{]}\{index=``Maloney, John''\} , of the
MIT{[}{]}\{index=``Massachusetts Institute of Technology''\} Scratch
Team{[}{]}\{index=``Scratch Team''\} , who developed the
Morphic{[}{]}\{index=``Morphic''\} graphics framework that's still at
the heart of Snap\emph{!}.

\textbf{\emph{The brilliant design of Scratch, from the Lifelong
Kindergarten Group}}{[}{]}\{index=``Lifelong Kindergarten Group''\}
\textbf{\emph{at the MIT Media Lab}}{[}{]}\{index=``Media Lab''\}
\textbf{\emph{, is crucial to} Snap\emph{!. Our earlier version, BYOB,
was a direct modification of the Scratch source code.} Snap\emph{! is a
complete rewrite, but its code structure and its user interface remain
deeply indebted to Scratch. And the Scratch Team, who could have seen us
as rivals, have been entirely supportive and welcoming to us.}}

Brian grew up at the MIT and Stanford Artificial Intelligence
Labs{[}{]}\{index=``MIT Artificial Intelligence Lab''\} , learning from
Lisp inventor John McCarthy{[}{]}\{index='' McCarthy, John''\} ,
Scheme{[}{]}\{index=``Scheme''\} inventors Gerald J.
Sussman{[}{]}\{index=``Sussman, Gerald J.''\} and Guy
Steele{[}{]}\{index=``Steele, Guy''\} , and the authors of the world's
best computer science book, \emph{Structure and Interpretation of
Computer Programs}{[}{]}\{index=``Structure and Interpretation of
Computer Programs''\} \emph{,} Hal Abelson{[}{]}\{index=``Abelson,
Hal''\} and Gerald J. Sussman with Julie Sussman{[}{]}\{index=``Sussman,
Julie''\} , among many other heroes of computer science. (Brian was also
lucky enough, while in high school, to meet Kenneth
Iverson{[}{]}\{index=``Iverson, Kenneth E.''\} , the inventor of
APL{[}{]}\{index=``APL''\} .)

\textbf{\emph{In the glory days of the MIT Logo Lab, we used to say,
``Logo is Lisp disguised as BASIC.'' Now, with its first class
procedures, lexical scope, and first class continuations,} Snap\emph{!
is Scheme disguised as Scratch.}}

Four people have made such massive contributions to the implementation
of Snap\emph{!} that we have officially declared them members of the
team: Michael Ball{[}{]}\{index=``Ball, Michael''\} and Bernat Romagosa,
in addition to contributions throughout the project, have primary
responsibility for the web site and cloud
storage{[}{]}\{index=``Romagosa, Bernat''\} . Joan Guillén i
Pelegay{[}{]}\{index=``Guillén i Pelegay, Joan''\} has contributed very
careful and wise analysis of outstanding issues, including help in
taming the management of translations to non-English languages. Jadga
Hügle{[}{]}\{index=``Huegle, Jadga''\} , has energetically contributed
to online mini-courses about Snap\emph{!} and leading workshops for kids
and for adults. Jens, Jadga, and Bernat are paid to work on Snap\emph{!}
by SAP, which also supports our computing needs.

We have been fortunate to get to know an amazing group of brilliant
middle school(!) and high school students through the Scratch Advanced
Topics forum, several of whom (since grown up) have contributed code to
Snap\emph{!}: Kartik Chandra{[}{]}\{index=``Chandra, Kartik''\} , Nathan
Dinsmore{[}{]}\{index=``Dinsmore, Nathan''\} , Connor
Hudson{[}{]}\{index=``Hudson, Connor''\} , Ian
Reynolds{[}{]}\{index=``Reynolds, Ian''\} , and Deborah
Servilla{[}{]}\{index=``Servilla, Deborah''\} . Many more have
contributed ideas and alpha-testing bug reports. UC Berkeley students
who've contributed code include Achal Dave{[}{]}\{index=``Dave,
Achal''\} . Kyle Hotchkiss{[}{]}\{index=``Hotchkiss. Kyle''\} , Ivan
Motyashov{[}{]}\{index=``Motyashov, Ivan''\} , and Yuan
Yuan{[}{]}\{index=``Yuan, Yuan''\} . Contributors of translations are
too numerous to list here, but they're in the ``About\ldots{}'' box in
Snap\emph{!} itself.

This material is based upon work supported in part by the National
Science Foundation under Grants No. 1138596, 1143566, and 1441075; and
in part by MioSoft, Arduino.org, SAP, and YC Research. Any opinions,
findings, and conclusions or recommendations expressed in this material
are those of the author(s) and do not necessarily reflect the views of
the National Science Foundation or other funders.

\textsc{\hfill\break
}\textbf{Snap\emph{!} Reference Manual}

\textbf{Version 8.0}

Snap\emph{!} (formerly BYOB) is an extended reimplementation of Scratch
(https://scratch.mit.edu) that allows you to Build Your Own Blocks. It
also features first class lists, first class procedures, first class
sprites, first class costumes, first class sounds, and first class
continuations. These added capabilities make it suitable for a serious
introduction to computer science for high school or college students.

In this manual we sometimes make reference to Scratch, e.g., to explain
how some Snap\emph{!} feature extends something familiar in Scratch.
It's very helpful to have some experience with Scratch before reading
this manual, but not essential.

To run Snap\emph{!}, open a browser window and connect to
https://snap.berkeley.edu/run. The Snap\emph{!} community web site at
https://snap.berkeley.edu is not part of this manual's scope.

\bookmarksetup{startatroot}

\chapter{Blocks, Scripts, and Sprites}\label{blocks-scripts-and-sprites}

This chapter describes the Snap\emph{!} features inherited from
Scratch{[}{]}\{index=``Scratch''\} ; experienced Scratch users can skip
to Section~B.

Snap\emph{!} is a programming language---a notation in which you can
tell a computer what you want it to do. Unlike most programming
languages, though, Snap\emph{!} is a \emph{visual} language; instead of
writing a program using the keyboard, the Snap\emph{!} programmer uses
the same drag-and-drop interface familiar to computer users.

Start Snap\emph{!}. You should see the following arrangement of
regions{[}{]}\{index=``layout, window''\} in the window:

(The proportions of these areas may be different, depending on the size
and shape of your browser window.)

A Snap\emph{!} program{[}{]}\{index='' Snap! program''\} consists of one
or more \emph{scripts,} each of which is made of \emph{blocks.} Here's a
typical script{[}{]}\{index=``script''\} :

The five block{[}{]}\{index=``block''\} s that make up this script have
three different colors, corresponding to three of the eight
\emph{palettes} in which blocks can be found. The
palette{[}{]}\{index=``palette''\} area at the left edge of the window
shows one palette at a time, chosen with the eight buttons just above
the palette area. In this script, the gold blocks are from the Control
palette; the green block is from the Pen palette; and the blue blocks
are from the Motion palette. A script is assembled by dragging blocks
from a palette into the \emph{scripting area}{[}{]}\{index=``scripting
area''\} in the middle part of the window. Blocks snap together (hence
the name Snap\emph{!} for the language) when you drag a block so that
its indentation is near the tab of the one above it:

The white horizontal line is a signal that if you let go of the green
block it will snap into the tab of the gold one.

\subsection{Hat Blocks and Command
Blocks}\label{hat-blocks-and-command-blocks}

At the top of the script is a \emph{hat} block, which indicates when the
script should be carried out. Hat block names typically start with the
word ``when''; in the square-drawing example on page 5, the script
should be run when the green flag{[}{]}\{index=``flag, green''\} near
the right end of the Snap\emph{!} tool bar{[}{]}\{index=``tool bar''\}
is clicked. (The Snap\emph{!} tool bar is part of the Snap\emph{!}
window, not the same as the browser's or operating system's menu bar.) A
script isn't required to have a hat block{[}{]}\{index=``block:hat''\} ,
but if not, then the script will be run only if the user clicks on the
script itself. A script can't have more than one hat block, and the hat
block can be used only at the top of the script; its distinctive shape
is meant to remind you of
that.{[}1{]}\phantomsection\label{generic_when}{}

The other blocks in our example script are \emph{command}
block{[}{]}\{index=``block:command''\} s. Each command
block{[}{]}\{index=``command block''\} corresponds to an action that
Snap\emph{!} already knows how to carry out. For example, the block
tells the sprite{[}{]}\{index=``sprite''\} (the arrowhead shape on the
\emph{stage}{[}{]}\{index=``stage''\} at the right end of the window) to
move ten steps (a step is a very small unit of distance) in the
direction in which the arrowhead is pointing. We'll see shortly that
there can be more than one sprite, and that each sprite has its own
scripts. Also, a sprite doesn't have to look like an arrowhead, but can
have any picture as a \emph{costume}{[}{]}\{index=``costume''\} \emph{.}
The shape of the move block is meant to remind you of a Lego™ brick; a
script is a stack of blocks{[}{]}\{index=``stack of blocks''\} . (The
word ``block'' denotes both the graphical shape on the screen and the
procedure, the action, that the block carries out.)

The number 10 in the move block above is called an \emph{input} to the
block. By clicking on the white oval, you can type any number in place
of the 10. The sample script on the previous page uses 100 as the
input{[}{]}\{index=``input''\} value. We'll see later that inputs can
have non-oval shapes that accept values other than numbers. We'll also
see that you can compute input values, instead of typing a particular
value into the oval. A block can have more than one input slot. For
example, the glide block located about halfway down the Motion palette
has three inputs.

Most command blocks have that brick shape, but some, like the repeat
block{[}{]}\{index=``repeat block''\} in the sample script, are
\emph{C‑shaped.} Most C-shaped block{[}{]}\{index=``block:C-shaped''\}
s{[}{]}\{index=``C-shaped block''\} are found in the Control
palette{[}{]}\{index=``Control palette''\} . The slot inside the C shape
is a special kind of input slot that accepts a \emph{script} as the
input.

the repeat block has two inputs: the number~4 and the script

In the sample script

C-shaped blocks can be put in a script in two ways. If you see a white
line and let go, the block will be inserted into the script like any
command block:

But if you see an orange halo and let go, the block will \emph{wrap}
around the haloed blocks:

The halo will always extend from the cursor position to the bottom of
the script:

If you want only some of those blocks, after wrapping you can grab the
first block you don't want wrapped, pull it down, and snap it under the
C-shaped block.

For ``E-shaped'' blocks with more than one C-shaped slot, only the first
slot will wrap around existing blocks in a script, and only if that
C-shaped slot is empty before wrapping. (You can fill the other slots by
dragging blocks into the desired slot.)

\section{\texorpdfstring{\hl{ }Sprites and
Parallelism}{ Sprites and Parallelism}}\label{sprites-and-parallelism}

Just below the stage is the ``new sprite{[}{]}\{index=''new sprite
button''\} '' button . Click the button to add a new sprite to the
stage. The new sprite will appear in a random position on the stage,
with a random color, but always facing to the right.

Each sprite has its own scripts. To see the scripts for a particular
sprite in the scripting area, click on the picture of that sprite in the
\emph{sprite corral}{[}{]}\{index=``sprite corral''\} in the bottom
right corner of the window. Try putting one of the following scripts in
each sprite's scripting area:

\begin{quote}
\end{quote}

When you click the green flag, you should see one sprite rotate while
the other moves back and forth. This experiment illustrates the way
different scripts can run in parallel. The turning and the moving happen
together. Parallelism{[}{]}\{index=``parallelism''\} can be seen with
multiple scripts of a single sprite also. Try this example:

\begin{quote}
\end{quote}

When you press the space key, the sprite should move forever in a
circle, because the move and turn blocks are run in parallel. (To stop
the program, click the red stop sign{[}{]}\{index=``stop sign''\} at the
right end of the tool bar.)

\subsection{Costumes and Sounds}\label{costumes-and-sounds}

To change the appearance of a sprite, paint or import a new
\emph{costume}{[}{]}\{index=``costume''\} for it. To paint a costume,
click on the Costumes tab above the scripting area, and click the paint
button . The \emph{Paint Editor} that appears is explained on page
\hyperref[the-paint-editor]{128}. There are three ways to import a
costume. First select the desired sprite in the sprite corral. Then, one
way is to click on the file icon in the tool bar , then choose the
``Costumes\ldots{}''menu item. You will see a list of costumes from the
public media library, and can choose one. The second way, for a costume
stored on your own computer, is to click on the file icon and choose the
``Import\ldots{}'' menu item. You can then select a file in any picture
format (PNG, JPEG, etc.) supported by your browser. The third way is
quicker if the file you want is visible on the desktop: Just drag the
file onto the Snap\emph{!} window. In any of these cases, the scripting
area will be replaced by something like this:

Just above this part of the window is a set of three tabs: Scripts,
Costumes, and Sounds. You'll see that the Costumes
tab{[}{]}\{index=``Costumes tab''\} is now selected. In this view, the
sprite's \emph{wardrobe}{[}{]}\{index=``wardrobe''\} \emph{,} you can
choose whether the sprite should wear its Turtle costume or its
Alonzo{[}{]}\{index=``Alonzo''\} costume. (Alonzo, the Snap\emph{!}
mascot, is named after Alonzo Church{[}{]}\{index=``Church, Alonzo''\} ,
a mathematician who invented the idea of procedures as
data{[}{]}\{index=``procedures as data''\} , the most important way in
which Snap\emph{!} is different from Scratch{[}{]}\{index=``Scratch''\}
.) You can give a sprite as many costumes as you like, and then choose
which it will wear either by clicking in its wardrobe or by using the or
block in a script. (Every costume has a number as well as a name. The
next costume block selects the next costume by number; after the
highest-numbered costume it switches to costume 1. The Turtle, costume
0, is never chosen by next costume.) The Turtle
costume{[}{]}\{index=``Turtle costume''\} is the only one that changes
color to match a change in the sprite's pen color. Protip: switches to
the \emph{previous} costume, wrapping like next costume.

In addition to its costumes, a sprite can have \emph{sounds;} the
equivalent for sounds of the sprite's wardrobe is called its
\emph{jukebox}{[}{]}\{index=``jukebox''\} \emph{.} Sound
files{[}{]}\{index=``play sound block''\} can be imported in any format
(WAV, OGG, MP3, etc.) supported by your browser. Two blocks accomplish
the task of playing sounds{[}{]}\{index=``playing sounds''\} . If you
would like a script to continue running while the sound is playing, use
the block . In contrast, you can use the block to wait for the sound's
completion before continuing the rest of the script\emph{.}

\subsection{Inter-Sprite Communication with
Broadcast}\label{inter-sprite-communication-with-broadcast}

Earlier we saw an example of two sprites moving at the same time. In a
more interesting program, though, the sprites on stage will
\emph{interact} to tell a story, play a game, etc. Often one sprite will
have to tell another sprite to run a script. Here's a simple example:

In the block, the word ``bark'' is just an arbitrary name I made up.
When you click on the downward arrowhead in that input slot, one of the
choices (the only choice, the first time) is ``new,'' which then prompts
you to enter a name for the new broadcast. When this block is run, the
chosen message is sent to \emph{every} sprite, which is why the block is
called ``broadcast.'' (But if you click the right arrow after the
message name, the block becomes , and you can change it to ~to send the
message just to one sprite.) In this program, though, only one sprite
has a script to run when that broadcast is sent, namely the dog. Because
the boy's script uses broadcast and wait{[}{]}\{index=``broadcast and
wait block''\} rather than just broadcast, the boy doesn't go on to his
next say block until the dog's script finishes. That's why the two
sprites take turns talking, instead of both talking at once. In Chapter
VII, ``Object-Oriented Programming with Sprites,'' you'll see a more
flexible way to send a message to a specific sprite using the tell and
ask blocks.

Notice, by the way, that the say block's first input slot is rectangular
rather than oval. This means the input can be any text string, not only
a number. In text input{[}{]}\{index=``text input''\} slots, a space
character is shown as a brown dot{[}{]}\{index=``brown dot''\} , so that
you can count the number of spaces between words, and in particular you
can tell the difference between an empty slot and one containing spaces.
The brown dots are \emph{not} shown on the stage if the text is
displayed.

The stage has its own scripting area. It can be selected by clicking on
the Stage icon at the left of the sprite corral. Unlike a sprite,
though, the stage can't move. Instead of costumes, it has
\emph{backgrounds:} pictures that fill the entire stage area. The
sprites appear in front of the current background. In a complicated
project, it's often convenient to use a script in the stage's scripting
area as the overall director of the action.

\section{Nesting Sprites{[}{]}\{index=``Nesting Sprites''\} : Anchors
and Parts}\label{nesting-spritesindexnesting-sprites-anchors-and-parts}

Sometimes it's desirable to make a sort of ``super-sprite'' composed of
pieces that can move together but can also be separately articulated.
The classic example is a person's body made up of a torso, limbs, and a
head. Snap\emph{!} allows one sprite to be designated as the
\emph{anchor}{[}{]}\{index=``anchor''\} of the combined shape, with
other sprites as its \emph{parts}{[}{]}\{index=``parts (of nested
sprite)''\} \emph{.} To set up sprite nesting{[}{]}\{index=``sprite
nesting''\} , drag the sprite corral icon of a \emph{part} sprite onto
the stage display (not the sprite corral icon!) of the desired
\emph{anchor} sprite. The precise place where you let go of the mouse
button will be the attachment point of the part on the anchor.

Sprite nesting is shown in the sprite corral icons of both anchors and
parts:

In this illustration, it is desired to animate Alonzo's arm. (The arm
has been colored green in this picture to make the relationship of the
two sprites clearer, but in a real project they'd be the same color,
probably.) Sprite, representing Alonzo's body, is the anchor; Sprite(2)
is the arm. The icon for the anchor shows small images of up to three
attached parts at the bottom. The icon for each part shows a small image
of the anchor in its top left corner, and a
\emph{synchronous}{[}{]}\{index=``synchronous rotation''\}
\emph{/dangling rotation}{[}{]}\{index=``dangling rotation''\}
\emph{flag} in the top right corner. In its initial, synchronous
setting, as shown above, it means that the when the anchor sprite
rotates, the part sprite also rotates as well as revolving around the
anchor. When clicked, it changes from a circular arrow to a straight
arrow, and indicates that when the anchor sprite rotates, the part
sprite revolves around it, but does not rotate, keeping its original
orientation. (The part can also be rotated separately, using its turn
blocks.) Any change in the position or size of the anchor is always
extended to its parts. Also, cloning the anchor (see Section VII. B)
will also clone all its parts.

\emph{Top: turning the part: the green arm. Bottom: turning the anchor,
with the arm synchronous (left) and dangling (right).}

\section{Reporter Blocks and
Expressions}\label{reporter-blocks-and-expressions}

So far, we've used two kinds of block{[}{]}\{index=``block:reporter''\}
s: hat blocks and command blocks. Another kind is the \emph{reporter}
block{[}{]}\{index=``Reporter block''\} , which has an oval shape: .
It's called a ``reporter'' because when it's run, instead of carrying
out an action, it reports a value that can be used as an input to
another block. If you drag a reporter into the scripting area by itself
and click on it, the value it reports will appear in a speech balloon
next to the block:

When you drag a reporter block over another block's input slot, a white
``halo{[}{]}\{index=''halo''\} '' appears around that input slot,
analogous to the white line that appears when snapping command blocks
together:

Don't drop the input over a \emph{red} halo:

That's used for a purpose explained on page
\hyperref[recursive-calls-to-multiple-input-blocks]{68}.

Here's a simple script that uses a reporter block:

Here the x position reporter provides the first input to the say block.
(The sprite's X position{[}{]}\{index=``X position''\} is its horizontal
position, how far left (negative values) or right (positive values) it
is compared to the center of the stage. Similarly, the Y
position{[}{]}\{index=``Y position''\} is measured vertically, in steps
above (positive) or below (negative) the center.)

You can do arithmetic{[}{]}\{index=``arithmetic''\} using reporters in
the Operators palette:

The round block rounds 35.3905\ldots{} to 35, and the + block adds 100
to that. (By the way, the round block is in the Operators palette, just
like +, but in this script it's a lighter color with black lettering
because Snap\emph{!} alternates light and dark versions of the palette
colors when a block is nested inside another block from the same
palette:

This aid to readability is called \emph{zebra
coloring}{[}{]}\{index=``zebra coloring''\} \emph{.}) A reporter block
with its inputs, maybe including other reporter blocks, such as , is
called an \emph{expression}{[}{]}\{index=``expression''\} \emph{.}

\section{Predicates and Conditional
Evaluation}\label{predicates-and-conditional-evaluation}

Most reporters report {[}{]}\{index=``block:predicate''\} either a
number, like , or a text string, like . A \emph{predicate} is a special
kind of reporter that always reports true or false.
Predicate{[}{]}\{index=``Predicate block''\} s have a hexagonal
shape{[}{]}\{index=``hexagonal shape''\} :

The special shape is a reminder that predicates don't generally make
sense in an input slot of blocks that are expecting a number or text.
You wouldn't say , although (as you can see from the picture)
Snap\emph{!} lets you do it if you really want. Instead, you normally
use predicates in special hexagonal input slots like this one:

The C-shaped if block{[}{]}\{index=``if block''\} runs its input script
if (and only if) the expression in its hexagonal input reports true.

A really useful block{[}{]}\{index=``repeat until block''\} in
animation{[}{]}\{index=``animation''\} s runs its input script
\emph{repeatedly} until a predicate is satisfied:

If, while working on a project, you want to omit temporarily some
commands in a script, but you don't want to forget where they belong,
you can say

Sometimes you want to take the same action whether some condition is
true or false, but with a different input value. For this purpose you
can use the \emph{reporter} if block{[}{]}\{index=``reporter if
block''\} :

The technical term for a true or false value is a
``Boolean{[}{]}\{index=''Boolean''\} '' value; it has a capital B
because it's named after a person, George Boole{[}{]}\{index=``Boole,
George''\} , who developed the mathematical theory of Boolean values.
Don't get confused; a hexagonal block is a \emph{predicate,} but the
value it reports is a \emph{Boolean.}

Another quibble about vocabulary: Many programming languages reserve the
name ``procedure{[}{]}\{index=''procedure''\} '' for Commands (that
carry out an action) and use the name ``function'' for Reporters and
Predicates. In this manual, a \emph{procedure} is any computational
capability, including those that report values and those that don't.
Commands, Reporters, and Predicates are all procedures. The words ``a
Procedure type'' are shorthand for ``Command type, Reporter type, or
Predicate type.''

If you want to put a \emph{constant} Boolean{[}{]}\{index=``Boolean
constant''\} value in a hexagonal slot instead of a predicate-based
expression, hover the mouse over the block and click on the control that
appears:

\section{Variables}\label{variables}

Try this script:

The input to the move block is an orange oval. To get it there, drag the
orange oval that's part of the for block{[}{]}\{index=``for block''\} :

The orange oval{[}{]}\{index=``orange oval''\} is a \emph{variable:} a
symbol that represents a value. (I took this screenshot before changing
the second number input to the for block from the default 10 to 200, and
before dragging in a turn block.) For runs its script input repeatedly,
just like repeat, but before each repetition it sets the
variable{[}{]}\{index=``variable''\} i to a number starting with its
first numeric input, adding 1 for each repetition, until it reaches the
second numeric input. In this case, there will be 200 repetitions, first
with i=1, then with i=2, then 3, and so on until i=200 for the final
repetition. The result is that each move draws a longer and longer line
segment, and that's why the picture you see is a kind of spiral. (If you
try again with a turn of 90 degrees instead of 92, you'll see why this
picture is called a ``squiral{[}{]}\{index=''squiral''\} .'')

The variable i is created by the for block, and it can only be used in
the script inside the block's C-slot. (By the way, if you don't like the
name i, you can change it by clicking on the orange oval without
dragging it, which will pop up a dialog window in which you can enter a
different name:

``I'' isn't a very descriptive name; you might prefer ``length'' to
indicate its purpose in the script. ``I'' is traditional because
mathematicians tend to use letters between i and n to represent integer
values, but in programming languages we don't have to restrict ourselves
to single-letter variable names.)

\subsection{Global Variable{[}{]}\{index=``variable:global''\}
s{[}{]}}\label{global-variableindexvariableglobal-s}

You can create variables ``by hand'' that aren't limited to being used
within a single block. At the top of the Variables palette, click the
``Make a variable{[}{]}\{index=''Make a variable''\} '' button:

This will bring up a dialog window in which you can give your variable a
name:

The dialog also gives you a choice to make the variable available to all
sprites (which is almost always what you want) or to make it visible
only in the current sprite{[}{]}\{index=``sprite-local variable''\} .
You'd do that if you're going to give several sprites individual
variables \emph{with the same name,} so that you can share a script
between sprites (by dragging it from the current sprite's scripting area
to the picture of another sprite in the sprite corral), and the
different sprites will do slightly different things when running that
script because each has a different value for that variable name.

If you give your variable the name ``name'' then the Variables palette
will look like this:

There's now a ``Delete a variable{[}{]}\{index=''Delete a variable''\}
'' button, and there's an orange oval with the variable name in it, just
like the orange oval in the for block. You can drag the variable into
any script in the scripting area. Next to the oval is a checkbox,
initially checked. When it's checked, you'll also see a \emph{variable
watcher}{[}{]}\{index=``variable watcher''\} on the stage:

When you give the variable a value, the orange box in its
watcher{[}{]}\{index=``watcher''\} will display the value.

How \emph{do} you give it a value? You use the set
block{[}{]}\{index=``set block''\} :

Note that you \emph{don't} drag the variable's oval into the set block!
You click on the downarrow in the first input slot, and you get a menu
of all the available variable names.

If you do choose ``For this sprite only{[}{]}\{index=''For this sprite
only''\} '' when creating a variable, its block in the palette looks
like this:

The \emph{location}-pin{[}{]}\{index=``location-pin''\} icon is a bit of
a pun on a sprite-\emph{local}
variable{[}{]}\{index=``variable:sprite-local''\} . It's shown only in
the palette.

\subsection{Script Variables}\label{script-variables}

In the name example above, our project is going to carry on an
interaction{[}{]}\{index=``interaction''\} with the user, and we want to
remember their name throughout the project. That's a good example of a
situation in which a \emph{global} variable{[}{]}\{index=``global
variable''\} (the kind you make with the ``Make a variable'' button) is
appropriate. Another common example is a variable called ``score'' in a
game project. But sometimes you only need a
variable{[}{]}\{index=``variable:script-local''\} temporarily, during
the running of a particular script. In that case you can use the script
variables block{[}{]}\{index=``script variables block''\} to make the
variable:

As in the for block, you can click on an orange oval in the script
variables block without dragging to change its name. You can also make
more than one temporary variable by clicking on the right arrow at the
end of the block to add another variable oval:

\subsection{Renaming variables{[}{]}}\label{renaming-variables}

There are several reasons why you might want to change the name of a
variable:

\begin{enumerate}
\def\labelenumi{\arabic{enumi}.}
\item
  It has a default name, such as the ``a'' in script variables or the
  ``i'' in for.
\item
  It conflicts with another name, such as a global variable, that you
  want to use in the same script.
\item
  You just decide a different name would be more self-documenting.
\end{enumerate}

In the first and third case, you probably want to change the name
everywhere it appears in that script, or even in all scripts. In the
second case, if you've already used both variables in the script before
realizing that they have the same name, you'll want to look at each
instance separately to decide which ones to rename. Both of these
operations are possible by right-clicking or control-clicking on a
variable oval.

If you right-click on an orange oval in a context in which the variable
is \emph{used,} then you are able to rename just that one orange oval:

If you right-click on the place where the variable is \emph{defined} (a
script variables block, the orange oval for a global variable in the
Variables palette, or an orange oval that's built into a block such as
the ``i'' in for), then you are given two renaming options, ``rename''
and ``rename all.'' If you choose ``rename,'' then the name is changed
only in that one orange oval, as in the previous case:

But if you choose ``rename all,'' then the name will be changed
throughout the scope of the variable (the script for a script variable,
or everywhere for a global variable):

\subsection{Transient variable{[}{]}\{index=``variable:transient''\}
s}\label{transient-variableindexvariabletransient-s}

So far we've talked about variables with numeric values, or with short
text strings such as someone's name. But there's no limit to the amount
of information you can put in a variable; in Chapter IV you'll see how
to use \emph{lists} to collect many values in one data structure, and in
Chapter VIII you'll see how to read information from web sites. When you
use these capabilities, your project may take up a lot of
memory{[}{]}\{index=``memory''\} in the computer. If you get close to
the amount of memory available to Snap\emph{!}, then it may become
impossible to save your project. (Extra space is needed temporarily to
convert from Snap\emph{!} 's internal representation to the form in
which projects are exported or saved.) If your program reads a lot of
data from the outside world that will still be available when you use it
next, you might want to have values containing a lot of data removed
from memory before saving the project. To do this, right-click or
control-click on the orange oval in the Variables palette, to see this
menu:

You already know about the rename options, and help\ldots{} displays a
help screen about variables in general. Here we're interested in the
check box next to transient. If you check it, this variable's value will
not be saved when you save your project. Of course, you'll have to
ensure that when your project is loaded, it recreates the needed value
and sets the variable to it.

\section{Debugging{[}{]}}\label{debugging}

Snap\emph{!} provides several tools to help you debug a program. They
center around the idea of \emph{pausing} the running of a script partway
through, so that you can examine the values of variables.

\subsection{The pause button}\label{the-pause-button}

The simplest way to pause a program is manually, by clicking the pause
button{[}{]}\{index=``button:pause''\} in the top right corner of the
window. While the program is paused, you can run other scripts by
clicking on them, show variables on stage with the checkbox next to the
variable in the Variables palette or with the show variable
block{[}{]}\{index=``hide variable block''\} , and do all the other
things you can generally do, including modifying the paused scripts by
adding or removing blocks. The button changes shape to and clicking it
again resumes the paused scripts.

\subsection{Breakpoint{[}{]}\{index=``breakpoint''\} s: the pause all
block{[}{]}}\label{breakpointindexbreakpoint-s-the-pause-all-block}

\phantomsection\label{pause_all}{}The pause button is great if your
program seems to be in an infinite loop, but more often you'll want to
set a \emph{breakpoint,} a particular point in a script at which you
want to pause. The block, near the bottom of the Control palette, can be
inserted in a script to pause when it is run. So, for example, if your
program is getting an error message in a particular block, you could use
pause all just before that block to look at the values of variables just
before the error happens.

The pause all block turns bright cyan while paused. Also, during the
pause, you can right-click on a running script and the menu that appears
will give you the option to show watchers for temporary variables of the
script:

But what if the block with the error is run many times in a loop, and it
only errors when a particular condition is true---say, the value of some
variable is negative, which shouldn't ever happen. In the iteration
library (see page \hyperref[libraries-1]{25} for more about how to use
libraries) is a breakpoint block that lets you set a \emph{conditional}
breakpoint, and automatically display the relevant variables before
pausing. Here's a sample use of it:

(In this contrived example, variable zot comes from outside the script
but is relevant to its behavior.) When you continue (with the pause
button), the temporary variable watchers are removed by this breakpoint
block before resuming the script. The breakpoint block isn't magic; you
could alternatively just put a pause all inside an if.{[}2{]}

\subsection{Visible stepping}\label{visible-stepping}

Sometimes you're not exactly sure where the error is, or you don't
understand how the program got there. To understand better, you'd like
to watch the program as it runs, at human speed rather than at computer
speed. You can do this by clicking the \emph{visible stepping
bu}{[}{]}\{index=``button:visible stepping''\}
\emph{tton}{[}{]}\{index=``visible stepping button''\} ( ), before
running a script or while the script is paused. The button will light up
( ) and a speed control slider will appear in the toolbar. When you
start or continue the script, its blocks and input slots will light up
cyan one at a time:

In this simple example, the inputs to the blocks are constant values,
but if an input were a more complicated expression involving several
reporter blocks, each of those would light up as they are called. Note
that the input to a block is evaluated before the block itself is
called, so, for example, the 100 lights up before the move.

\textbf{. . .}

The speed of stepping is controlled by the
slider{[}{]}\{index=``slider:stepping speed''\} . If you move the slider
all the way to the left, the speed is zero, the pause button turns into
a step button , and the script takes a single step each time you push
it. The name for this is \emph{single stepping}{[}{]}\{index=``single
stepping''\} \emph{.}

If several scripts that are visible in the scripting area are running at
the same time, all of them are stepped in parallel. However, consider
the case of two repeat loops with different numbers of blocks. While not
stepping, each script goes through a complete cycle of its loop in each
display cycle, despite the difference in the length of a cycle. In order
to ensure that the visible result of a program on the stage is the same
when stepped as when not stepped, the shorter script will wait at the
bottom of its loop for the longer script to catch up.

When we talk about custom blocks in Chapter III, we'll have more to say
about visible stepping as it affects those blocks.

\section{Etcetera}\label{etcetera}

This manual doesn't explain every block in detail. There are many more
motion blocks, sound blocks, costume and graphics effects blocks, and so
on. You can learn what they all do by experimentation, and also by
reading the ``help screens'' that you can get by right-clicking or
control-clicking a block and selecting ``help\ldots{}'' from the menu
that appears. If you forget what palette (color) a block is, but you
remember at least part of its name, type control-F and enter the name in
the text block that appears in the palette area.

Here are the primitive blocks that don't exist in Scratch:

reports{[}{]}\{index=``pen trails block''\} a{[}{]}\{index=``pen vectors
block''\} new costume consisting of everything that's drawn on the stage
by any sprite. Right-clicking the block in the scripting area gives the
option to change it to if vector logging is enabled. See page
\hyperref[logpenvectors]{116}.

Print characters{[}{]}\{index=``write block''\} in the given point size
on the stage, at the sprite's position and in its direction. The sprite
moves to the end of the text. (That's not always what you want, but you
can save the sprite's position before using it, and sometimes you need
to know how big the text turned out to be, in turtle steps.) If the pen
is down, the text will be underlined.

Takes a sprite as input. Like stamp except that the costume is stamped
onto the selected sprite instead of onto the stage. (Does nothing if the
current sprite doesn't overlap the chosen sprite.)

Takes a sprite as input. Erases from that sprite's costume the area that
overlaps with the current sprite's costume. (Does not affect the costume
in the chosen sprite's wardrobe, only the copy currently visible.)

Runs{[}{]}\{index=``warp block''\} only this script

until finished. In the Control palette even though it's gray.

See page \hyperref[generic_when]{6}. See page \hyperref[pause_all]{17}.

Reporter version of the if/else primitive command
block{[}{]}\{index=``if else reporter block''\} . Only one of the two
branches is evaluated, depending on the value of the first input.

Looping block like repeat but {[}{]}\{index=``for block''\} with an
index variable{[}{]}\{index=``index variable''\} .

Declare local variables{[}{]}\{index=``local variables''\} in a
script.{[}{]}\{index=``script variables block''\}

See page \hyperref[url]{91}.

reports the value of a graphics effect{[}{]}\{index=``graphics
effect''\} .

Constant true{[}{]}\{index=``true block''\} or
false{[}{]}\{index=``false block''\} value. See page
\hyperref[predicates-and-conditional-evaluation]{12}.

Create a primitive using JavaScript{[}{]}\{index=``JavaScript''\} .
(This block is disabled by default; the user must check ``Javascript
extensions'' in the setting menu \emph{each time} a project is
loaded.){[}{]}\{index=``pen down? block''\}

The at block{[}{]}\{index=``at block''\} lets you examine the screen
pixel{[}{]}\{index=``screen pixel''\} directly behind the rotation
center of a sprite, the mouse, or an arbitrary (x,y) coordinate pair
dropped onto the second menu slot. The first five items of the left menu
let you examine the color visible at the position. (The ``RGBA''
option{[}{]}\{index=``RGBA option''\} reports a list.) The ``sprites''
option reports a list of all sprites, including this one, any point of
which overlaps this sprite's rotation center (behind or in front). This
is a hyperblock with respect to its second input.

Checks the {[}{]}\{index=``is \_ a \_ ? block''\}
data{[}{]}\{index=``stage blocks''\} type{[}{]}\{index=``type''\} of a
value.

\textbf{Blocks only for the Stage:}

Get or set selected global flags.

{[}{]}\{index=``set flag block''\} Turn the{[}{]}\{index=``split
block''\} text into a list, using the second input as the delimiter
between items. The default delimiter, indicated by the brown dot in the
input slot, is a single space character. ``Letter'' puts each character
of the text in its own list item. ``Word'' puts each word in an item.
({[}{]}\{index=``whitespace''\} Words are separated by any number of
consecutive space, tab, carriage return, or newline characters.)
``Line'' is a newline character{[}{]}\{index=``newline character''\}
(0xa); ``tab'' is a tab character{[}{]}\{index=``tab character''\}
(0x9); ``cr'' is a carriage return{[}{]}\{index=``carriage return
character''\} (0xd). ``Csv''{[}{]}\{index=``CSV format''\} and
``json''{[}{]}\{index=``JSON format''\} split formatted text into lists
of lists; see page \hyperref[comma-separated-values]{54}. ``Blocks''
takes a script as the first input, reporting a list structure
representing the structure of the script. See Chapter XI.

For lists, {[}{]}\{index=``identical to''\} reports true only if its two
input values are the very same list, so changing an item in one of them
is visible in the other. (For =, lists that look the same are the same.)
For text strings, uses case-sensitive comparison, unlike =, which is
case-independent.

These \emph{hidden} blocks can be found with the relabel
option{[}{]}\{index=``relabel option''\} of any dyadic arithmetic block.
They're hidden partly because writing them in Snap\emph{!} is a good,
pretty easy programming exercise. Note: the two inputs to
atan2{[}{]}\{index=``max block''\} are Δ\emph{x} and Δ\emph{y} in that
order, because we measure angles clockwise from north. Max and min are
\emph{variadic;} by clicking the arrowhead, you can provide additional
inputs.

Similarly, these{[}{]}\{index=``≤ block''\} hidden predicates can be
found by relabeling the relational predicates.

\textbf{Metaprogramming (see Chapter XI.} \textbf{, page
\hyperref[metaprogramming]{101})}

These blocks support \emph{metaprogramming,} which means manipulating
blocks and scripts as data. This is not the same as manipulating
procedures (see Chapter VI. ), which are what the blocks \emph{mean;} in
metaprogramming the actual blocks, what you see on the screen, are the
data. This capability is new in version 8.0.

\textbf{First class list blocks (see Chapter IV, page
\hyperref[first-class-lists]{46}):}

Numbers from{[}{]}\{index=``numbers from block''\} {[}{]}\{index=``for
each block''\} will count up or down.

The script input to for each can refer to an

item of the list with the item variable.

**\\
** report{[}{]}\{index=``position block''\} the sprite or mouse position
as a two-item vector (x,y).

\textbf{First class procedure blocks (see Chapter VI, page
\hyperref[procedures-as-data]{65}):}

\textbf{First class continuation blocks (see Chapter X, page
\hyperref[continuations]{93}):}

\textbf{First class sprite, costume, and sound blocks (see Chapter VII,
page \hyperref[object-oriented-programming-with-sprites]{73}):}

Object is a hyperblock.

\textbf{Scenes:}

The major new feature of version 7.0 is \emph{scenes:} A project can
include within it sub-projects, called scenes, each with its own stage,
sprites, scripts, and so on. This block makes another scene active,
replacing the current one.

Nothing is automatically shared between scenes: no sprites, no blocks,
no variables. But the old scene can send a message to the new one, to
start it running, with optional payload as in
broadcast{[}{]}\{index=``broadcast block''\} (page
\hyperref[broadcast]{23}).

In particular, you can say

\begin{quote}
if the new scene expects to be started with a green flag signal.
\end{quote}

**\\
These aren't new blocks but they have a new feature:**

These accept two-item (x,y) {[}{]}\{index=``points as inputs''\}
lists{[}{]}\{index=``two-item (x,y) lists''\} as input, and have
extended menus (also including other sprites): {[}{]}\{index=``to
block''\}

``Center'' means the center of the stage{[}{]}\{index=``center of the
stage''\} , the point at (0,0). ``Direction'' is in the point in
direction sense, the direction that would leave this sprite pointing
toward another sprite, the mouse, or the center. ``Ray length'' is the
distance from the center of this sprite to the nearest point on the
other sprite, in the current direction.

The stop block{[}{]}\{index=``stop block''\} has two extra menu choices.
Stop this block is used inside the definition of a custom block to stop
just this invocation of this custom block and continue the script that
called it. Stop all but this script is good at the end of a game to stop
all the game pieces from moving around, but keep running this script to
provide the user's final score. The last two menu choices add a tab at
the bottom of the block because the current script can continue after
it.

The new ``pen trails'' option is true if the sprite is touching any
drawn or stamped ink on the stage. Also,
touching{[}{]}\{index=``touching block''\} will not detect hidden
sprites, but a hidden sprite can use it to detect visible sprites.

The video block{[}{]}\{index=``video block''\} has a snap
option{[}{]}\{index=``snap option''\} that takes a snapshot and reports
it as a costume. It is hyperized with respect to its second input.

The ``neg'' option{[}{]}\{index=``neg option''\} is a
monadic{[}{]}\{index=``of block (operators)''\} {[}{]}\{index=``length
of text block''\} negation operator{[}{]}\{index=``negation operator''\}
, equivalent to . ``lg'' is log2. ``id'' is the identity function, which
reports its input. ``sign'' reports 1 for positive input, 0 for zero
input, or -1 for negative input. {[}{]}\{index=``set background
block''\}

name changed to clarify that it's different from

+ and × are \emph{variadic:} they take two or more inputs. If you drop a
list on the arrowheads, the block name changes to sum or product.

I

Extended {[}{]}\{index=``when I am block''\} mouse interaction events,
sensing clicking, dragging, hovering, etc. The ``stopped'' option
triggers when all scripts are stopped, as with the stop button; it is
useful for robots whose hardware interface must be told to turn off
motors. A when I am stopped script{[}{]}\{index=``when I am stopped
script''\} can run only for a limited time.

\phantomsection\label{broadcast}{}Extended
broadcast{[}{]}\{index=``broadcast block''\} : Click the right arrowhead
to direct the message to a single sprite or the stage. Click again to
add any value as a payload to the message.

Extended when I receive{[}{]}\{index=``when I receive block''\} : Click
the right arrowhead to expose a script variable (click on it to change
its name, like any script variable) that will be set to the data of a
matching broadcast. If the first input is set to ``any message,'' then
the data variable will be set to the message, if no payload is included
with the broadcast, or to a two-item list containing the message and the
payload.

If the input is set to ``any key,'' then a right arrowhead appears:

\begin{quote}
and if you click it, a script variable key is created whose value is the
key that was pressed. (If the key is one that' represented in the input
menu by a word or phrase, e.g., ``enter'' or ``up arrow,'' then the
value of key will be that word or phrase, \emph{except for} the space
character, which is represented as itself in key.)\\
\phantomsection\label{ask_lists}{}
\end{quote}

The RGB(A) {[}{]}\{index=``set pen block''\} option accepts a single
number, which is a grayscale value 0-255; a two-number list, grayscale
plus opacity 0-255; a three-item RGB list, or a four-item RGBA list.

These ask features{[}{]}\{index=``ask and wait block''\} and more in the
Menus library.

The of block{[}{]}\{index=``of block (sensing)''\} has an extended menu
of attributes of a sprite. Position reports an (x,y) vector. Size
reports the percentage of normal size, as controlled by the set size
block in the Looks category. Left, right, etc. report the stage
coordinates of the corresponding edge of the sprite's bounding box.
Variables reports a list of the names of all variables in scope (global,
sprite-local, and script variables if the right input is a script.

\section{Libraries}\label{libraries}

\phantomsection\label{libraries-1}{}There are several collections of
useful procedures that aren't Snap\emph{!} primitives, but are provided
as libraries. To include a library in your project, choose the
Libraries\ldots{} option{[}{]}\{index=``Libraries\ldots{} option''\} in
the file ( ) menu.

The library menu is divided into five broad categories. The first is,
broadly, utilities: blocks that might well be primitives. They might be
useful in all kinds of projects.

The second category is blocks related to media computation: ones that
help in dealing with costumes and sounds (a/k/a Jens libraries). There
is some overlap with ``big data'' libraries, for dealing with large
lists of lists.

The third category is, roughly, specific to non-media applications
(a/k/a Brian libraries). Three of them are imports from other
programming languages: words and sentences from Logo, array functions
from APL, and streams from Scheme. Most of the others are to meet the
needs of the BJC curriculum.

The fourth category is major packages (extensions) provided by users.

The fifth category provides support for hardware devices such as robots,
through general interfaces, replacing specific hardware libraries in
versions before 7.0.

When you click on the one-line description of a library, you are shown
the actual blocks in the library and a longer explanation of its
purpose. You can browse the libraries to find one that will satisfy your
needs.

The libraries and their contents may change, but as of this writing the
list library{[}{]}\{index=``list library''\} has these blocks:

(The lightning bolt{[}{]}\{index=``lightning bolt symbol''\} before the
name in several of these blocks means that they use compiled HOFs or
JavaScript primitives to achieve optimal speed. They are officially
considered experimental.) Remove duplicates from{[}{]}\{index=``remove
duplicates from block''\} reports a list in which no two items are
equal. The sort{[}{]}\{index=``sort block''\} block takes a list and a
two-input comparison predicate, such as \textless, and reports a list
with the items sorted according to that comparison. The assoc
block{[}{]}\{index=``assoc block''\} is for looking up a key in an
\emph{association list:} a list of two-item lists. In each two-item
list, the first is a \emph{key} and the second is a \emph{value.} The
inputs are a key and an association list; the block reports the first
key-value pair whose key is equal to the input key.

For each item{[}{]}\{index=``for each item block''\} is a variant of the
primitive version that provides a \# variable{[}{]}\{index=``\#
variable''\} containing the position in the input list of the currently
considered item. Multimap {[}{]}\{index=``multimap block''\} is a
version of map that allows multiple list inputs, in which case the
mapping function must take as many inputs as there are lists; it will be
called with all the first items, all the second items, and so on. Zip
takes any number of lists as inputs; it reports a list of lists: all the
first items, all the second items, and so on. The no-name identity
function reports its input.

Sentence{[}{]}\{index=``sentence block''\} and
sentence➔list{[}{]}\{index=``sentence➔list block''\} are borrowed from
the word and sentence library (page \hyperref[wordsent]{27}) to serve as
a variant of append that accepts non-lists as inputs. Printable takes a
list structure of any depth as input and reports a compact
representation of the list as a text string.

The iteration, composition library{[}{]}\{index=``iteration library''\}
has these blocks:

Catch{[}{]}\{index=``catch block''\} and throw{[}{]}\{index=``throw
block''\} provide a nonlocal exit facility. You can drag the tag from a
catch block to a throw inside its C-slot, and the throw will then jump
directly out to the matching catch without doing anything in between.

If do and pause all{[}{]}\{index=``if do and pause all block''\} is for
setting a breakpoint while debugging code. The idea is to put show
variable blocks for local variables in the C-slot; the watchers will be
deleted when the user continues from the pause.

Ignore{[}{]}\{index=``ignore block''\} is used when you need to call a
reporter but you don't care about the value it reports. (For example,
you are writing a script to time how long the reporter takes.)

The cascade{[}{]}\{index=``cascade blocks''\} blocks take an initial
value and call a function repeatedly on that value,
\emph{f}(\emph{f}(\emph{f}(\emph{f}\ldots(\emph{x})))).

The compose{[}{]}\{index=``compose block''\} block takes two functions
and reports the function \emph{f}(\emph{g}(\emph{x})).

The first three repeat blocks{[}{]}\{index=``repeat blocks''\} are
variants of the primitive repeat until block, giving all four
combinations of whether the first test happens before or after the first
repetition, and whether the condition must be true or false to continue
repeating. The last repeat block is like the repeat primitive, but makes
the number of repetitions so far available to the repeated script. The
next two blocks are variations on for{[}{]}\{index=``for block''\} : the
first allows an explicit step instead of using ±1, and the second allows
any values, not just numbers; inside the script you say

replacing the grey block in the picture with an expression to give the
next desired value for the loop index. Pipe allows reordering a nested
composition with a left-to-right one:

The stream library{[}{]}\{index=``stream library''\} has these blocks:

\emph{Streams} are a special kind of list whose items are not computed
until they are needed. This makes certain computations more efficient,
and also allows the creation of lists with infinitely many items, such
as a list of all the positive integers. The first five blocks are stream
versions of the list blocks in front of{[}{]}\{index=``in front of
stream block''\} , item 1 of{[}{]}\{index=``item 1 of stream block''\} ,
all but first of{[}{]}\{index=``all but first of stream block''\} ,
map{[}{]}\{index=``map over stream block''\} , and keep. Show
stream{[}{]}\{index=``show stream block''\} takes a stream and a number
as inputs, and reports an ordinary list of the first \emph{n} items of
the stream. Stream{[}{]}\{index=``Stream block''\} is like the primitive
list; it makes a finite stream from explicit items.
Sieve{[}{]}\{index=``sieve block''\} is an example block that takes as
input the stream of integers starting with 2 and reports the stream of
all the prime numbers. Stream with numbers from is{[}{]}\{index=``Stream
with numbers from block''\} like the numbers from block for lists,
except that there is no endpoint; it reports an infinite stream of
numbers.

The \phantomsection\label{wordsent}{}word and sentence
library{[}{]}\{index=``sentence library''\} has these blocks:

This library has the goal of recreating the Logo approach to handling
text: A text isn't best viewed as a string of characters, but rather as
a \emph{sentence}, made of \emph{words,} each of which is a string of
\emph{letters.} With a few specialized exceptions, this is why people
put text into computers: The text is sentences of natural (i.e., human)
language, and the emphasis is on words as constitutive of sentences. You
barely notice the letters of the words, and you don't notice the spaces
between them at all, unless you're proof-reading. (Even then:
Proofreading is \emph{diffciult,} because you see what you expect to
see, what will make the snetence make sense, rather than the misspelling
in front of of your eyes.) Internally, Logo stores a sentence as a list
of words, and a word as a string of letters.{[}{]}\{index=``all but
first blocks''\}

Inexplicably, the designers of Scratch chose to abandon that tradition,
and to focus on the representation of text as a string of characters.
The one vestige of the Logo tradition{[}{]}\{index=``Logo tradition''\}
from which Scratch developed is the block named letter (1) of
(world){[}{]}\{index=``letter (1) of (world) block''\} , rather than
character (1) of (world). Snap\emph{!} inherits its text handling from
Scratch.

In Logo, the visual representation of a sentence{[}{]}\{index=``visual
representation of a sentence''\} (a list of words) looks like a natural
language sentence: a string of words with spaces between them. In
Snap\emph{!}, the visual representation of a list looks nothing at all
like natural language. On the other hand, representing a sentence as a
string means that the program must continually re-parse the text on
every operation, looking for spaces, treating multiple consecutive
spaces as one, and so on. Also, it's more convenient to treat a sentence
as a list of words rather than a string of words because in the former
case you can use the higher order functions map, keep, and combine on
them. This library attempts to be agnostic as to the internal
representation of sentences. The sentence selectors accept any
combination of lists and strings; there are two sentence constructors,
one to make a string (join words) and one to make a list (sentence).

The selector names come from Logo, and should be self-explanatory.
However, because in a block language you don't have to type the block
name, instead of the terse butfirst or the cryptic bf we spell out ``all
but first of'' and include ``word'' or ``sentence'' to indicate the
intended domain. There's no first letter of block because letter 1 of
serves that need. Join words (the sentence-as-string constructor) is
like the primitive join except that it puts a space in the reported
value between each of the inputs. Sentence (the List-colored
sentence-as-list constructor) accepts any number of inputs, which can be
words, sentences-as-lists, or sentences-as-strings. (If inputs are lists
of lists, only one level of flattening is done.) Sentence reports a list
of words; there will be no empty words or words containing spaces. The
four blocks with right-arrows in their names{[}{]}\{index=``list ➔
sentence block''\} convert back and forth between text strings (words or
sentences) and lists. (Splitting a word into a list of letters is
unusual unless you're a linguist investigating orthography.)
Printable{[}{]}\{index=``printable block''\} takes a list (including a
deep list) of words as input and reports a text string in which
parentheses are used to show the structure, as in Lisp/Scheme.

The pixels library{[}{]}\{index=``pixels library''\} has one block:

Costumes are first class data in Snap\emph{!}. Most of the processing of
costume data is done by primitive blocks in the Looks category. (See
page \hyperref[media-computation-with-costumes]{79}.) This library
provides snap{[}{]}\{index=``snap block''\} , which takes a picture
using your computer's camera and reports it as a costume.

The bar charts library{[}{]}\{index=``bar charts library''\} has these
blocks:

Bar chart{[}{]}\{index=``bar chart block''\} takes a table (typically
from a CSV data set) as input and reports a summary of the table grouped
by the field in the specified column number. The remaining three inputs
are used only if the field values are numbers, in which case they can be
grouped into buckets (e.g., decades, centuries, etc.). Those inputs
specify the smallest and largest values of interest and, most
importantly, the width of a bucket (10 for decades, 100 for centuries).
If the field isn't numeric, leave these three inputs empty or set them
to zero. Each string value of the field is its own bucket, and they
appear sorted alphabetically.

Bar chart reports a new table with three columns. The first column
contains the bucket name or smallest number. The second column contains
a nonnegative integer that says how many records in the input table fall
into this bucket. The third column is a subtable containing the actual
records from the original table that fall into the bucket. Plot bar
chart{[}{]}\{index=``plot bar chart block''\} takes the table reported
by bar chart and graphs it on the stage, with axes labelled
appropriately. The remaining blocks are helpers for those.

If your buckets aren't of constant width, or you want to group by some
function of more than one field, load the ``Frequency Distribution
Analysis'' library instead.

The multi-branched conditional library{[}{]}\{index=``conditional
library:multiple-branch''\} has these blocks:

The catch and throw blocks duplicate ones in the iteration library, and
are included because they are used to implement the others. The cases
block{[}{]}\{index=``cases block''\} sets up a multi-branch conditional,
similar to cond in Lisp{[}{]}\{index=``cond in Lisp''\} or switch in
C{[}{]}\{index=``switch in C''\} -family languages. The first branch is
built into the cases block; it consists of a Boolean test in the first
hexagonal slot and an action script, in the C-slot, to be run if the
test reports true. The remaining branches go in the variadic hexagonal
input at the end; each branch consists of an else if
block{[}{]}\{index=``else if block''\} , which includes the Boolean test
and the corresponding action script, except possibly for the last
branch, which can use the unconditional else block{[}{]}\{index=``else
block''\} . As in other languages, once a branch succeeds, no other
branches are tested.

\subsection{}\label{section}

The variadic library{[}{]}\{index=``variadic library''\} has these
blocks:

These are {[}{]}\{index=``sum block''\} versions{[}{]}\{index=``all of
block''\} {[}{]}\{index=``any of block''\} of the associative operators
and, and or that take any number of inputs instead of exactly two
inputs. As with any variadic input, you can also drop a list of values
onto the arrowheads instead of providing the inputs one at a time As of
version 8.0, the arithmetic operators sum, product, minimum, and maximum
are no longer included, because the primitive operators +. ×, min, and
max are themselves variadic.

The colors and crayons library{[}{]}\{index=``colors library''\} has
these blocks:

It is intended as a more powerful replacement for the primitive set pen
block{[}{]}\{index=``set pen block''\} , including \emph{first class
color} support; HSL color{[}{]}\{index=``HSL color''\} specification as
a better alternative to the HSV that Snap\emph{!} inherits from
JavaScript; a ``fair hue{[}{]}\{index=''fair hue''\} '' scale that
compensates for the eye's grouping a wide range of light frequencies as
green while labelling mere slivers as orange or yellow; the X11/W3C
standard color names{[}{]}\{index=``X11/W3C color names''\} ; RGB in
hexadecimal; a linear color scale (as in the old days, but better) based
on fair hues and including shades (darker colors) and grayscale. Another
linear scale is a curated set of 100 ``crayons,'' explained further on
the next page.

Colors are created by the block (for direct user selection), the color
from block{[}{]}\{index=``color from block''\} to specify a color
numerically, or by , which reports the color currently in use by the
pen. The from color block{[}{]}\{index=``from color block''\} reports
names or numbers associated with a color:

Colors can be created from other colors: {[}{]}\{index=``mix colors
block''\}

The three blocks with pen in their names are improved versions of
primitive Pen blocks. In principle set pen{[}{]}\{index=``set pen
block''\} , for example, could be implemented using a (hypothetical) set
pen to color composed with the color from block, but in fact set pen
benefits from knowing how the pen color was set in its previous
invocation, so it's implemented separately from color from. Details in
Appendix A.

The recommended way to choose a color is from one of two linear scales:
the continuous \emph{color numbers} and the discrete \emph{crayons:}

Color numbers{[}{]}\{index=``color numbers''\} are based on \emph{fair
hues,} a modification of the spectrum (rainbow) hue scale that devotes
less space to green and more to orange and yellow, as well as promoting
brown to a real color. Here is the normal hue scale, for reference:

Here is the fair hue scale:

Here is the color number scale:

(The picture is wider so that pure spectral colors line up with the fair
hue scale.)

And here are the 100 crayons{[}{]}\{index=``crayons''\} :

The color from block, for example, provides different pulldown menus
depending on which scale you choose:

You can also type the crayon name: There are many scales:

The white slot at the end of some of the blocks has two purposes. It can
be used to add a transparency{[}{]}\{index=``transparency''\} to a color
(0=opaque, 100=transparent):

or it can be expanded to enter three or four numbers for a vector
directly into the block, so these are equivalent:

But note that a transparency number in a four-number RGBA vector is on
the scale 255=opaque, 0=transparent, so the following are \emph{not}
equivalent:

Set pen crayon to provides the equivalent of a box of 100 crayons. They
are divided into color groups, so the menu in the set pen crayon to
input{[}{]}\{index=``set pen to crayon block''\} slot has submenus. The
colors are chosen so that starting from crayon 0, change pen crayon by
10 rotates through an interesting, basic set of ten colors:

Using change pen crayon by 5 instead gives ten more colors, for a total
of 20:

(Why didn't we use the colors of the 100-crayon Crayola™ box? A few
reasons, one of which is that some Crayola colors aren't representable
on RGB screens. Some year when you have nothing else to do, look up
``color space'' on Wikipedia. Also ``crayon.'' Oh, it's deliberate that
change pen crayon by 5 doesn't include white, since that's the usual
stage background color. White is crayon 14.) Note that crayon 43 is
``Variables''; all the standard block colors are included.

See Appendix A (page \hyperref[crayons-and-color-numbers]{139}) for more
information.

The \textbf{crayon library}{[}{]}\{index=``crayon library''\} has only
the crayon features, without the rest of the colors package.

The catch errors library{[}{]}\{index=``catch errors library''\} has
these blocks:

The safely try block{[}{]}\{index=``safely try block''\} allows you to
handle errors that happen when your program is run within the program,
instead of stopping the script with a red halo and an obscure error
message. The block runs the script in its first C-slot. If it finishes
without an error, nothing else happens. But if an error happens, the
code in the second C-slot is run. While that second script is running,
the variable contains the text of the error message that would have been
displayed if you weren't catching the error. The error
block{[}{]}\{index=``error block''\} is sort of the opposite: it lets
your program \emph{generate} an error message, which will be displayed
with a red halo unless it is caught by safely try. Safely try reporting
is the reporter version of safely try.

The text costumes library{[}{]}\{index=``text costume library''\} has
only two blocks:

Costume from text reports a costume{[}{]}\{index=``costume from text
block''\} that can be used with the switch to costume block to make a
button:

Costume with background{[}{]}\{index=``costume with background block''\}
reports a costume made from another costume by coloring its background,
taking a color input like the set pen color to RGB(A) block and a number
of turtle steps of padding around the original costume. These two blocks
work together to make even better buttons:

The text to speech library{[}{]}\{index=``speech synthesis library''\}
has these blocks:

This library interfaces with a capability in up-to-date browsers, so it
might not work for you.{[}{]}\{index=``speak block''\} It works best if
the accent matches the text!

The parallelization library{[}{]}\{index=``parallelization library''\}
contains these blocks:

The two do in parallel block{[}{]}\{index=``do in parallel block''\} s
take any number of scripts as inputs. Those scripts will be run in
parallel, like ordinary independent scripts in the scripting area. The
and wait version waits until all of those scripts have finished before
continuing the script below the block.

The create variables library{[}{]}\{index=``variables library''\} has
these blocks: {[}{]}\{index=``does var exist block''\}

These blocks allow a program to perform the same operation as the

button, making global, sprite local, or script variables, but allowing
the program to compute the variable name(s). It can also set and find
the values of these variables, show and hide their stage watchers,
delete them, and find out if they already exist.

The getters and setters library{[}{]}\{index=``getter/setter library''\}
has these blocks:

The purpose of this library is to allow program access to the settings
controlled by user interface elements, such as the settings menu. The
setting block{[}{]}\{index=``setting block''\} reports a setting; the
set flag block{[}{]}\{index=``set flag block''\} sets yes-or-no options
that have checkboxes in the user interface, while the set value
block{[}{]}\{index=``set value block''\} controls settings with numeric
or text values, such as project name.

Certain settings are ordinarily remembered on a per-user basis, such as
the ``zoom blocks'' value. But when these settings are changed by this
library, the change is in effect only while the project using the
library is loaded. No permanent changes are made. Note: this library has
not been converted for version 7.0, so you'll have to enable Javascript
extensions to use it.

The bignums, rationals, complex \#s library{[}{]}\{index=``infinite
precision integer library''\} has these blocks:

The USE BIGNUMS block{[}{]}\{index=``BIGNUMS block''\} takes a Boolean
input, to turn the infinite precision feature on or off. When on, all of
the arithmetic operators are redefined to accept and report integers of
any number of digits (limited only by the memory of your computer) and,
in fact, the entire Scheme numeric tower, with exact rationals and with
complex numbers. The Scheme number block{[}{]}\{index=``Scheme number
block''\} has a list of functions applicable to Scheme numbers,
including subtype predicates such as rational? and infinite?, and
selectors such as numerator and real-part.

The ! block{[}{]}\{index=``! block''\} computes the factorial
function{[}{]}\{index=``factorial''\} , useful to test whether bignums
are turned on. Without bignums:

With bignums:

The 375-digit value of 200! isn't readable on this page, but if you
right-click on the block and choose ``result pic,'' you can open the
resulting picture in a browser window and scroll through it. (These
values end with a bunch of zero digits. That's not roundoff error; the
prime factors of 100! and 200! include many copies of 2 and 5.) The
block with no name{[}{]}\{index=``block with no name''\} is a way to
enter things like 3/4 and 4+7i into numeric input slots by converting
the slot to Any type.

The strings, multi-line input library{[}{]}\{index=``string processing
library''\} provides these blocks: {[}{]}\{index=``case-independent
comparisons block''\}

All of these could be written in Snap\emph{!} itself, but these are
implemented using the corresponding JavaScript library functions
directly, so they run fast. They can be used, for example, in scraping
data from a web site. The command use case-independent comparisons
applies only to this library. The multiline
block{[}{]}\{index=``multiline block''\} accepts and reports a text
input that can include newline characters.

The animation library{[}{]}\{index=``animation library''\} has these
blocks:

Despite the name, this isn't only about graphics; you can animate the
values of a variable, or anything else that's expressed numerically.

The central idea of this library is an \emph{easing
function}{[}{]}\{index=``easing function''\} \emph{,} a reporter whose
domain and range are real numbers between 0 and 1 inclusive. The
function represents what fraction of the ``distance'' (in quotes because
it might be any numeric value, such as temperature in a simulation of
weather) from here to there should be covered in what fraction of the
time. A linear easing{[}{]}\{index=``easing block''\} function means
steady progression. A quadratic easing function means starting slowly
and accelerating. (Note that, since it's a requirement that
\emph{f}(0)=0 and \emph{f}(1)=1, there is only one linear easing
function, \emph{f}(\emph{x})=\emph{x}, and similarly for other
categories.) The block reports some of the common easing functions.

The two Motion blocks in this library animate a sprite. Glide always
animates the sprite's motion. Animate's first pulldown menu input allows
you to animate horizontal or vertical motion, but will also animate the
sprite's direction or size. The animate block{[}{]}\{index=``animate
block''\} in Control lets you animate any numeric quantity with any
easing function. The getter and setter inputs are best explained by
example:

is equivalent to

The other blocks in the library are helpers for these four.

The serial ports library{[}{]}\{index=``serial-ports library''\}
contains these blocks:

It is used to allow hardware developers to control devices such as
robots that are

connected to your computer via a serial port.

The frequency distribution analysis library{[}{]}\{index=``frequency
distribution analysis library''\} has these blocks:

This is a collection of tools for analyzing large data sets and plotting
histogram{[}{]}\{index=``histogram''\} s of how often some value is
found in some column of the table holding the data.

For more information go here:

https://tinyurl.com/jens-data

The audio comp library{[}{]}\{index=``sound manipulation library''\}
includes these blocks:

This library takes a sound, one that you record or one from our
collection of sounds, and manipulates it by systematically changing the
intensity of the samples in the sound and by changing the sampling rate
at which the sound is reproduced. Many of the blocks are helpers for the
plot sound block, used to plot the waveform of a
sound{[}{]}\{index=``plot sound block''\} . The play sound (primitive)
block{[}{]}\{index=``play block''\} plays a sound. \_\_ Hz
for{[}{]}\{index=``Hz for block''\} reports a sine wave as a list of
samples.

The web services library{[}{]}\{index=``web services library''\} has
these blocks:

The first block is a generalization of the primitive url
block{[}{]}\{index=``url block''\} , allowing more control over the
various options in web requests: GET, POST, PUT, and DELETE, and fine
control over the content of the message sent to the server. Current
location{[}{]}\{index=``current location block''\} reports your latitude
and longitude. Listify{[}{]}\{index=``listify block''\} takes some text
in JSON format (see page
\hyperref[multi-dimensional-lists-and-json]{54}) and converts it to a
structured list. Value at key{[}{]}\{index=``value at key block''\}
looks up a key-value pair in a (listified) JSON dictionary. The
key:value: block{[}{]}\{index=``key\textbackslash value\textbackslash{}
block''\} is just a constructor for an abstract data type used with the
other blocks

The database library{[}{]}\{index=``database library''\} contains these
blocks:

It is used to keep data that persist from one Snap\emph{!} session to
the next, if you use the same browser and the same login.

The world map library{[}{]}\{index=``map library''\} has these blocks:

Using any of the command blocks puts a map on the screen, in a layer in
front of the stage's background but behind the pen trails layer (which
is in turn behind all the sprites). The first block asks your browser
for your current physical location, for which you may be asked to give
permission. The next two blocks get and set the map's zoom amount; the
default zoom of 10 fits from San Francisco not quite down to Palo Alto on
the screen. A zoom of 1 fits almost the entire world. A zoom of 3 fits
the United States; a zoom of 5 fits Germany. The zoom can be changed in
half steps, i.e., 5.5 is different from 5, but 5.25 isn't.

The next five blocks convert between stage coordinates (pixels) and
Earth coordinates (latitude and longitude). The change by x: y: block
shifts the map relative to the stage. The distance to block measures the
map distance (in meters) between two sprites. The three reporters with
current in their names find \emph{your} actual location, again supposing
that geolocation is enabled on your device. Update redraws the map; as
costume reports the visible section of the map as a costume. Set style
allows things like satellite pictures.

The APL primitives library contains{[}{]}\{index=``APL library''\} these
blocks:

For more information about APL, see Appendix B (page
\hyperref[appendix-b.-apl-features]{148}).

The \textbf{list comprehension library}{[}{]}\{index=``list
comprehension library''\} has one block, zip. Its first input is a
function of two inputs. The two Any-type inputs are deep lists (lists of
lists of\ldots) interpreted as trees, and the function is called with
every possible combination of a leaf node of the first tree and a leaf
node of the second tree. But instead of taking atoms (non-lists) as the
leaves, zip allows the leaves of each tree to be vectors
(one-dimensional lists), matrices (two-dimensional lists), etc. The
Number-type inputs specify the leaf dimension for each tree, so the
function input might be called with a vector from the first tree and an
atom from the second tree.

The \textbf{bitwise library}{[}{]}\{index=``bitwise library''\} provides
bitwise logic functions; each bit of the reported value is the result of
applying the corresponding Boolean function to the corresponding bits of
the input(s). The Boolean functions are not for ¬, and for ∧, or for ∨,
and xor (exclusive or) for ⊻. The remaining functions shift their first
input left or right by the number of bits given by the second input.
\textless\textless{} is left shift, \textgreater\textgreater{} is
arithmetic right shift (shifting in one bits from the left), and
\textgreater\textgreater\textgreater{} is logical right shift (shifting
in zero bits from the left). If you don't already know what these mean,
find a tutorial online.

The \textbf{MQTT library}{[}{]}\{index=``MQTT library''\} supports the
Message Queuing Telemetry Transport protocol, for connecting with IOT
devices. See \url{https://mqtt.org/} for more information.

The \textbf{Signada library}{[}{]}\{index=``Signada library''\} allows
you to control a microBit or similar device that works with the Signada
MicroBlocks project.

The \textbf{menus library}{[}{]}\{index=``menus library''\} provides the
ability to display hierarchical menus on the stage, using the ask
block's ability to take lists as inputs. See page
\hyperref[ask_lists]{24}.

The \textbf{SciSnap\emph{!} library}{[}{]}\{index=``SciSnap! library''\}
and the \textbf{TuneScope library}{[}{]}\{index=``TuneScope library''\}
are too big to discuss here and are documented separately at
\url{http://emu-online.de/ProgrammingWithSciSnap.pdf} and
\url{https://maketolearn.org/creating-art-animations-and-music/}
respectively.

\bookmarksetup{startatroot}

\chapter{Saving and Loading Projects and
Media}\label{saving-and-loading-projects-and-media}

After you've created a project, you'll want to save it, so that you can
have access to it the next time you use Snap\emph{!}. There are two ways
to do that. You can save a project on your own computer, or you can save
it at the Snap\emph{!} web site. The advantage of saving on the net is
that you have access to your project even if you are using a different
computer, or a mobile device such as a tablet or smartphone. The
advantage of saving on your computer is that you have access to the
saved project while on an airplane or otherwise not on the net. Also,
cloud projects are limited in size, but you can have all the costumes
and sounds you like if you save locally. This is why we have multiple
ways to save.

\phantomsection\label{saveas}{}In either case, if you choose ``Save
as\ldots{}'' from the File menu. You'll see something like this:

(If you are not logged in to your Snap\emph{!} cloud account, Computer
will be the only usable option.) The text box at the bottom right of the
Save dialog allows you to enter project notes that are saved with the
project.

\section{Local Storage}\label{local-storage}

Click on Computer and Snap\emph{!}'s Save Project dialog window will be
replaced by your operating system's standard save window. If your
project has a name, that name will be the default filename if you don't
give a different name. Another, equivalent way to save to disk is to
choose ``Export project'' from the File menu.

\section{Creating a Cloud Account{[}{]}}\label{creating-a-cloud-account}

The other possibility is to save your project ``in the
cloud,''{[}{]}\{index=``save your project in the cloud''\} at the
Snap\emph{!} web site. In order to do this, you need an account with us.
Click on the Cloud button{[}{]}\{index=``Cloud button''\} ( ) in the
Tool Bar. Choose the ``Signup\ldots{}'' option. This will show you a
window that looks like the picture at the right.

You must choose a user name{[}{]}\{index=``user name''\} that will
identify you on the web site, such as Jens or bh. If you're a Scratch
user, you can use your Scratch name for Snap\emph{!} too. If you're a
kid, don't pick a user name that includes your family name, but first
names or initials are okay. Don't pick something you'd be embarrassed to
have other users (or your parents) see! If the name you want is already
taken, you'll have to choose another one. You must also supply a
password.

We ask for your month and year of birth; we use this information only to
decide whether to ask for your own email address or your parent's email
address. (If you're a kid, you shouldn't sign up for anything on the
net, not even Snap\emph{!}, without your parent's knowledge.) We do not
store your birthdate information on our server; it is used on your own
computer only during this initial signup. We do not ask for your
\emph{exact} birthdate, even for this one-time purpose, because that's
an important piece of personally identifiable information.

When you click OK, an email will be sent to the email address you gave,
asking you to verify (by clicking a link) that it's really your email
address. We keep your email address on file so that, if you forget your
password, we can send you a password-reset link. We will also email you
if your account is suspended for violation of the Terms of Service. We
do not use your address for any other purpose. You will never receive
marketing emails of any kind through this site, neither from us nor from
third parties. If, nevertheless, you are worried about providing this
information, do a web search for ``temporary email.''

Finally, you must read and agree to the Terms of
Service{[}{]}\{index=``Terms of Service''\} . A quick summary: Don't
interfere with anyone else's use of the web site, and don't put
copyrighted media or personally identifiable information in projects
that you share with other users. And we're not responsible if something
goes wrong. (Not that we \emph{expect} anything to go wrong; since
Snap\emph{!} runs in JavaScript in your browser, it is strongly isolated
from the rest of your computer. But the lawyers make us say this.)

\section{Saving to the Cloud}\label{saving-to-the-cloud}

Once you've created your account, you can log into it using the
``Login\ldots{}'' option from the Cloud menu:

Use the user name and password that you set up earlier. If you check the
``Stay signed in'' box, then you will be logged in automatically the
next time you run Snap\emph{!} from the same browser on the same
computer. Check the box if you're using your own computer and you don't
share it with siblings. \emph{Don't} check the box if you're using a
public computer at the library, at school, etc.

Once logged in, you can choose the ``Cloud'' option in the ``Save
Project'' dialog shown on page \hyperref[saveas]{37}. You enter a
project name, and optionally project notes; your project will be saved
online and can be loaded from anywhere with net access. The project
notes will be visible to other users if you publish your project.

\section{Loading Saved Projects{[}{]}}\label{loading-saved-projects}

Once you've saved a project, you want to be able to load it back into
Snap\emph{!}. There are two ways to do this:

1. If you saved the project in your online Snap\emph{!} account, choose
the ``Open\ldots{}'' option from the File menu. Choose the ``Cloud''
button, then select your project from the list in the big text box and
click OK, or choose the ``Computer'' button to open an operating system
open dialog. (A third button, ``Examples,'' lets you choose from example
projects that we provide. You can see what each of these projects is
about by clicking on it and reading its project notes.)

2. If you saved the project as an XML file on your computer, choose
``Import\ldots{}'' from the File menu. This will give you an ordinary
browser file-open window, in which you can navigate to the file as you
would in other software. Alternatively, find the XML file on your
desktop, and just drag it onto the Snap\emph{!} window.

The second technique above also allows you to import media (costumes and
sounds) into a project. Just choose ``Import\ldots{}'' and then select a
picture or sound file instead of an XML file.

Snap\emph{!} can also import projects created in BYOB 3.0 or 3.1, or
(with some effort; see our web site) in Scratch 1.4, 2.0 or 3.0. Almost
all such projects work correctly in Snap\emph{!}, apart from a small
number of incompatible blocks.

If you saved projects in an earlier version of Snap\emph{!} using the
``Browser'' option, then a Browser button will be shown in the Open
dialog to allow you to retrieve those projects. But you can save them
only with the Computer and Cloud options.

\section{If you lose your project, do this
first!}\label{if-you-lose-your-project-do-this-first}

If you are still in \textbf{Snap\emph{!}} and realize that you've loaded
another project without saving the one you were working on:
\textbf{\emph{Don't edit the new project.}} From the File menu choose
the Restore unsaved project option{[}{]}\{index=``Restore unsaved
project option''\} .

Restore unsaved project will also work if you log out of Snap\emph{!}
and later log back in, as long as you don't edit another project
meanwhile. Snap\emph{!} remembers only the most recent project that
you've edited (not just opened, but actually changed in the project
editor).

If your project on the cloud is missing, empty, or otherwise broken and
isn't the one you edited most recently, or if Restore unsaved project
fails: \textbf{\emph{Don't edit the broken project.}} In the
Open\ldots{} box, enter your project name, then push the Recover
button{[}{]}\{index=``recover button''\} . \emph{Do this right away,}
because we save only the version before the most recent, and the latest
before today. So don't keep saving bad versions; Recover right away. The
Recover feature works only on a project version that you actually saved,
so Restore unsaved project is your first choice if you switch away from
a project without saving it.

To help you remember to save your projects, when you've edited the
project and haven't yet saved it, Snap\emph{!} displays a pencil icon to
the left of the project name on the toolbar at the top of the window:

nnnnnnnnnnnnnnnnnnnnnnnnnnnnnnnn

\section{Private and Public Projects}\label{private-and-public-projects}

By default, a project you save in the cloud is private; only you can see
it. There are two ways to make a project available to others. If you
share a project, you can give your friends a project URL (in your
browser's URL bar after you open the project) they can use to read it.
If you publish a project, it will appear on the Snap\emph{!} web site,
and the whole world can see it. In any case, nobody other than you can
ever overwrite your project; if others ask to save it, they get their
own copy in their own account.

\bookmarksetup{startatroot}

\chapter{Building a Block}\label{building-a-block}

The first version of Snap\emph{!} was called BYOB, for ``Build Your Own
Blocks{[}{]}\{index=''Build Your Own Blocks''\} .'' This was the first
and is still the most important capability we added to
Scratch{[}{]}\{index=``Scratch''\} . (The name was changed because a few
teachers have no sense of humor. ☹ You pick your battles.) Scratch 2.0
and later also has a partial custom block capability.

\section{Simple Blocks}\label{simple-blocks}

In every palette, at or near the bottom, is a button labeled ``Make a
block{[}{]}\{index=''Make a block''\} .'' Also, floating near the top of
the palette is a plus sign. Also, the menu you get by right-clicking on
the background of the scripting area has a ``make a block'' option.

Clicking any of these will display a dialog window in which you choose
the block's name, shape, and palette/color. You also decide whether the
block will be available to all sprites, or only to the current sprite
and its children.

In this dialog box, you can choose the block's palette, shape, and name.
With one exception, there is one color{[}{]}\{index=``color of
blocks''\} per palette, e.g., all Motion blocks are blue. But the
Variables palette includes the orange variable-related blocks and the
red list-related blocks. Both colors are available, along with an
``Other'' option that makes grey blocks in the Variables palette for
blocks that don't fit any category.

There are three block shapes{[}{]}\{index=``shapes of blocks''\} ,
following a convention that should be familiar to Scratch users: The
jigsaw-puzzle-piece shaped blocks{[}{]}\{index=``jigsaw-piece blocks''\}
are Commands, and don't report a value. The oval
blocks{[}{]}\{index=``oval blocks''\} are Reporters, and the hexagonal
blocks{[}{]}\{index=``hexagonal blocks''\} are Predicates, which is the
technical term for reporters that report Boolean (true or false) values.

Suppose you want to make a block named ``square'' that draws a square.
You would choose Motion, Command, and type ``square'' into the name
field. When you click OK, you enter the Block
Editor{[}{]}\{index=``Block Editor''\} . This works just like making a
script in the sprite's scripting area, except that the ``hat'' block at
the top, instead of saying something like ``when I am clicked,'' has a
picture of the block you're building. This hat block{[}{]}\{index=``hat
block''\} is called the \emph{prototype}{[}{]}\{index=``prototype''\} of
your custom block.{[}3{]} You drag blocks under the hat to program your
custom block, then click OK:

Your block appears at the bottom of the Motion palette. Here's the block
and the result of using it:

\subsection{Custom Blocks with Inputs}\label{custom-blocks-with-inputs}

But suppose you want to be able to draw squares of different sizes.
Control-click or right-click on the block, choose ``edit,'' and the
Block Editor{[}{]}\{index=``Block Editor''\} will open. Notice the plus
signs before and after the word square in the prototype block. If you
hover the mouse over one, it lights up:

Click on the plus on the right. You will then see the ``input name''
dialog{[}{]}\{index=``input name dialog''\} :

Type in the name ``size'' and click OK. There are other options in this
dialog; you can choose ``title text{[}{]}\{index=''title text''\} '' if
you want to add words to the block name, so it can have text after an
input slot, like the ``move ( ) steps'' block. Or you can select a more
extensive dialog with a lot of options about your input name. But we'll
leave that for later. When you click OK, the new input appears in the
block prototype:

You can now drag the orange variable down into the script, then click
okay:

Your block now appears in the Motion palette with an input box: You can
draw any size square by entering the length of its side in the box and
running the block as usual, by clicking it or by putting it in a script.

\subsection{Editing Block Properties}\label{editing-block-properties}

What if you change your mind about a block's color (palette) or shape
(command, reporter, predicate)? If you click in the hat block at the top
that holds the prototype, but not in the prototype itself, you'll see a
window in which you can change the color, and \emph{sometimes} the
shape, namely, if the block is not used in any script, whether in a
scripting area or in another custom block. (This includes a one-block
script consisting of a copy of the new block pulled out of the palette
into the scripting area, seeing which made you realize it's the wrong
category. Just delete that copy (drag it back to the palette) and then
change the category.)

If you right-click/control-click the hat block, you get this menu:

Script pic{[}{]}\{index=``script pic''\} exports a picture of the
script. (Many of the illustrations in this manual were made that way.)
Translations{[}{]}\{index=``translations option''\} opens a window in
which you can specify how your block should be translated if the user
chooses a language other than the one in which you are programming.
Block variables lets you create a variant of script variables for this
block: A script variable is created when a block is called, and it
disappears when that call finishes. What if you want a variable that's
local to this block, as a script variable is, but doesn't disappear
between invocations? That's a block variable{[}{]}\{index=``block
variable''\} . If the definition of a block includes a block variable,
then every time that (custom) block is dragged from the palette into a
script, the block variable is created. Every time \emph{that copy} of
the block is called, it uses the same block variable, which preserves
its value between calls. Other copies of the block have their own block
variables. The in palette checkbox determines whether or not this block
will be visible in the palette. It's normally checked, but you may want
to hide custom blocks if you're a curriculum writer creating a Parsons
problem. To unhide blocks, choose ``Hide blocks'' from the File menu and
uncheck the checkboxes. Edit does the same thing as regular clicking, as
described earlier.

\section{Recursion}\label{recursion}

Since the new custom{[}{]}\{index=``recursion''\} block appears in its
palette as soon as you \emph{start} editing it, you can write recursive
blocks (blocks that call themselves) by dragging the block into its own
definition:

(If you added inputs to the block since opening the editor, click Apply
before finding the block in the palette, or drag the
{[}{]}\{index=``drag from prototype''\} block from the top of the block
editor rather than from the palette.)

If recursion is new to you, here are a few brief hints: It's crucial
that the recursion have a \emph{base case}{[}{]}\{index=``base case''\}
\emph{,} that is, some small(est) case that the block can handle without
using recursion. In this example, it's the case depth=0, for which the
block does nothing at all, because of the enclosing if. Without a base
case, the recursion would run forever, calling itself over and over.

Don't try to trace the exact sequence of steps that the computer follows
in a recursive program. Instead, imagine that inside the computer there
are many small people, and if Theresa is drawing a tree of size 100,
depth 6, she hires Tom to make a tree of size 70, depth 5, and later
hires Theo to make another tree of size 70, depth~5. Tom in turn hires
Tammy and Tallulah, and so on. Each little person{[}{]}\{index=``little
people''\} has his or her own local variables size and depth, each with
different values.

You can also write recursive reporters{[}{]}\{index=``reporters,
recursive''\} , like this block to compute the
factorial{[}{]}\{index=``factorial''\} function:

Note the use of the report block{[}{]}\{index=``report block''\} . When
a reporter block uses this block, the reporter finishes its work and
reports the value given; any further blocks in the script are not
evaluated. Thus, the if else block in the script above could have been
just an if, with the second report block below it instead of inside it,
and the result would be the same, because when the first report is seen
in the base case, that finishes the block invocation, and the second
report is ignored. There is also a stop this block
block{[}{]}\{index=``stop block block''\} that has a similar purpose,
ending the block invocation early, for command blocks. (By contrast, the
stop this script block{[}{]}\{index=``stop script block''\} stops not
only the current block invocation, but also the entire toplevel script
that called it.)

Here's a slightly more compact way to write the factorial function:

For more on recursion, see \emph{Thinking
Recursively}{[}{]}\{index=``Thinking Recursively''\} by Eric
Roberts{[}{]}\{index=``Roberts, Eric''\} . (The original edition is ISBN
978‑0471816522; a more recent \emph{Thinking Recursively in Java} is
ISBN 978-0471701460.)

\section{Block Libraries}\label{block-libraries}

When you save a project (see Chapter II above), any custom blocks you've
made are saved with it. But sometimes you'd like to save a collection of
blocks that you expect to be useful in more than one project. Perhaps
your blocks implement a particular data structure (a stack, or a
dictionary, etc.), or they're the framework for building a multilevel
game. Such a collection of blocks is called a \emph{block library.}

\begin{itemize}
\tightlist
\item
  *To create a block library,{[}{]}\{index=``library:block''\} choose
  ``Export blocks\ldots{}'' from the File menu. You then see a window
  like this:
\end{itemize}

The window shows all of your global custom blocks. You can uncheck some
of the checkboxes to select exactly which blocks you want to include in
your library. (You can right-click or control-click on the export window
for a menu that lets you check or uncheck all the boxes at once.) Then
press OK. An XML file containing the blocks will appear in your
Downloads location.

To import a block library, use the ``Import\ldots{}'' command in the
File menu, or just drag the XML file into the Snap\emph{!} window.

Several block libraries are included with Snap\emph{!}; for details
about them, see page \hyperref[libraries-1]{25}.

\section{Custom blocks and Visible
Stepping}\label{custom-blocks-and-visible-stepping}

Visible stepping{[}{]}\{index=``visible stepping''\} normally treats a
call to a custom block as a single step. If you want to see stepping
inside a custom block you must take these steps \emph{in order:}

\begin{enumerate}
\def\labelenumi{\arabic{enumi}.}
\item
  Turn on Visible Stepping.
\item
  Select ``Edit'' in the context menu(s) of the block(s) you want to
  examine.
\item
  Then start the program.
\end{enumerate}

The Block Editor windows you open in step 2 do not have full editing
capability. You can tell because there is only one ``OK'' button at the
bottom, not the usual three buttons. Use the button to close these
windows when done stepping.

\bookmarksetup{startatroot}

\chapter{First class lists}\label{first-class-lists}

A data type is \emph{first class}{[}{]}\{index=``first class data
type''\} in a programming language if data of that type can be

\begin{itemize}
\item
  the value of a variable
\item
  an input to a procedure
\item
  the value returned by a procedure
\item
  a member of a data aggregate
\item
  anonymous (not named)
\end{itemize}

In Scratch{[}{]}\{index=``Scratch''\} , numbers and text strings are
first class. You can put a number in a variable, use one as the input to
a block, call a reporter that reports a number, or put a number into a
list.

But Scratch's lists are not first class. You create one using the ``Make
a list{[}{]}\{index=''Make a list''\} '' button, which requires that you
give the list a name. You can't put the list into a variable, into an
input slot of a block, or into a list item---you can't have lists of
lists. None of the Scratch reporters reports a list value. (You can use
a reduction of the list into a text string as input to other blocks, but
this loses the list structure; the input is just a text string, not a
data aggregate.)

A fundamental design principle{[}{]}\{index=``design principle''\} in
Snap\emph{!} is that \emph{\textbf{all data should be first class}.} If
it's in the language, then we should be able to use it fully and freely.
We believe that this principle avoids the need for many special-case
tools, which can instead be written by Snap\emph{!} users themselves.

Note that it's a data \emph{type} that's first class, not an individual
value. Don't think, for example, that some lists are first class, while
others aren't. In Snap\emph{!}, lists are first class, period.

\section{The list Block}\label{the-list-block}

At the heart of providing first class lists is the ability to make an
``anonymous'' list{[}{]}\{index=``anonymous list''\} ---to make a list
without simultaneously giving it a name. The list reporter
block{[}{]}\{index=``list block''\} does that.

At the right end of the block are two left-and-right
arrowheads{[}{]}\{index=``arrowheads''\} . Clicking on these changes the
number of inputs to list, i.e., the number of elements in the list you
are building. Shift-clicking changes by three at a time.

You can use this block as input to many other blocks:

Snap\emph{!} does not have a ``Make a list'' button like the one in
Scratch{[}{]}\{index=``Scratch''\} . If you want a global ``named
list,'' make a global variable and use the set block to put a list into
the variable.

\section{Lists of Lists}\label{lists-of-lists}

Lists can be inserted as elements in larger lists. We can easily create
ad hoc structures as needed:

Notice that this list is presented in a different format from the ``She
Loves You'' list above. A two-dimensional list is called a \emph{table}
and is by default shown in \emph{table view.} We'll have more to say
about this later.

We can also build any classic computer science data
structure{[}{]}\{index=``data structure''\} out of lists of
lists{[}{]}\{index=``lists of lists''\} , by defining
\emph{constructors}{[}{]}\{index=``constructors''\} (blocks to make an
instance of the structure),
\emph{selectors}{[}{]}\{index=``selectors''\} (blocks to pull out a
piece of the structure), and \emph{mutators}{[}{]}\{index=``mutators''\}
(blocks to change the contents of the structure) as needed. Here we
create binary tree{[}{]}\{index=``binary tree''\} s with selectors that
check for input of the correct data type; only one selector is shown but
the ones for left and right children are analogous.

\section{Functional and Imperative List
Programming}\label{functional-and-imperative-list-programming}

There are two ways to create a list inside a program.
Scratch{[}{]}\{index=``Scratch''\} users will be familiar with the
\emph{imperative} programming style{[}{]}\{index=``imperative
programming style''\} , which is based on a set of command blocks that
modify a list:

As an example, here are two blocks that take a list of numbers as input,
and report a new list containing only the even numbers from the original
list:{[}4{]}

or

In this script, we first create a temporary variable, then put an empty
list in it, then go through the items of the input list using the add
\textbf{\ldots{}} to (result) block to modify the result list, adding
one item at a time, and finally report the result.

\emph{Functional} programming is a different approach that is becoming
important in ``real world'' programming because of
parallelism{[}{]}\{index=``parallelism''\} , i.e., the fact that
different processors can be manipulating the same data at the same time.
This makes the use of mutation{[}{]}\{index=``mutation''\} (changing the
value associated with a variable, or the items of a list) problematic
because with parallelism it's impossible to know the exact sequence of
events, so the result of mutation may not be what the programmer
expected. Even without parallelism, though, functional
programming{[}{]}\{index=``functional programming style''\} is sometimes
a simpler and more effective technique, especially when dealing with
recursively defined data structures. It uses reporter blocks, not
command blocks, to build up a list value:

In a functional program, we often use recursion to construct a list, one
item at a time. The in front of block{[}{]}\{index=``in front of
block''\} makes a list that has one item added to the front of an
existing list, \emph{without changing the value of the original list.} A
nonempty list is processed by dividing it into its first item (item 1
of{[}{]}\{index=``item 1 of block''\} ) and all the rest of the items
(all but first of{[}{]}\{index=``all but first of block''\} ), which are
handled through a recursive call:

Snap\emph{!} uses two different internal representations of lists, one
(dynamic{[}{]}\{index=``array, dynamic''\} array{[}{]}\{index=``dynamic
array''\} ) for imperative programming and the other
(linked{[}{]}\{index=``list, linked''\} list{[}{]}\{index=``linked
list''\} ) for functional programming. Each representation makes the
corresponding built-in list blocks (commands or reporters, respectively)
most efficient. It's possible to mix styles in the same program, but if
\emph{the same list} is used both ways, the program will run more slowly
because it converts from one representation to the other repeatedly.
(The item ( ) of {[} {]} block doesn't change the representation.) You
don't have to know the details of the internal representations, but it's
worthwhile to use each list in a consistent way.

\section{Higher Order List Operations and
Rings}\label{higher-order-list-operations-and-rings}

There's an even easier way to select the even numbers from a list:

The keep block takes a Predicate expression as its first input, and a
list as its second input. It reports a list containing those elements of
the input list for which the predicate returns true. Notice two things
about the predicate input: First, it has a grey
ring{[}{]}\{index=``ring, gray''\} around it. Second, the mod block has
an empty input. Keep puts each item of its input list, one at a time,
into that empty input before evaluating the predicate. (The empty input
is supposed to remind you of the ``box'' notation for variables in
elementary school: ☐+3=7.) The grey ring is part of the keep block as it
appears in the palette:

What the ring means is that this input is a block (a predicate block, in
this case, because the interior of the ring is a hexagon), rather than
the value reported by that block. Here's the difference:

Evaluating the = block without a ring reports true or false; evaluating
the block \emph{with} a ring reports the block itself. This allows keep
to evaluate the = predicate repeatedly, once for each list item. A block
that takes another block as input is called a \emph{higher order} block
(or higher order procedure, or higher order
function{[}{]}\{index=``higher order function''\} ).

Snap\emph{!} provides four higher order function blocks for operating on
lists:

\phantomsection\label{map}{}You've already seen keep. Find first
is{[}{]}\{index=``find first''\} similar, but it reports just the first
item that satisfies the predicate, not a list of all the matching items.
It's equivalent to but faster because it

stops looking as soon as it finds a match. If there are no matching
items, it returns an empty string.

Map{[}{]}\{index=``map block''\} takes a Reporter block and a list as
inputs. It reports a new list in which each item is the value reported
by the Reporter block as applied to one item from the input list. That's
a mouthful, but an example will make its meaning clear:

These examples use small lists, to fit the page, but the higher order
blocks work for any size list.

By the way, we've been using arithmetic examples, but the list items can
be of any type, and any reporter can be used. We'll make the plurals of
some words:

An \emph{empty} gray ring represents the \emph{identity function,} which
just reports its input. Leaving the ring in map empty is the most
concise way to make a shallow copy of a list (that is, in the case of a
list of lists, the result is a new toplevel list whose items are the
same (uncopied) lists that are items of the toplevel input
list){[}{]}\{index=``shallow copy of a list''\} . To make a deep copy of
a list{[}{]}\{index=``deep copy of a list''\} (that is, one in which all
the sublists, sublists of sublists, etc. are copied), use the list as
input to the block (one of the variants of the sqrt of block). This
works because id of is a hyperblock (page \hyperref[hyperblocks]{55}).

The third higher order block, combine{[}{]}\{index=``combine block''\} ,
computes a single result from \emph{all} the items of a list, using a
\emph{two-input} reporter as its second input. In practice, there are
only a few blocks you'll ever use with combine:

These blocks take the sum of the list items, take their product, string
them into one word, combine them into a sentence (with spaces between
items), see if all items of a list of Booleans are true, see if any of
the items is true, find the smallest, or find the largest.

Why + but not −? It only makes sense to combine list items using an
\emph{associative}{[}{]}\{index=``function, associative''\}
function{[}{]}\{index=``associative function''\} : one that doesn't care
in what order the items are combined (left to right or right to left).
(2+3)+4 = 2+(3+4), but (2−3)−4 ≠ 2−(3−4).

The functions map, keep, and find first have an advanced mode with
rarely-used features: If their function input is given explicit input
names (by clicking the arrowhead at the right end of the gray ring; see
page \hyperref[formal-parameters]{69}), then it will be called for each
list item with \emph{three} inputs: the item's value (as usual), the
item's position in the input list (its index), and the entire input
list. No more than three input names can be used in this contex

\section{}\label{section-1}

\section{Table View{[}{]}\{index=``table view''\} vs.~List
View{[}{]}}\label{table-viewindextable-view-vs.-list-view}

We mentioned earlier that there are two ways of representing lists
visually. For one-dimensional lists (lists whose items are not
themselves lists) the visual differences are small:

For one-dimensional lists, it's not really the appearance that's
important. What matters is that the \emph{list view} allows very
versatile direct manipulation of the list through the picture: you can
edit the individual items, you can delete items by clicking the tiny
buttons next to each item, and you can add new items at the end by
clicking the tiny plus sign in the lower left corner. (You can just
barely see that the item deletion buttons have minus signs in them.)
Even if you have several watchers for the same list, all of them will be
updated when you change anything. On the other hand, this versatility
comes at an efficiency cost; a list view watcher for a long list would
be way too slow. As a partial workaround, the list view can only contain
100 items at a time; the downward-pointing arrowhead opens a menu in
which you can choose which 100 to display.

By contrast, because it doesn't allow direct editing, the \emph{table
view} watcher can hold hundreds of thousands of items and still scroll
through them efficiently. The table view has flatter graphics for the
items to remind you that they're not clickable to edit the values.

Right-clicking on a list watcher (in either form) gives you the option
to switch to the other form. The right-click menu also offers an open in
dialog\ldots{} option that opens an \emph{offstage} table view watcher,
because the watchers can take up a lot of stage space that may make it
hard to see what your program is actually doing. Once the offstage
dialog box is open, you can close the stage watcher. There's an OK
button on the offstage dialog to close it if you want. Or you can
right-click it to make \emph{another} offstage watcher, which is useful
if you want to watch two parts of the list at once by having each
watcher scrolled to a different place.

Table view is the default if the list has more than 100 items, or if any
of the first ten items of the list are lists, in which case it makes a
very different-looking two-dimensional picture:

In this format, the column of red items has been replaced by a
spreadsheet-looking display. For short, wide lists, this display makes
the content of the list very clear. A vertical display, with much of the
space taken up by the ``machinery'' at the bottom of each sublist, would
make it hard to show all the text at once. (The pedagogic cost is that
the structure is no longer explicit; we can't tell just by looking that
this is a list of row-lists, rather than a list of column-lists or a
primitive two-dimensional array type. But you can choose list view to
see the structure.)

Beyond such simple cases, in which every item of the main list is a list
of the same length, it's important to keep in mind that the design of
table view has to satisfy two goals, not always in agreement: (1) a
visually compelling display of two-dimensional arrays, and (2) highly
efficient display generation, so that Snap\emph{!} can handle very large
lists, since ``big data'' is an important topic of study. To meet the
first goal perfectly in the case of ``ragged right'' arrays in which
sublists can have different lengths, Snap\emph{!} would scan the entire
list to find the maximum width before displaying anything, but that
would violate the second goal.

Snap\emph{!} uses the simplest possible compromise between the two
goals: It examines only the first ten items of the list to decide on the
format. If none of those are lists, or they're all lists of one item,
and the overall length is no more than 100, list view is used. If the
any of first ten items is a list, then table view is used, and the
number of columns in the table is equal to the largest number of items
among the first ten items (sublists) of the main list.

Table views open with standard values for the width and height of a
cell, regardless of the actual data. You can change these values by
dragging the column letters or row numbers. Each column has its own
width, but changing the height of a row changes the height for all rows.
(This distinction is based not on the semantics of rows vs.~columns, but
on the fact that a constant row height makes scrolling through a large
list more efficient.) Shift-dragging a column label will change the
width of that column.

If you tried out the adjustments in the previous paragraph, you may have
noticed that a column letter turns into a number when you hover over it.
Labeling rows and columns differently makes cell references such as
``cell 4B'' unambiguous; you don't have to have a convention about
whether to say the row first or the column first. (``Cell B4'' is the
same as ``cell 4B.'') On the other hand, to extract a value from column
B in your program, you have to say item 2 of, not item B of. So it's
useful to be able to find out a column number by hovering over its
letter.

Any value that can appear in a program can be displayed in a table cell:

This display shows that the standard cell dimensions may not be enough
for large value images. By expanding the entire speech balloon and then
the second column and all the rows, we can make the result fit:

But we make an exception for cases in which the value in a cell is a
list (so that the entire table is three-dimensional). Because lists are
visually very big, we don't try to fit the entire value in a cell:

Even if you expand the size of the cells, Snap\emph{!} will not display
sublists of sublists in table view. There are two ways to see these
inner sublists: You can switch to list view, or you can double-click on
a list icon in the table to open a dialog box showing just that
sub-sub-list in table view.

One last detail: If the first item of a list is a list (so table view is
used), but a later item \emph{isn't} a list, that later item will be
displayed on a red background, like an item of a single-column list:

So, in particular, if only the first item is a list, the display will
look almost like a one-column display.

\subsection{Comma-Separated Values}\label{comma-separated-values}

Spreadsheet and database programs generally offer the option to export
their data as CSV (comma-separated values{[}{]}\{index=``CSV
(comma-separated values)''\} lists. You can import these files into
Snap\emph{!} and turn them into tables (lists of lists), and you can
export tables in CSV format. Snap\emph{!} recognizes a CSV file by the
extension .csv in its filename.

A CSV file has one line per table row, with the fields separated by
commas within a row:

John,Lennon,rhythm guitar

table view

list view

Paul,McCartney,bass guitar

George,Harrison,lead guitar

Ringo,Starr,drums

Here's what the corresponding table looks like:

Here's how to read a spreadsheet into Snap\emph{!}:

1. Make a variable with a watcher on stage:

2. Right-click on the watcher and choose the ``import'' option. (If the
variable's value is already a list, be sure to click on the outside
border of the watcher; there is a different menu if you click on the
list itself.) Select the file with your csv data.

3. There is no 3; that's it! Snap\emph{!} will notice that the name of
the file you're importing is something.csv and will turn the text into a
list of lists automatically.

Or, even easier, just drag and drop the file from your desktop onto the
Snap\emph{!} window, and Snap\emph{!} will automatically create a
variable named after the file and import the data into it.

If you actually want to import the raw CSV data into a variable, either
change the file extension to .txt before loading it, or choose ``raw
data'' instead of ``import'' in the watcher menu.

If you want to export a list, put a variable watcher containing the list
on the stage, right-click its border, and choose ``Export.'' (Don't
right-click an item instead of the border; that gives a different menu.)

\subsection{Multi-dimensional lists and
JSON}\label{multi-dimensional-lists-and-json}

CSV format is easy to read, but works only for one- or two-dimensional
lists. If you have a list of lists of lists, Snap\emph{!} will instead
export your list as a JSON (JavaScript Object Notation)
file{[}{]}\{index=``JSON (JavaScript Object Notation) file''\} . I
modified my list:

and then exported again, getting this file:

{[}{[}``John'',``Lennon'',``rhythm
guitar''{]},{[}{[}``James'',``Paul''{]},``McCartney'',``bass
guitar''{]},{[}``George'',``Harrison'',``lead
guitar''{]},{[}``Ringo'',``Starr'',``drums''{]}{]}

You can also import lists, including tables, from a .json file. (And you
can import plain text from a .txt file.) Drag and drop works for these
formats also.

\section{Hyperblocks{[}{]}}\label{hyperblocks}

A \emph{scalar} is anything other than a list. The name comes from
mathematics, where it means a magnitude without direction, as opposed to
a vector, which points toward somewhere. A scalar
function{[}{]}\{index=``scalar function''\} is one whose domain and
range are scalars, so all the arithmetic operations are scalar
functions, but so are the text ones such as letter and the Boolean ones
such as not.

The major new feature in Snap\emph{!} 6.0 is that the domain and range
of most scalar function blocks is extended to
multi-dimensional{[}{]}\{index=``list, multi-dimensional''\} lists, with
the underlying scalar function applied termwise:

Mathematicians, note in the last example above that the result is just a
termwise application of the underlying function (7×3, 8×5, etc.),
\emph{not} matrix multiplication. See Appendix B for that. For a dyadic
(two-input) function, if the lengths don't agree, the length of the
result (in each dimension) is the length of the shorter input:

However, if the \emph{number of dimensions} differs in the two inputs,
then the number of dimensions in the result agrees with the
\emph{higher-}dimensional input; the lower-dimensional one is used
repeatedly in the missing dimension(s):

(7×6. 8×10, 1×20, \emph{40}×\emph{6, 20}×\emph{10,} etc.). In
particular, a \emph{scalar} input is paired with every scalar in the
other input:

One important motivation for this feature is how it simplifies and
speeds up media computation{[}{]}\{index=``media computation''\} , as in
this shifting of the Alonzo{[}{]}\{index=``Alonzo''\} costume to be
bluer:

Each pixel of the result has ¾ of its original red and green, and three
times its original blue (with its transparency unchanged). By putting
some sliders on the stage, you can play with colors dynamically:

There are a few naturally scalar functions that have already had
specific meanings when applied to lists and therefore are not
hyperblocks: = and identical to (because they compare entire structures,
not just scalars, always reporting a single Boolean result), and and or
(because they don't evaluate their second input at all if the first
input determines the result), join (because it converts non-scalar (and
other non-text) inputs to text string form), and is a (type) (because it
applies to its input as a whole). Blocks whose inputs are ``natively''
lists, such as and , are never hyperblocks.

The reshape block{[}{]}\{index=``reshape block''\} takes a list (of any
depth) as its first input, and then takes zero or more sizes along the
dimensions of an array. In the example it will report a table (a matrix)
of four rows and three columns. If no sizes are given, the result is an
empty list. Otherwise, the cells of the specified shape are filled with
the atomic values from the input list. If more values are needed than
provided, the block starts again at the head of the list, using values
more than once. If more values are provided than needed, the extras are
ignored; this isn't an error.

The combinations block takes any number of lists as input; it reports a
list in which each item is a list whose length is the number of inputs;
item \emph{i} of a sublist is an item of input \emph{i.} Every possible
combination of items of the inputs is included, so the length of the
reported list is the product of the lengths of the inputs.

The item of block{[}{]}\{index=``item of block''\} has a special set of
rules, designed to preserve its pre-hyperblock meaning and also provide
a useful behavior when given a list as its first (index) input:

\begin{enumerate}
\def\labelenumi{\arabic{enumi}.}
\item
  If the index is a number, then item of reports the indicated top-level
  item of the list input; that item may be a sublist, in which case the
  entire sublist is reported (the original meaning of item of):
\item
  If the index is a list of numbers (no sublists), then item of reports
  a list of the indicated top-level items (rows, in a matrix; a
  straightforward hyperization):
\item
  If the index is a list of lists of numbers, then item of reports an
  array of only those scalars whose position in the list input matches
  the index input in all dimensions (changed in Snap\emph{!} 6.6!):
\item
  If a list of list of numbers includes an empty sublist, then all items
  are chosen along that dimension:
\end{enumerate}

To get a column or columns of a spreadsheet, use an empty list in the
row selector (changed in Snap\emph{!} 6.6!):

The length of block{[}{]}\{index=``length of block''\} is extended to
provide various ways of looking at the shape and contents of a list. The
options other than length are mainly useful for \emph{lists of lists,}
to any depth. These new options work well with hyperblocks and the APL
library. (Examples are on the next page.)

length: reports the number of (toplevel) items in the list, as always.

rank{[}{]}\{index=``rank of block''\} : reports the number of
\emph{dimensions} of the list, i.e., the maximum depth of lists of lists
of lists of lists. (That example would be rank 4.)

dimensions{[}{]}\{index=``dimensions of block''\} : reports a list of
numbers, each of which is the maximum length in one dimension, so a
spreadsheet of 1000 records, each with 4 fields, would report the list
{[}1000 4{]}.

flatten{[}{]}\{index=``flatten of block''\} : reports a flat,
one-dimensional list containing the \emph{atomic} (non-list) items
anywhere in the input list.

columns{[}{]}\{index=``columns of block''\} : reports a list in which
the rows and columns of the input list are interchanged, so the shape of
the transpose of a shape {[}1000 4{]} list would be {[}4 1000{]}. This
option works only for lists whose rank is at most 2. The name reflects
the fact that the toplevel items of the reported table are the columns
of the original table.

reverse: reports a list in which the (toplevel) items of the input list
are in reverse order.

The remaining three options report a (generally multi-line) text string.
The input list may not include any atomic (non-list)
data{[}{]}\{index=``atomic data''\} other than text or numbers. The
lines{[}{]}\{index=``lines of block''\} option is intended for use with
rank-one lists of text strings; it reports a string in which each list
item becomes a line of text. You can think of it as the opposite of the
split by line block{[}{]}\{index=``split by line block''\} . The
csv{[}{]}\{index=``csv of block''\} option (comma-separated values) is
intended for rank-two lists that represent a spreadsheet or other
tabular data. Each item of the input list should be a list of atoms; the
block reports a text string in which each item of the big list becomes a
line of text in which the items of that sublist are separated by commas.
The json{[}{]}\{index=``json of block''\} option is for lists of any
rank; it reports a text string in which the list structure is explicitly
represented using square brackets. These are the opposites of split by
csv and split by json.

input

The idea of extending the domain and range of scalar functions to
include arrays comes from the language APL{[}{]}\{index=``APL''\} . (All
the great programming languages are based on mathematical ideas. Our
primary ancestors are Smalltalk{[}{]}\{index=``Smalltalk''\} , based on
models, and Lisp{[}{]}\{index=``Lisp''\} , based on lambda calculus.
Prolog{[}{]}\{index=``Prolog''\} , a great language not (so far)
influencing Snap\emph{!}, is based on logic. And APL, now joining our
family, is based on linear algebra, which studies vectors and matrices.
Those \emph{other} programming languages are based on the weaknesses of
computer hardware.) Hyperblocks are not the whole story about APL, which
also has mixed-domain functions and higher order functions. Some of
what's missing is provided in the APL library. (See Appendix B.)

\bookmarksetup{startatroot}

\chapter{Typed Inputs}\label{typed-inputs}

\section{Scratch's Type Notation}\label{scratchs-type-notation}

Prior to version 3, Scratch{[}{]}\{index=``Scratch''\} block inputs came
in two types{[}{]}\{index=``data type''\} : Text-or-number type and
Number type. The former is indicated by a rectangular box, the latter by
a rounded box: . A third Scratch type, Boolean (true/false), can be used
in certain Control blocks with hexagonal slots.

The Snap\emph{!} types are an expanded collection including Procedure,
List, and Object types. Note that, with the exception of Procedure
types, all of the input type shapes are just reminders to the user of
what the block expects; they are not enforced by the language.

\section{\texorpdfstring{The Snap\emph{!} Input Type
Dialog}{The Snap! Input Type Dialog}}\label{the-snap-input-type-dialog}

In the Block Editor{[}{]}\{index=``Block Editor''\} input name
dialog{[}{]}\{index=``input name dialog''\} , there is a right-facing
arrowhead after the ``Input name'' option:

Clicking that arrowhead opens the ``long'' input name
dialog{[}{]}\{index=``long input name dialog''\} :

There are twelve input type shapes{[}{]}\{index=``input-type shapes''\}
, plus three mutually exclusive modifiers, listed in addition to the
basic choice between title text and an input name. The default type, the
one you get if you don't choose anything else, is ``Any,'' meaning that
this input slot is meant to accept any value of any
type{[}{]}\{index=``Any type''\} . If the size input in your block
should be an oval-shaped numeric slot rather than a generic rectangle,
click ``Number.''

The arrangement of the input types is systematic. As the pictures on
this and the next page show, each row of types is a category, and parts
of each column form a category. Understanding the arrangement will make
it a little easier to find the type you want.

The second row of input types contains the ones found in Scratch:
Number, Any, and Boolean. (The reason these are in the second row rather
than the first will become clear when we look at the column arrangement.)
The first row contains the new Snap\emph{!} types other than procedures:
Object, Text, and List. The last two rows are the types related to
procedures, discussed more fully below.

The List type{[}{]}\{index=``List type''\} is used for first class lists,
discussed in Chapter IV above. The red rectangles inside the input slot
are meant to resemble the appearance of lists as Snap\emph{!} displays
them on the stage: each element in a red rectangle.

The Object type{[}{]}\{index=``Object type''\} is for sprites, costumes,
sounds, and similar data types.

The Text type{[}{]}\{index=``Text type''\} is really just a variant form
of the Any type, using a shape that suggests a text input.{[}5{]}

\subsection{Procedure Types}\label{procedure-types}

Although the procedure types are discussed more fully later, they are
the key to understanding the column arrangement in the input types. Like
Scratch, Snap\emph{!} has three block shapes{[}{]}\{index=``block
shapes''\} : jigsaw-piece{[}{]}\{index=``jigsaw-piece blocks''\} for
command blocks, oval{[}{]}\{index=``oval blocks''\} for reporters, and
hexagonal{[}{]}\{index=``hexagonal blocks''\} for predicates. (A
\emph{predicate} is a reporter that always reports true or false.) In
Snap\emph{!} these blocks are first class data; an input to a block can
be of Command type, Reporter type, or Predicate type. Each of these
types is directly below the type of value that that kind of block
reports, except for Commands, which don't report a value at all. Thus,
oval Reporters are related to the Any type, while hexagonal Predicates
are related to the Boolean (true or false) type.

The unevaluated procedure types{[}{]}\{index=``unevaluated procedure
types''\} in the fourth row are explained in Section VI.E below. In one
handwavy sentence, they combine the \emph{meaning} of the procedure
types with the \emph{appearance} of the reported value types two rows
higher. (Of course, this isn't quite right for the C-shaped command
input type, since commands don't report values. But you'll see later
that it's true in spirit.)

\subsection{Pulldown inputs}\label{pulldown-inputs}

Certain primitive blocks have \emph{pulldown}
inputs{[}{]}\{index=``pulldown input''\} , either
\emph{read-only}{[}{]}\{index=``read-only pulldown input''\} \emph{,}
like the input to the touching block:

(indicated by the input slot being the same (cyan, in this case) color
as the body of the block), or \emph{writeable}{[}{]}\{index=``writeable
pulldown inputs''\} \emph{,} like the input to the point in direction
block:

(indicated by the white input slot), which means that the user can type
in an arbitrary input instead of using the pulldown menu.

Custom blocks can also have such inputs. To make a pulldown input, open
the long form input dialog, choose a text type (Any, Text, or Number)
and click the icon in the bottom right corner, or control/right-click in
the dialog. You will see this menu:

Click the read-only checkbox if you want a read-only pulldown input.
Then from the same menu, choose options\ldots{} to get this dialog box:

Each line in the text box represents one menu item. If the line does not
contain any of the characters =\textasciitilde\{\} then the text is both
what's shown in the menu and the value of the input if that entry is
chosen.

If the line contains an equal sign =, then the text to the left of the
equal sign is shown in the menu, and the text to the right is what
appears in the input slot if that entry is chosen, and is also the value
of the input as seen by the procedure.

If the line consists of a tilde \textasciitilde, then it represents a
separator{[}{]}\{index=``separator:menu''\} (a horizontal line) in the
menu, used to divide long menus into visible categories. There should be
nothing else on the line. This separator is not choosable, so there is
no input value corresponding to it.

If the line ends with the two characters equal sign and open brace =\{,
then it represents a \emph{submenu.} The text before the equal sign is a
name for the submenu{[}{]}\{index=``submenu''\} , and will be displayed
in the menu with an arrowhead ► at the end of the line. This line is not
clickable, but hovering the mouse over it displays the submenu next to
the original menu. A line containing a close brace \} ends the submenu;
nothing else should be on that line. Submenus may be nested to arbitrary
depth.

\subsection{}\label{section-2}

Alternatively, instead of giving a menu listing as described above, you
can put a JavaScript function that returns the desired menu in the
textbox. This is an experimental feature and requires that JavaScript be
enabled in the Settings menu.\\
It is also possible to get the special menus used in some primitive
blocks, by choosing from the menu submenu: broadcast messages, sprites
and stage, costumes, sounds, variables that can be set in this scope,
the play note piano keyboard, or the point in direction 360° dial.
Finally, you can make the input box accept more than one line of text
(that is, text including a newline character) from the special submenu,
either ``multi-line'' for regular text or ``code'' for monospace-font
computer code.

If the input type is something other than text, then clicking the button
will instead show this menu:

As an example, we want to make this block: The second input must be a
read-only object menu:

the move (10) steps block. In the prototype block input at the top of
the script in the Block Editor, an input with name ``size'' and default
value 10 looks like this:

The ``single input'' option: In Scratch, all inputs are in this
category. There is one input slot in the block as it appears in its
palette. If a single input is of type Any, Number, Text, or Boolean,
then you can specify a default value{[}{]}\{index=``default value''\}
that will be shown in that slot in the palette, like the ``10'' in the
move (10) steps block. In the prototype block at the top of the script
in the Block editor, an

\subsection{Input variants}\label{input-variants}

We now turn to the three mutually exclusive options that come below the
type array.

The ``Multiple inputs'' option: The list block introduced earlier
accepts any number of inputs to specify the items of the new list. To
allow this, Snap\emph{!} introduces the arrowhead notation (⏴⏵) that
expands and contracts the block, adding and removing input slots.
(Shift-clicking on an arrowhead adds or removes three input slots at
once.) Custom blocks made by the Snap\emph{!} user have that capability,
too. If you choose the ``Multiple inputs'' button, then
arrowheads{[}{]}\{index=``arrowheads''\} will appear after the input
slot in the block. More or fewer slots (as few as zero) may be used.
When the block runs, all of the values in all of the slots for this
input name are collected into a list, and the value of the input as seen
inside the script is that list of values:

The ellipsis{[}{]}\{index=``ellipsis''\} (\ldots) in the orange input
slot name box in the prototype indicates a multiple or \emph{variadic}
input{[}{]}\{index=``variadic input''\} .

The third category, ``Upvar - make internal
variable{[}{]}\{index=''internal variable''\} visible to
caller{[}{]}\{index=``make internal variable visible''\} ,'' isn't
really an input at all, but rather a sort of output from the block to
its user. It appears as an orange variable oval in the block, rather
than as an input slot. Here's an example; the uparrow
(\textbf{↑}){[}{]}\{index=``upward-pointing arrow''\} in the prototype
indicates this kind of internal variable name:

➜

The variable i (in the block on the right above) can be dragged from the
for block{[}{]}\{index=``for block''\} into the blocks used in its
C-shaped command slot. Also, by clicking on the orange i, the user can
change the name of the variable as seen in the calling script (although
the name hasn't changed inside the block's definition). This kind of
variable is called an \emph{upvar}{[}{]}\{index=``upvar''\} for short,
because it is passed \emph{upward} from the custom block to the script
that uses it.

Note about the example: for is a primitive block, but it doesn't need to
be. You're about to see (next chapter) how it can be written in
Snap\emph{!}. Just give it a different name to avoid confusion, such as
my for as above.

\subsection{Prototype Hints}\label{prototype-hints}

We have mentioned three notations that can appear in an input slot in
the prototype to remind you of what kind of input this is. Here is the
complete list of such notations:

= default value \ldots{} multiple input ↑ upvar \# number

λ procedure types ⫶ list ? Boolean object ¶ multi-line text

\subsection{Title Text and Symbols}\label{title-text-and-symbols}

Some primitive blocks have symbols{[}{]}\{index=``icons in title
text''\} as part of the block name: . Custom blocks can use symbols too.
In the Block Editor, click the plus sign in the prototype at the point
where you want to insert the symbol. Then click the title text picture
below the text box that's expecting an input slot name. The dialog will
then change to look like this:

The important part to notice is the arrowhead that has appeared at the
right end of the text box. Click it to see the menu shown here at the
left.

Choose one of the symbols. The result will have the symbol you want: The
available symbols are, pretty much, the ones that are used in
Snap\emph{!} icons.

But I'd like the arrow symbol bigger, and yellow, so I edit its name:

This makes the symbol 1.5 times as big as the letters in the block text,
using a color with red-green-blue values of 255-255-150 (each between 0
and 255). Here's the result:

The size and color controls can also be used with text:
\$foo-8-255-120-0 will make a huge orange ``foo.''

Note the last entry in the symbol menu: ``new line{[}{]}\{index=''new
line character''\} .'' This can be used in a block with many inputs to
control where the text continues on another line, instead of letting
Snap\emph{!} choose the line break itself.

\bookmarksetup{startatroot}

\chapter{Procedures as Data}\label{procedures-as-data}

\section{Call and Run}\label{call-and-run}

In the for block{[}{]}\{index=``for block''\} example above, the input
named action has been declared as type ``Command (C-shaped)''; that's
why the finished block is C-shaped. But how does the block actually tell
Snap\emph{!} to carry out the commands inside the C-slot? Here is a
simple version of the block script:

This is simplified because it assumes, without checking, that the ending
value is greater than the starting value; if not, the block should
(depending on the designer's purposes) either not run at all, or change
the variable by −1 for each repetition instead of by 1.

The important part of this script is the run block{[}{]}\{index=``run
block''\} near the end. This is a Snap\emph{!} built-in command block
that takes a Command-type value (a script) as its input, and carries out
its instructions. (In this example, the value of the input is the script
that the user puts in the C-slot of the my for block.) There is a
similar call reporter block for invoking a Reporter or Predicate block.
The call{[}{]}\{index=``call block''\} and run blocks are at the heart
of Snap\emph{!}'s first class procedure{[}{]}\{index=``first class
procedures''\} feature; they allow scripts and blocks to be used as
data---in this example, as an input to a block---and eventually carried
out under control of the user's program.

Here's another example, this time using a Reporter-type input in a map
block (see page \hyperref[map]{50}):{[}{]}\{index=``map block''\}

Here we are calling the Reporter ``multiply by 10'' three times, once
with each item of the given list as its input, and collecting the
results as a list. (The reported list will always be the same length as
the input list.) Note that the multiplication block has two inputs, but
here we have specified a particular value for one of them (10), so the
call block knows to use the input value given to it just to fill the
other (empty) input slot in the multiplication block. In the my map
definition, the input function is declared to be type Reporter, and data
is of type List.

\subsection{Call/Run with inputs}\label{callrun-with-inputs}

The call block (like the run block) has a right arrowhead at the end;
clicking on it adds the phrase ``with inputs'' and then a slot into
which an input can be inserted:

If the left arrowhead is used to remove the last input slot, the ``with
inputs{[}{]}\{index=''with inputs''\} '' disappears also. The right
arrowhead can be clicked as many times as needed for the number of
inputs required by the reporter block being called.

If the number of inputs given to call (not counting the Reporter-type
input that comes first) is the same as the number of empty input
slots{[}{]}\{index=``empty input slots, filling''\} , then the empty
slots are filled from left to right with the given input values. If call
is given exactly one input, then \emph{every} empty input slot of the
called block is filled with the same value:

If the number of inputs provided is neither one nor the number of empty
slots, then there is no automatic filling of empty slots. (Instead you
must use explicit parameters in the ring, as discussed in Section C
below.)

An even more important thing to notice about these examples is the
\emph{ring}{[}{]}\{index=``ring, gray''\} around the Reporter-type input
slots in call and map above. This notation indicates that \emph{the
block itself,} not the number or other value that the block would report
when called, is the input. If you want to use a block itself in a
non-Reporter-type (e.g., Any-type) input slot, you can enclose it
explicitly in a ring, found at the top of the Operators palette.

As a shortcut, if you right-click or control-click on a block (such as
the + block in this example), one of the choices in the menu that
appears is ``ringify{[}{]}\{index=''ringify''\} '' and/or
``unringify{[}{]}\{index=''unringify''\} .'' The ring indicating a
Reporter-type or Predicate-type input slot is essentially the same idea
for reporters as the C-shaped input slot with which you're already
familiar; with a C-shaped slot, it's \emph{the script} you put in the
slot that becomes the input to the C-shaped block.

There are three ring shapes. All are oval on the outside, indicating
that the ring reports a value, the block or script inside it, but the
inside shapes are command, reporter, or predicate, indicating what kind
of block or script is expected. Sometimes you want to put something more
complicated than a single reporter inside a reporter ring; if so, you
can use a script, but the script must report a value, as in a custom
reporter definition.

\subsection{Variables in Ring Slots}\label{variables-in-ring-slots}

Note that the run block{[}{]}\{index=``variables in ring slots''\} in
the definition of the my for block (page \hyperref[call-and-run]{65})
doesn't have a ring around its input variable action. When you drag a
variable into a ringed input slot, you generally \emph{do} want to use
\emph{the value of} the variable, which will be the block or script
you're trying to run or call, rather than the orange variable reporter
itself. So Snap\emph{!} automatically removes the ring in this case. If
you ever do want to use the variable \emph{block itself,} rather than
the value of the variable, as a Procedure-type input, you can drag the
variable into the input slot, then control-click or right-click it and
choose ``ringify'' from the menu that appears. (Similarly, if you ever
want to call a function that will report a block to use as the input,
such as item 1 of applied to a list \emph{of blocks,} you can choose
``unringify'' from the menu. Almost all the time, though, Snap\emph{!}
does what you mean without help.)

\section{Writing Higher Order
Procedures}\label{writing-higher-order-procedures}

A \emph{higher order procedure}{[}{]}\{index=``higher order
procedure''\} is one that takes another procedure as an input, or that
reports a procedure. In this document, the word
``procedure{[}{]}\{index=''procedure''\} '' encompasses scripts,
individual blocks, and nested reporters. (Unless specified otherwise,
``reporter'' includes predicates. When the word is capitalized inside a
sentence, it means specifically oval-shaped blocks. So, ``nested
reporters'' includes predicates, but ``a Reporter-type input'' doesn't.)

Although an Any-type input slot (what you get if you use the small
input-name dialog box) will accept a procedure input, it doesn't
automatically ring the input as described above. So the declaration of
Procedure-type inputs makes the use of your custom higher order block
much more convenient.

Why would you want a block to take a procedure as input? This is
actually not an obscure thing to do; the primitive conditional and
looping blocks (the C-shaped ones in the Control palette) take a script
as input. Users just don't usually think about it in those terms! We
could write the repeat block{[}{]}\{index=``repeat block''\} as a custom
block this way, if Snap\emph{!} didn't already have one:

The lambda (λ) next to action in the prototype indicates that this is a
C-shaped block{[}{]}\{index=``C-shaped block''\} , and that the script
enclosed by the C when the block is used is the input named action in
the body of the script. The only way to make sense of the variable
action is to understand that its value is a script.

To declare an input to be Procedure-type, open the input name dialog as
usual, and click on the arrowhead:

Then, in the long dialog, choose the appropriate Procedure type. The
third row of input types has a ring in the shape of each block type
(jigsaw for Commands, oval for Reporters, and hexagonal for Predicates).
In practice, though, in the case of Commands it's more common to choose
the C-shaped slot on the fourth row, because this ``container'' for
command scripts is familiar to Scratch users. Technically the C-shaped
slot is an \emph{unevaluated} procedure type, something discussed in
Section E below. The two Command-related input types (inline and
C-shaped) are connected by the fact that if a variable, an item (\#) of
{[}list{]} block, or a custom Reporter block is dropped onto a C-shaped
slot of a custom block, it turns into an inline slot, as in the repeater
block's recursive call above. (Other built-in Reporters can't report
scripts, so they aren't accepted in a C-shaped slot.)

\strut \\
Why would you ever choose an inline Command slot rather than a C shape?
Other than the run block discussed below, the only case I can think of
is something like the C{[}{]}\{index=``C programming language''\}
/C++/Java{[}{]}\{index=``Java programming language''\} for loop, which
actually has \emph{three} command script inputs (and one predicate
input), only one of which is the ``featured'' loop body:

Okay, now that we have procedures as inputs to our blocks, how do we use
them? We use the blocks run{[}{]}\{index=``run block''\} (for commands)
and call{[}{]}\{index=``call block''\} (for reporters). The run block's
script input is an inline ring, not C-shaped, because we anticipate that
it will be rare to use a specific, literal script as the input. Instead,
the input will generally be a variable whose \emph{value} is a script.

The run and call blocks have arrowheads at the end that can be used to
open slots for inputs to the called procedures. How does Snap\emph{!}
know where to use those inputs? If the called procedure (block or
script) has empty input slots, Snap\emph{!} ``does the right thing.''
This has several possible meanings:

1. If the number of empty slots{[}{]}\{index=``empty input slots,
filling''\} is exactly equal to the number of inputs provided, then
Snap\emph{!} fills the empty slots from left to right:

2. If exactly one input is provided, Snap\emph{!} will fill any number
of empty slots with it:

3. Otherwise, Snap\emph{!} won't fill any slots, because the user's
intention is unclear.

If the user wants to override these rules, the solution is to use a
ring{[}{]}\{index=``ring, gray''\} with explicit input names that can be
put into the given block or script to indicate how inputs are to be
used. This will be discussed more fully below.

\subsection{Recursive Calls to Multiple-Input
Blocks}\label{recursive-calls-to-multiple-input-blocks}

A relatively rare situation not yet considered here is the case of a
recursive block that has a variable number of inputs. Let's say the user
of your project calls your block with five inputs one time, and 87
inputs another time. How do you write the recursive
call{[}{]}\{index=``recursive call''\} to your block when you don't know
how many inputs to give it? The answer is that you collect the inputs in
a list{[}{]}\{index=``input list''\} (recall that, when you declare an
input name to represent a variable number of inputs, your block sees
those inputs as a list of values in the first place), and then, in the
recursive call, you drop that input list \emph{onto the arrowheads} that
indicate a variable-input slot{[}{]}\{index=``variable-input slot''\} ,
rather than onto the input slot:

\strut \\
Note that the halo{[}{]}\{index=``halo:red''\} you see while dragging
onto the arrowheads{[}{]}\{index=``arrowheads''\} is
red{[}{]}\{index=``red halo''\} instead of white, and covers the input
slot as well as the arrowheads. And when you drop the expression onto
the arrowheads, the words ``input list{[}{]}\{index=''input list''\} :''
are added to the block text and the arrowheads disappear (in this
invocation only) to remind you that the list represents all of the
multiple inputs, not just a single input. The items in the list are
taken \emph{individually} as inputs to the script. Since numbers is a
list of numbers, each individual item is a number, just what sizes
wants. This block will take any number of numbers as inputs, and will
make the sprite grow and shrink accordingly:

The user of this block calls it with any number of \emph{individual
numbers} as inputs. But inside the definition of the block, all of those
numbers form \emph{a} \emph{list} that has a single input name, numbers.
This recursive definition first checks to make sure there are any inputs
at all. If so, it processes the first input (item 1 of the list), then it
wants to make a recursive call with all but the first number. But sizes
doesn't take a list as input; it takes numbers as inputs! So this would
be wrong:

\section{Formal Parameters}\label{formal-parameters}

The rings around Procedure-type inputs{[}{]}\{index=``input name''\}
have an arrowhead at the right. Clicking the arrowhead allows you to
give the inputs to a block or script explicit names{[}{]}\{index=``name,
input''\} , instead of using empty input slots as we've done until now.

The names \#1{[}{]}\{index=``\#1''\} , \#2, etc. are provided by
default, but you can change a name by clicking on its orange oval in the
input names list. Be careful not to \emph{drag} the oval when clicking;
that's how you use the input inside the ring. The names of the input
variables are called the \emph{formal parameters}{[}{]}\{index=``formal
parameters''\} of the encapsulated procedure.

Here's a simple but contrived example using explicit names to control
which input goes where inside the ring:

Here we just want to put one of the inputs into two different slots. If
we left all three slots empty, Snap\emph{!} would not fill any of them,
because the number of inputs provided (2) would not match the number of
empty slots (3).

Here is a more realistic, much more advanced
example{[}{]}\{index=``crossproduct''\} :

This is the definition of a block that takes any number of lists, and
reports the list of all possible combinations of one item from each
list. The important part for this discussion is that near the bottom
there are two \emph{nested} calls{[}{]}\{index=``nested calls''\} to
map, the higher order function{[}{]}\{index=``higher order function''\}
that applies an input function to each item of an input list. In the
inner block, the function being mapped is in front of, and that block
takes two inputs. The second, the empty List-type slot, will get its
value in each call from an item of the inner map's list input. But there
is no way for the outer map to communicate values to empty slots of the
in front of block. We must give an explicit name, newitem, to the value
that the outer map is giving to the inner one, then drag that variable
into the in front of block.

By the way, once the called block provides names for its inputs,
Snap\emph{!} will not automatically fill empty
slots{[}{]}\{index=``empty input slots, filling''\} , on the theory that
the user has taken control. In fact, that's another reason you might
want to name the inputs explicitly: to stop Snap\emph{!} from filling a
slot that should really remain empty.

\section{Procedures as Data}\label{procedures-as-data-1}

Here's an example of a situation in which a procedure must be explicitly
marked as data by pulling a ring from the Operators palette and putting
the procedure (block or script) inside it:

Here, we are making a list of procedures{[}{]}\{index=``list of
procedures''\} . But the list block accepts inputs of any type, so its
input slots are not ringed. We must say explicitly that we want the
block \emph{itself} as the input, rather than whatever value would
result from evaluating the block.

Besides the list block in the example above, other blocks into which you
may want to put procedures are set (to set the value of a variable to a
procedure), say and think (to display a procedure to the user), and
report (for a reporter that reports a procedure):

\section{Special Forms}\label{special-forms}

The primitive if else{[}{]}\{index=``if else block''\} block has two
C-shaped command slots and chooses one or the other depending on a
Boolean test. Because Scratch doesn't emphasize functional programming,
it lacks a corresponding reporter block to choose between two
expressions. Snap\emph{!} has one, but we could write our own:

Our block works for these simple examples, but if we try to use it in
writing a recursive operator{[}{]}\{index=``recursive operator''\} ,
it'll fail:

The problem is that when any block is called, all of its inputs are
computed (evaluated) before the block itself runs. The block itself
knows only the values of its inputs, not what expressions were used to
compute them. In particular, all of the inputs to our if then else block
are evaluated first thing. That means that even in the base case,
factorial{[}{]}\{index=``factorial''\} will try to call itself
recursively, causing an infinite loop. We need our if then else block to
be able to select only one of the two alternatives to be evaluated.

We have a mechanism to allow that: declare the then and else inputs to
be of type Reporter rather than type Any. Then, when calling the block,
those inputs will be enclosed in a ring so that the expressions
themselves, rather than their values, become the inputs:

In this version, the program works, with no infinite loop. But we've
paid a heavy price: this reporter-if is no longer as intuitively obvious
as the Scratch command-if. You have to know about procedures as data,
about rings, and about a trick to get a constant value in a ringed
slot{[}{]}\{index=``constant functions''\} . (The id
block{[}{]}\{index=``id block''\} implements the identity
function{[}{]}\{index=``identity function''\} , which reports its
input.{[}6{]} We need it because rings take only reporters as input, not
numbers.) What we'd like is a reporter-if that \emph{behaves} like this
one, delaying the evaluation of its inputs, but \emph{looks} like our
first version, which was easy to use except that it didn't work.

Such blocks are indeed possible. A block that seems to take a simple
expression as input, but delays the evaluation of that input by wrapping
an ``invisible ring'' around it (and, if necessary, an id-like
transformation of constant data into constant functions) is called a
\emph{special form}{[}{]}\{index=``special form''\} . To turn our if
block into a special form, we edit the block's prototype, declaring the
inputs yes and no to be of type ``Any (unevaluated){[}{]}\{index=''Any
(unevaluated) type''\} '' instead of type Reporter. The script for the
block is still that of the second version, including the use of call to
evaluate either yes or no but not both. But the slots appear as white
Any-type rectangles, not Reporter-type rings, and the factorial block
will look like our first attempt.

In a special form's prototype, the
unevaluated{[}{]}\{index=``unevaluated type''\} input slot(s) are
indicated by a lambda (λ) next to the input name, just as if they were
declared as Procedure type{[}{]}\{index=``Procedure type''\} . They
\emph{are} Procedure type, really; they're just disguised to the user of
the block.

Special forms trade off implementor
sophistication{[}{]}\{index=``sophistication''\} for user
sophistication. That is, you have to understand all about procedures as
data to make sense of the special form implementation of my if then
else. But any experienced Scratch programmer can \emph{use} my if then
else without thinking at all about how it works internally.

\subsection{Special Forms in Scratch}\label{special-forms-in-scratch}

Special forms are actually not a new invention in Snap\emph{!}. Many of
Scratch's conditional and looping blocks are really special forms. The
hexagonal input slot in the if block is a straightforward Boolean value,
because the value can be computed once, before the if block makes its
decision about whether or not to run its action input. But the forever
if, repeat until, and wait until blocks' inputs can't be Booleans; they
have to be of type ``Boolean (unevaluated){[}{]}\{index=''Boolean
(unevaluated) type''\} ,'' so that Scratch can evaluate them over and
over again. Since Scratch doesn't have custom C‑shaped blocks, it can
afford to handwave away the distinction between evaluated and
unevaluated Booleans, but Snap\emph{!} can't. The pedagogic value of
special forms is proven by the fact that no Scratcher ever notices that
there's anything strange about the way in which the hexagonal inputs in
the Control blocks are evaluated.

Also, the C-shaped slot{[}{]}\{index=``C-shaped slot''\} familiar to
Scratch users is an unevaluated procedure type; you don't have to use a
ring to keep the commands in the C-slot from being run before the
C-shaped block is run. Those commands themselves, not the result of
running them, are the input to the C-shaped Control block. (This is
taken for granted by Scratch users, especially because Scratchers don't
think of the contents of a C-slot as an input at all.) This is why it
makes sense that ``C‑shaped'' is on the fourth row of types in the long
form input dialog, with other unevaluated types.

\bookmarksetup{startatroot}

\chapter{Object Oriented Programming with
Sprites}\label{object-oriented-programming-with-sprites}

Object oriented programming{[}{]}\{index=``object oriented
programming''\} is a style based around the abstraction \emph{object:} a
collection of \emph{data} and {[}{]}\{index=``method''\} \emph{methods}
(procedures, which from our point of view are just more data) that you
interact with by sending it a {[}{]}\{index=``message''\} \emph{message}
(just a name, maybe in the form of a text string, and perhaps additional
inputs). The object responds to the message by carrying out a method,
which may or may not report a value back to the asker. Some people
emphasize the {[}{]}\{index=``data hiding''\} \emph{data hiding} aspect
of OOP (because each object has local variables that other objects can
access only by sending request messages to the owning object) while
others emphasize the \emph{simulation} aspect (in which each object
abstractly represents something in the world, and the interactions of
objects in the program model real interactions of real people or
things). Data hiding is important for large multi-programmer industrial
projects, but for Snap\emph{!} users it's the
simulation{[}{]}\{index=``simulation''\} aspect that's important. Our
approach is therefore less restrictive than that of some other OOP
languages; we give objects easy access to each others' data and methods.

Technically, object oriented programming rests on three legs: (1)
{[}{]}\{index=``message passing''\} \emph{Message passing:} There is a
notation by which any object can send a message to another object. (2)
{[}{]}\{index=``local state''\} \emph{Local state:} Each object can
remember the important past history of the computation it has performed.
(``Important'' means that it need not remember every message it has
handled, but only the lasting effects of those messages that will affect
later computation.) (3) {[}{]}\{index=``inheritance''\}
\emph{Inheritance:} It would be impractical if each individual object
had to contain methods, many of them identical to those of other
objects, for all of the messages it can accept. Instead, we need a way
to say that this new object is just like that old object except for a
few differences, so that only those differences need be programmed
explicitly.

\section{First Class Sprites}\label{first-class-sprites}

Like Scratch, Snap\emph{!} comes with things that are natural objects:
its sprites{[}{]}\{index=``sprite''\} . Each sprite can own local
variables; each sprite has its own scripts (methods). A Scratch
animation is plainly a simulation of the interaction of characters in a
play. There are two ways in which Scratch sprites are less versatile
than the objects of an OOP language. First, Scratch message passing is
weak in three respects: Messages can only be
broadcast{[}{]}\{index=``broadcast block''\} , not addressed to an
individual sprite; messages can't take inputs; and methods can't return
values to their caller. Second, and more basic, in the OOP paradigm
objects are \emph{data;} they can be the value of a variable, an element
of a list, and so on, but that's not the case for Scratch sprites.

Snap\emph{!} sprites are first class{[}{]}\{index=``first class
sprites''\} data. They can be created and deleted by a script, stored in
a variable or list, and sent messages individually. The children of a
sprite can inherit sprite-local variables, methods (sprite-local
procedures), and other attributes (e.g., x position).

The fundamental means by which programs get access to sprites is the my
reporter block{[}{]}\{index=``my block''\} . It has a dropdown-menu
input slot that, when clicked, gives access to all the sprites, plus the
stage{[}{]}\{index=``stage''\} . reports a single sprite, the one asking
the question. reports a list of all sprites other than the one asking
the question. reports a list of all sprites that are \emph{near} the one
asking---the ones that are candidates for having collided with this one,
for example. The my block{[}{]}\{index=``my block''\} has many other
options, discussed below. If you know the name of a particular sprite,
the object reporter will report the sprite itself.

An object or list of objects reported by my or object can be used as
input to any block that accepts any input type, such as set's second
input. If you say an object, the resulting speech balloon will contain a
smaller image of the object's costume or (for the stage) background.

\section{Permanent and Temporary
Clones}\label{permanent-and-temporary-clones}

The block is used to create and report an instance (a clone) of any
sprite. (There is also a command version, for historical reasons.) There
are two different kinds of situations in which clones are used. One is
that you've made an example sprite and, when you start the project, you
want a fairly large number of essentially identical sprites that behave
like the example. (Hereafter we'll call the example sprite the
``parent'' and the others the ``children.'') Once the game or animation
is over, you don't need the copies any more. (As we'll see, ``copies''
is the wrong word because the parent and the children \emph{share} a lot
of properties. That's why we use the word ``clones'' to describe the
children rather than ``copies.'') These are \emph{temporary}
clones{[}{]}\{index=``temporary clone''\} . They are automatically
deleted when the user presses either the green flag or the red stop
sign. In Scratch 2.0 and later, all
clones{[}{]}\{index=``clone:temporary''\} are temporary.

The other kind of situation is what happens when you want
specializations of sprites. For example, let's say you have a sprite
named Dog. It has certain behaviors, such as running up to a person who
comes near it. Now you decide that the family in your story really likes
dogs, so they adopt a lot of them. Some are cocker spaniels, who wag
their tails when they see you. Others are rottweilers, who growl at you
when they see you. So you make a clone of Dog, perhaps rename it Cocker
Spaniel, and give it a new costume and a script for what to do when
someone gets near. You make another clone of Dog, perhaps rename it
Rottweiler, and give it a new costume, etc. Then you make three clones
of Cocker Spaniel (so there are four altogether) and two clones of
Rottweiler. Maybe you hide the Dog sprite after all this, since it's no
breed in particular. Each dog has its own position, special behaviors,
and so on. You want to save all of these dogs in the project. These are
\emph{permanent} clones{[}{]}\{index=``permanent clone''\} . In BYOB
3.1, the predecessor to Snap\emph{!,} all
clones{[}{]}\{index=``clone:permanent''\} are permanent.

One advantage of temporary clones is that they don't slow down
Snap\emph{!} even when you have a lot of them. (If you're curious, one
reason is that permanent clones appear in the sprite corral, where their
pictures have to be updated to reflect the clone's current costume,
direction, and so on.) We have tried to anticipate your needs, as
follows: When you make a clone in a script, using the block, it is
``born'' temporary. But when you make a clone from the user interface,
for example by right-clicking on a sprite and choosing ``clone,'' it is
born permanent. The reason this makes sense is that you don't create 100
\emph{kinds} of dogs automatically. Each kind has many different
characteristics, programmed by hand. But when your project is running,
it might create 100 rottweilers, and those will be identical unless you
change them in the program.

You can change a temporary sprite to permanent by right-clicking it and
choosing ``edit.'' (It's called ``edit'' rather than, say, ``permanent''
because it also shifts the scripting area to reflect that sprite, as if
you'd pressed its button in the sprite corral.) You can change a
permanent sprite to temporary by right-clicking it and choosing
``release.'' You can also change the status of a clone in your program
with with true or false as the second input.

\section{Sending Messages to Sprites}\label{sending-messages-to-sprites}

The messages that a sprite accepts are the blocks in its palettes,
including both all-sprites and this-sprite-only blocks. (For custom
blocks, the corresponding methods are the scripts as seen in the Block
Editor.

The way to send a message to a sprite (or the stage) is with the tell
block (for command messages) or the ask block (for reporter messages).

A small point to note in the examples above: all dropdown menus include
an empty entry at the top, which can be selected for use in higher order
procedures like the for each and map examples. Each of the sprites in my
neighbors or my other sprites is used to fill the blank space in turn.

By the way, if you want a list of \emph{all} the sprites, including this
sprite, you can use either of these:

Tell and ask wait until the other sprite has carried out its method
before this sprite's script continues. (That has to be the case for ask,
since we want to do something with the value it reports.) So tell is
analogous to broadcast and wait. Sometimes the other sprite's method may
take a long time, or may even be a forever loop, so you want the
originating script to continue without waiting. For this purpose we have
the launch block:

Launch is analogous to broadcast without the ``wait.''

Snap\emph{!} 4.1, following BYOB 3.1, used an extension of the of block
to provide access to other sprites' methods. That interface was designed
back when we were trying hard to avoid adding new primitive blocks; it
allowed us to write ask and tell as tool procedures in Snap\emph{!}
itself. That technique still works, but is deprecated, because nobody
understood it, and now we have the more straightforward primitives.

\subsection{Polymorphism{[}{]}}\label{polymorphism}

Suppose you have a Dog sprite with two clones CockerSpaniel and PitBull.
In the Dog sprite you define this method{[}{]}\{index=``method''\}
(``For this sprite only'' block{[}{]}\{index=``block:sprite-local''\} ):

Note the \emph{loca}tion (map-pin) symbol{[}{]}\{index=``map-pin
symbol''\} before the block's name. The symbol is not part of the block
title; it's a visual reminder that this is a sprite-\emph{loca}l block.
Sprite-local variables are similarly marked.

But you don't define greet as friend or greet as enemy in Dog. Each kind
of dog has a different behavior. Here's what a CockerSpaniel does:

And here's what a PitBull does:

Greet is defined in the Dog sprite. If Fido is a particular cocker
spaniel, and you ask Fido to greet someone, Fido inherits the greet
method from Dog, but Dog itself couldn't actually run that method,
because Dog doesn't have greet as friend or greet as enemy. And perhaps
only individual dogs such as Fido have friend? methods. Even though the
greet method is defined in the Dog sprite, when it's running it
remembers what specific dog sprite called it, so it knows which greet as
friend to use. Dog's greet block is called a \emph{polymorphic} method,
because it means different things to different dogs, even though they
all share the same script.

\section{Local State in Sprites: Variables and
Attributes}\label{local-state-in-sprites-variables-and-attributes}

A sprite's memory of its own past history takes two main forms. It has
\emph{variables,} created explicitly by the user with the ``Make a
variable{[}{]}\{index=''variable''\} '' button; it also has
\emph{attributes,} the qualities every sprite has automatically, such as
position, direction, and pen color. Each variable can be examined using
its own orange oval block; there is one set block to modify all
variables. Attributes, however, have a less uniform programming
interface in Scratch:

\begin{itemize}
\item
  A sprite's \emph{direction} can be examined with the direction block,
  and modified with the point in direction block. It can also be
  modified less directly using the blocks turn, point towards, and if on
  edge, bounce.
\item
  There is no way for a script to examine a sprite's \emph{pen color,}
  but there are blocks set pen color to \textless color\textgreater, set
  pen color to \textless number\textgreater, and change pen color to
  modify it.
\item
  A sprite's \emph{name} can be neither examined nor modified by
  scripts; it can be modified by typing a new name directly into the box
  that displays the name, above the scripting area.
\end{itemize}

The block, if any, that examines a variable or
attribute{[}{]}\{index=``attribute''\} is called its
{[}{]}\{index=``getter''\} \emph{getter;} a block (there may be more
than one, as in the direction example above) that modifies a variable or
attribute is called a {[}{]}\{index=``setter''\} \emph{setter.}

In Snap\emph{!} we allow virtually all attributes to be examined. But
instead of adding dozens of reporters, we use a more uniform interface
for attributes: The my block{[}{]}\{index=``my block''\} 's menu (in
Sensing; see page \hyperref[attrib.pnglist-of-attributes]{78}) includes
many of the attributes of a sprite. It serves as a general getter for
those attributes, e.g., my {[}anchor{]} to find the sprite, if any, to
which this sprite is attached in a nesting arrangement (see page
\hyperref[nesting-sprites-anchors-and-parts]{10}). Similarly, the same
set block used to set variable values allows setting some sprite
attributes.

\section{Prototyping: Parents and
Children}\label{prototyping-parents-and-children}

Most current OOP languages use a \emph{class/instance} approach to
creating objects. A class is a particular \emph{kind of object,} and an
instance is an \emph{actual object} of that type. For example, there
might be a Dog class, and several instances Fido, Spot, and Runt. The
class typically specifies the methods shared by all dogs (RollOver,
SitUpAndBeg, Fetch, and so on), and the instances contain data such as
species, color, and friendliness. Snap\emph{!} uses a different approach
called \emph{prototyping,} in which there is no distinction between
classes and instances. Prototyping{[}{]}\{index=``prototyping''\} is
better suited to an experimental, tinkering style of work: You make a
single dog sprite, with both methods (blocks) and data (variables); you
can actually watch it and interact with it on the stage; and when you
like it, you use it as the prototype from which to clone other dogs. If
you later discover a bug in the behavior of dogs, you can edit a method
in the parent, and all of the children will automatically share the new
version of the method block. Experienced
class/instance{[}{]}\{index=``class/instance''\} programmers may find
prototyping{[}{]}\{index=``prototyping''\} strange at first, but it is
actually a more expressive system, because you can easily simulate a
class/instance hierarchy by hiding the prototype sprite! Prototyping is
also a better fit with the Scratch design
principle{[}{]}\{index=``design principle''\} that everything in a
project should be concrete and visible on the stage; in class/instance
OOP the programming process begins with an abstract, invisible entity,
the class, that must be designed before any concrete objects can be
made.{[}7{]}

There are three ways to make a child sprite. If you control-click or
right-click on a sprite in the ``sprite corral'' at the bottom right
corner of the window, you get a menu that includes ``clone'' as one of
the choices. There is an a new clone of block{[}{]}\{index=``a new clone
of block''\} in the Control palette that creates and reports a child
sprite. And sprites have a ``parent'' attribute{[}{]}\{index=``parent
attribute''\} that can be set, like any attribute, thereby
\emph{changing} the parent of an existing sprite.

\section{Inheritance by Delegation}\label{inheritance-by-delegation}

A clone \emph{inherits} properties of its parent. ``Properties'' include
scripts, custom blocks, variables, named lists, system attributes,
costumes, and sounds. Each individual property can be shared between
parent and child, or not shared (with a separate one in the child). The
getter block for a shared property, in the child's palette, is displayed
in a lighter color; separate properties of the child are displayed in
the traditional colors.

\begin{quote}
When a new clone is created, by default it shares only its methods,
wardrobe, and jukebox with its parent. All other properties are copied
to the clone, but not shared. (One exception is that a new
\emph{permanent} clone is given a random position. Another is that
\emph{temporary} clones share the scripts in their parent's scripting
area. A third is that sprite-local variables that the parent creates
\emph{after} cloning are shared with its children.) If the value of a
shared property is changed in the parent, then the children see the new
value. If the value of a shared property is changed in the \emph{child},
then the sharing link is broken, and a new private version is created in
that child. (This is the mechanism by which a child chooses not to share
a property with its parent.) ``Changed'' in this context means using the
set or change block for a variable, editing a block in the Block Editor,
editing a costume or sound, or inserting, deleting, or reordering
costumes or sounds. To change a property from unshared to shared, the
child uses the inherit command block. The pulldown menu in the block
lists all the things this sprite can inherit from its parent (which
might be nothing, if this sprite has no parent) and is not already
inheriting. But that would prevent telling a child to inherit, so if the
inherit block{[}{]}\{index=``inherit block''\} is inside a ring, its
pulldown menu includes all the things a child could inherit from this
sprite. Right-clicking on the scripting area of a permanent clone gives
a menu option to share the entire collection of scripts from its parent,
as a temporary clone does.
\end{quote}

The rules are full of details, but the basic idea is simple: Parents can
change their children, but children can't directly change their parents.
That's what you'd expect from the word ``inherit'': the influence just
goes in one direction. When a child changes some property, it's
declaring independence from its parent (with respect to that one
property). What if you really want the child to be able to make a change
in the parent (and therefore in itself and all its siblings)? Remember
that in this system any object can tell any other object to do
something:

When a sprite gets a message for which it doesn't have a corresponding
block, the message is \emph{delegated} to that sprite's parent. When a
sprite does have the corresponding block, then the message is not
delegated. If the script that implements a delegated message refers to
my (self), it means the child to which the message was originally sent,
not the parent to which the message was delegated.

\section{List of attributes}\label{list-of-attributes}

At the right is a picture of the dropdown menu of
attributes{[}{]}\{index=``attributes, list of''\} in the my block.

Several of these are not real attributes, but things related to
attributes:

\begin{itemize}
\item
  self{[}{]}\{index=``self (in my block)''\} : this sprite
\item
  neighbors{[}{]}\{index=``neighbors (in my block)''\} : a list of
  \emph{nearby} sprites{[}8{]}
\item
  other sprites{[}{]}\{index=``other sprites (in my block)''\} : a list
  of all sprites except myself
\item
  stage{[}{]}\{index=``stage (in my block)''\} : the stage, which is
  first-class, like a sprite
\item
  clones{[}{]}\{index=``clones (in my block)''\} : a list of my
  \emph{temporary} clones
\item
  other clones{[}{]}\{index=``other clones (in my block)''\} : a list of
  my \emph{temporary} siblings
\item
  parts{[}{]}\{index=``parts (in my block)''\} : a list of sprites whose
  anchor attribute is this sprite
\item
  children{[}{]}\{index=``children (in my block)''\} : a list of all my
  clones, temporary and permanent
\end{itemize}

The others are individual attributes:

\begin{itemize}
\item
  anchor{[}{]}\{index=``anchor (in my block)''\} : the sprite of which I
  am a (nested) part
\item
  parent{[}{]}\{index=``parent (in my block)''\} : the sprite of which I
  am a clone
\item
  temporary?: am I a temporary clone?
\item
  name{[}{]}\{index=``name (in my block)''\} : my name (same as parent's
  name if I'm temporary)
\item
  costumes{[}{]}\{index=``costumes (in my block)''\} : a list of the
  sprite's costumes
\item
  sounds{[}{]}\{index=``sounds (in my block)''\} : a list of the
  sprite's sounds
\item
  blocks: a list of the blocks visible in this sprite
\item
  categories: a list of all the block category names
\item
  dangling?{[}{]}\{index=``dangling? (in my block)''\} : True if I am a
  part and not in synchronous orbit
\item
  draggable?: True if the user can move me with the mouse
\item
  width, height, left, right, top, bottom: The width or height of my
  costume \emph{as seen right now,} or the left, etc., edge of my
  bounding box, taking rotation into account.
\item
  rotation x{[}{]}\{index=``rotation x (in my block)''\} , rotation
  y{[}{]}\{index=``rotation y (in my block)''\} : when reading with my,
  the same as x position, y~position. When set, changes the sprite's
  rotation center \emph{without moving the sprite,} like dragging the
  rotation center in the paint editor.
\item
  center x{[}{]}\{index=``center x (in my block)''\} , center
  y{[}{]}\{index=``center y (in my block)''\} : the x and y position of
  the center of my
\end{itemize}

\begin{quote}
bounding box, rounded off--the geometric center of the costume.
\end{quote}

\section{First Class Costumes and
Sounds}\label{first-class-costumes-and-sounds}

Costumes and sounds don't have methods, as sprites do; you can't ask
them to do things. But they \emph{are} first
class:{[}{]}\{index=``costumes, first class''\} you can make a list of
them, put them in variables, use them as input to a procedure, and so
on. My {[}costumes{]} and my {[}sounds{]} report lists of them.

\subsection{Media Computation with
Costumes}\label{media-computation-with-costumes}

The components of a costume are its name, width, height, and pixels. The
block gives access to these components {[}{]}\{index=``of costume
block''\} using its left menu. From its right menu you can choose the
current costume, the Turtle costume, or any costume in the sprite's
wardrobe. Since costumes are first class, you can also drop an
expression whose value is a costume, or a list of costumes, on that
second input slot. (Due to a misfeature, even though you can select
Turtle in the right menu, the block reports 0 for its width and height,
and an empty string for the other components.) The costume's width and
height are in its standard orientation, regardless of the sprite's
current direction. (This is different from the \emph{sprite's} width and
height, reported by the my block.)

But the really interesting part of a costume is its
bitmap{[}{]}\{index=``bitmap''\} , a list of \emph{pixels}. (A
pixel{[}{]}\{index=``pixel''\} , short for ``picture element,''
represents one dot on your display.) Each pixel is itself a list of four
items, the red, green, and blue components of its color (in the range
0-255) and what is standardly called its
``transparency{[}{]}\{index=''transparency''\} '' but should be called
its opacity, also in the range 0-255, in which 0 means that the pixel is
invisible and 255 means that it's fully opaque: you can't see anything
from a rearward layer at that point on the stage. (Costume pixels
typically have an opacity of 0 only for points inside the bounding box
of the costume but not actually part of the costume; points in the
interior of a costume typically have an opacity of 255. Intermediate
values appear mainly at the edge of a costume, or at sharp boundaries
between colors inside the costume, where they are used to reduce
``jaggies{[}{]}\{index=''jaggies''\} '': the stairstep-like shape of a
diagonal line displayed on an array of discrete rectangular screen
coordinates. Note that the opacity of a \emph{sprite} pixel is
determined by combining the costume's opacity with the sprite's ghost
effect. (The latter really is a measure of transparency: 0 means opaque
and 100 means invisible.)

The bitmap is a one-dimensional list of pixels, not an array of
\emph{height} rows of \emph{width} pixels each. That's why the pixel
list has to be combined with the dimensions to produce a costume. This
choice partly reflects the way bitmaps are stored in the computer's
hardware and operating system, but also makes it easy to produce
transformations of a costume with map:

In this simplest possible transformation, the red value of all the
pixels have been changed to a constant 150. Colors that were red in the
original (such as the logo printed on the t-shirt) become closer to
black (the other color components being near zero); the blue jeans
become purple (blue plus red); perhaps counterintuitively, the white
t-shirt, which has the maximum value for all three color components,
loses some of its red and becomes cyan, the color opposite red on the
color wheel. In reading the code, note that the function that is the
first input to map is applied to a single pixel, whose first item is its
red component. Also note that this process works only on bitmap
costumes; if you call pixels of on a vector costume (one with ``svg'' in
the corner of its picture), it will be converted to pixels first.

One important point to see here is that a bitmap (list of pixels) is
not, by itself, a costume. The new costume block{[}{]}\{index=``new
costume block''\} creates a costume by combining a bitmap, a width, and
a height. But, as in the example above, switch to costume will accept a
bitmap as input and will automatically use the width and height of the
current costume. Note that there's no name input; costumes computed in
this way are all named costume. Note also that the use of switch to
costume does \emph{not} add the computed costume to the sprite's
wardrobe; to do that, say

Here's a more interesting example of color manipulation:

Each color value is constrained to be 0, 80, 160, or 240. This gives the
picture a more cartoonish look. Alternatively, you can do the
computation taking advantage of hyperblocks:

Here's one way to exchange red and green values:

It's the list that determines the rearrangement of colors: green➔red,
red➔green, and the other two unchanged. That list is inside another list
because otherwise it would be selecting \emph{rows} of the pixel array,
and we want to select columns. We use pixels of costume current rather
than costume apple because the latter is always a red apple, so this
little program would get stuck turning it green, instead of alternating
colors.

The stretch block{[}{]}\{index=``stretch block''\} takes a costume as
its first input, either by selecting a costume from the menu or by
dropping a costume-valued expression such as onto it. The other two
inputs are percents of the original width and height, as advertised, so
you can make fun house mirror versions of costumes:

The resulting costumes can be used with switch to costume and so on.

Finally, you can use pictures from your computer's camera in your
projects using these blocks:

Using the video on block{[}{]}\{index=``video on block''\} turns on the
camera and displays what it sees on the stage, regardless of the inputs
given. The camera remains on until you click the red stop button, your
program runs the stop all block, or you turn it off explicitly with the
block. The video image on the stage is partly ghosted, to an extent
determined by the set video transparency block, whose input really is
transparency and not opacity. (Small numbers make the video more
visible.) By default, the video image is mirrored, like the selfie
camera on your cell phone: When you raise your left hand, your image
raises its right hand. You can control this mirroring with the block.

The video snap on block then takes a still picture from the camera, and
trims it to fit on the selected sprite. (Video snap on stage means to
use the entire stage-sized rectangle.) For example, here's a camera
snapshot trimmed to fit Alonzo:

The ``Video Capture'' project in the Examples collection repeatedly
takes such trimmed snapshots and has the Alonzo sprite use the current
snapshot as its costume, so it looks like this:

(The picture above was actually taken with transparency set to 50, to
make the background more visible for printing.) Because the sprite is
always still in the place where the snapshot was taken, its costume
exactly fits in with the rest of the full-stage video. If you were to
add a move 100 steps block after the switch to costume, you'd see
something like this:

This time, the sprite's costume was captured at one position, and then
the sprite is shown at a different position. (You probably wouldn't want
to do this, but perhaps it's helpful for explanatory purposes.)

What you \emph{would} want to do is push the sprite around the stage:

(Really these should be Jens's picture; it's his project. But he's
vacationing. ☺) Video motion compares two snapshots a moment apart,
looking only at the part within the given trim (here myself, meaning the
current sprite, not the person looking into the camera), to detect a
difference between them. It reports a number, measuring the number of
pixels through which some part of the picture has moved. Video direction
also compares two snapshots to detect motion, but what it reports is the
direction (in the point in direction sense) of the motion. So the script
above moves the sprite in the direction in which it's being pushed, but
only if a significant amount of motion is found; otherwise the sprite
would jiggle around too much. And yes, you can run the second script
without the first to push a balloon around the stage.

\subsection{Media Computation with
Sounds}\label{media-computation-with-sounds}

The starting point for computation with sound{[}{]}\{index=``sound''\}
is the microphone block{[}{]}\{index=``microphone block''\} . It starts
by recording a brief burst of sound from your
microphone{[}{]}\{index=``microphone''\} . (How brief? On my computer,
0.010667 seconds, but you'll see shortly how to find out or control the
sample size on your computer.)

Just as the \emph{pixel} is the smallest piece of a picture, the
\emph{sample} is the smallest piece of a sound. It says here: that on my
computer, 48,000 samples are recorded per second, so each
sample{[}{]}\{index=``sample''\} is 1/48,000 of a second. The value of a
sample is between -1 and 1, and represents the sound pressure on the
microphone---how hard the air is pushing---at that instant. (You can
skip the next page or so if you know about Fourier analysis.) Here's a
picture of 400 samples:

In this graph, the \emph{x} axis represents the time at which each
sample was measured; the \emph{y} axis measures the value of the sample
at that time. The first obvious thing about this graph is that it has a
lot of ups and downs. The most basic up-and-down function is the
\emph{sine wave:}

Every periodic function (more or less, any sample that sounds like music
rather than sounding like static) is composed of a sum of sine
wave{[}{]}\{index=``sine wave''\} s of different frequencies.

Look back at the graph of our sampled sound. There is a green dot every
seven samples. There's nothing magic about the number seven; I tried
different values until I found one that looked right. What ``right''
means is that, for the first few dots at least, they coincide almost
perfectly with the high points and low points of the graph. Near the
middle (horizontally) of the graph, the green dots don't seem anywhere
near the high and low points, but if you find the very lowest point of
the graph, about 2/3 of the way along, the dots start lining up almost
perfectly again.

The red graph above shows two \emph{cycles} of a sine wave. One cycle
goes up, then down, then up again. The amount of time taken for one
cycle is the \emph{period} of the sine function. If the green dots match
both ups and downs in the captured sound, then two dots---14 samples, or
14/48000 of a second---represent the period. The first cycle and a half
of the graph looks like it could be a pure sine wave, but after that,
the tops and bottoms don't line up, and there are peculiar little
jiggles, such as the one before the fifth green dot. This happens
because sine waves of different periods are added together.

It turns out to be more useful to measure the reciprocal of the period,
in our case, 48000/14 or about 3429 \emph{cycles per second.} Another
name for ``cycles per second'' is ``Hertz,'' abbreviated Hz, so our
sound has a component at 3249 Hz. As a musical note, that's about an A
(a little flat), four octaves above middle C. (Don't worry too much
about the note being a little off; remember that the 14-sample period
was just eyeballed and is unlikely to be exactly right.)

Four octaves above middle C is really high! That would be a
shrill-sounding note. But remember that a complex waveform is the sum of
multiple sine waves at different frequency. Here's a different
up-and-down regularity:

It's not obvious, but in the left part of the graph, the signal is more
above the \emph{x} axis than below it. Toward the right, it seems to be
more below than above the axis. At the very right it looks like it might
be climbing again.

The period of the red sine wave is 340 samples, or 340/48000 second.
That's a frequency of about 141 Hz, about D below middle C. Again, this
is measuring by eyeball, but likely to be close to the right frequency.

All this eyeballing doesn't seem very scientific. Can't we just get the
computer to find all the relevant frequencies? Yes, we can, using a
mathematical technique called \emph{Fourier analysis.} (Jean-Baptiste
Joseph Fourier, 1768--1830, made many contributions to mathematics and
physics, but is best known for working out the nature of periodic
functions as a sum of sine waves.) Luckily we don't have to do the math;
the microphone block will do it for us, if we ask for microphone
spectrum:

These are frequency spectra from (samples of) three different songs. The
most obvious thing about these graphs is that their overall slope is
downward; the loudest frequency is the lowest frequency. That's typical
of music.

The next thing to notice is that there's a regularity in the spacing of
spikes in the graph. This is partly just an artifact; the frequency
(horizontal) axis isn't continuous. There are a finite number of
``buckets'' (default: 512), and all the frequencies within a bucket
contribute to the amplitude (vertical axis) of that bucket. The spectrum
is a list of that many amplitudes. But the patterns of alternating
rising and falling values are real; the frequencies that are multiples
of the main note being sampled will have higher amplitude than other
frequencies.

Samples and spectrum are the two most detailed representations of a
sound. But the microphone block has other, simpler options also:

volume the instantaneous volume when the block is called

note the MIDI note number (as in play note) of the main note heard

frequency the frequency in Hz of the main note heard

sample rate the number of samples being collected per second

resolution the size of the array in which data are collected (typically
512, must be a power of 2)

The block for sounds that corresponds to new picture for pictures
is{[}{]}\{index=``new sound block''\}

Its first input is a list of samples, and its second input specifies how
many samples occupy one second.

\bookmarksetup{startatroot}

\chapter{OOP with Procedures}\label{oop-with-procedures}

The idea of object oriented programming{[}{]}\{index=``object oriented
programming''\} is often taught in a way that makes it seem as if a
special object oriented programming language is necessary. In fact, any
language with first class procedures and lexical
scope{[}{]}\{index=``scope:lexical''\} allows objects to be implemented
explicitly; this is a useful exercise to help demystify objects.

The central idea of this implementation is that an object is represented
as a {[}{]}\{index=``dispatch procedure''\} \emph{dispatch procedure}
that takes a message as input and reports the corresponding method. In
this section we start with a stripped-down example to show how local
state works, and build up to full implementations of class/instance and
prototyping OOP.

\section{Local State with Script
Variables}\label{local-state-with-script-variables}

This script implements an object \emph{class}, a type of object, namely
the counter class{[}{]}\{index=``counter class''\} . In this first
simplified version there is only one method, so no explicit message
passing is necessary. When the make a counter block is called, it
reports a procedure, the ringed script inside its body. That procedure
implements a specific counter object, an
\emph{instance}{[}{]}\{index=``instance''\} of the counter
class{[}{]}\{index=``class''\} . When invoked, a counter instance
increases and reports its count variable. Each counter has its own local
count:{[}{]}\{index=``objects, building explicitly''\}

This example will repay careful study, because it isn't obvious why each
instance has a separate count. From the point of view of the make a
counter procedure, each invocation causes a new count variable to be
created. Usually such \emph{script variables} are temporary, going out
of existence when the script ends. But this one is special, because make
a counter returns \emph{another script} that makes reference to the
count variable, so it remains active. (The script
variables{[}{]}\{index=``script variables block''\} block makes
variables local to a script. It can be used in a sprite's script area or
in the Block Editor. Script variables can be ``exported'' by being used
in a reported procedure, as here.)

In this approach to OOP, we are representing both classes and instances
as procedures. The make a counter block represents the class, while each
instance is represented by a nameless script created each time make a
counter is called. The script variables created inside the make a
counter block but outside the ring are \emph{instance variables,}
belonging to a particular counter.

\section{Messages and Dispatch
Procedures}\label{messages-and-dispatch-procedures}

In the simplified class above, there is only one method, and so there
are no messages; you just call the instance to carry out its one method.
Here is a more refined version that uses message
passing{[}{]}\{index=``message passing''\} :

Again, the make a counter block represents the counter class, and again
the script creates a local variable count each time it is invoked. The
large outer ring represents an instance. It is a \emph{dispatch
procedure}{[}{]}\{index=``dispatch procedure''\} \emph{:} it takes a
message (just a text word) as input, and it reports a method. The two
smaller rings are the methods. The top one is the next method; the
bottom one is the reset method. The latter requires an input, named
value.

In the earlier version, calling the instance did the entire job. In this
version, calling the instance gives access to a
method{[}{]}\{index=``method''\} , which must then be called to finish
the job. We can provide a block to do both procedure calls in one:

The ask block{[}{]}\{index=``ask block''\} has two required inputs: an
object and a message. It also accepts optional additional inputs, which
Snap\emph{!} puts in a list; that list is named args inside the block.
Ask has two nested call blocks. The inner one calls the object, i.e.,
the dispatch procedure. The dispatch procedure always takes exactly one
input, namely the message. It reports a method, which may take any
number of inputs; note that this is the situation in which we drop a
list of values onto the arrowheads of a multiple input (in the outer
call block). Note also that this is one of the rare cases in which we
must unringify{[}{]}\{index=``unringify''\} the inner call block, whose
\emph{value when called} gives the method.

\section{Inheritance via Delegation}\label{inheritance-via-delegation}

So, our objects now have local state variables and message passing. What
about inheritance{[}{]}\{index=``inheritance''\} ? We can provide that
capability using the technique of
\emph{delegation}{[}{]}\{index=``delegation''\} . Each instance of the
child class{[}{]}\{index=``child class''\} contains an instance of the
parent class{[}{]}\{index=``parent class''\} , and simply passes on the
messages it doesn't want to specialize:

This script implements the buzzer class, which is a child of counter.
Instead of having a count (a number) as a local state variable, each
buzzer has a counter (an object) as a local state variable. The class
specializes the next method, reporting what the counter reports unless
that result is divisible by 7, in which case it reports ``buzz.'' (Yeah,
it should also check for a digit 7 in the number, but this code is
complicated enough already.) If the message is anything other than next,
though, such as reset, then the buzzer simply invokes its counter's
dispatch procedure. So the counter handles any message that the buzzer
doesn't handle explicitly. (Note that in the non-next case we call the
counter, not ask it something, because we want to report a method, not
the value that the message reports.) So, if we ask a buzzer to reset to
a value divisible by 7, it will end up reporting that number, not
``buzz.''

\section{An Implementation of Prototyping
OOP}\label{an-implementation-of-prototyping-oop}

In the class/instance system above, it is necessary to design the
complete behavior of a class before you can make any instances of the
class. This is okay for top-down design, but not great for
experimentation. Here we sketch the implementation of a
\emph{prototyping}{[}{]}\{index=``prototyping''\} OOP system: You make
an object, tinker with it, make clones of it, and keep tinkering. Any
changes you make in the parent are inherited by its children. In effect,
that first object is both the class and an instance of the class. In the
implementation below, children share properties (methods and local
variables) of their parent unless and until a child changes a property,
at which point that child gets a private copy. (If a child wants to
change something for its entire family, it must ask the parent to do
it.)

Because we want to be able to create and delete properties dynamically,
we won't use Snap\emph{!} variables to hold an object's variables or
methods. Instead, each object has two \emph{tables,} called methods and
data, each of which is an {[}{]}\{index=``association list''\}
\emph{association list:} a list of two-item lists, in which each of the
latter contains a \emph{key} and a corresponding \emph{value.} We
provide a lookup procedure to locate the key-value
pair{[}{]}\{index=``key-value pair''\} corresponding to a given key in a
given table.

There are also commands to insert and delete entries:

As in the class/instance version, an object is represented as a dispatch
procedure{[}{]}\{index=``dispatch procedure''\} that takes a message as
its input and reports the corresponding method. When an object gets a
message, it will first look for that keyword in its methods
table{[}{]}\{index=``methods table''\} . If it's found, the
corresponding value is the method we want. If not, the object looks in
its data table{[}{]}\{index=``data table''\} . If a value is found
there, what the object returns is \emph{not} that value, but rather a
reporter method that, when called, will report the value. This means
that what an object returns is \emph{always} a method.

If the object has neither a method nor a datum with the desired name,
but it does have a parent, then the parent (that is, the parent's
dispatch procedure) is invoked with the message as its input.
Eventually, either a match is found, or an object with no parent is
found; the latter case is an error, meaning that the user has sent the
object a message not in its repertoire.

Messages can take any number of inputs, as in the class/instance system,
but in the prototyping version, every method automatically gets the
object to which the message was originally sent as an extra first input.
We must do this so that if a method is found in the parent (or
grandparent, etc.) of the original recipient, and that method refers to
a variable or method, it will use the child's variable or method if the
child has its own version.

The clone of block{[}{]}\{index=``clone of block''\} below takes an
object as its input and makes a child object. It should be considered as
an internal part of the implementation; the preferred way to make a
child of an object is to send that object a clone message.

Every object is created with predefined methods for set, method,
delete-var, delete-method, and clone. It has one predefined variable,
parent. Objects without a parent are created by calling new object:

As before, we provide procedures to call an object's dispatch procedure
and then call the method. But in this version, we provide the desired
object as the first method input. We provide one procedure for Command
methods and one for Reporter methods:

(Remember that the ``Input list:'' variant of the run and call blocks is
made by dragging the input expression over the arrowheads rather than
over the input slot.)

The script below demonstrates how this prototyping system can be used to
make counters. We start with one prototype counter, called counter1. We
count this counter up a few times, then create a child counter2 and give
it its own count variable, but \emph{not} its own total variable. The
next method always sets counter1's total variable, which therefore keeps
count of the total number of times that \emph{any} counter is
incremented. Running this script should {[}say{]} and (think) the
following lists:

{[}1 1{]} {[}2 2{]} {[}3 3{]} {[}4 4{]} (1 5) (2 6) (3 7) {[}5 8{]} {[}6
9{]} {[}7 10{]} {[}8 11{]}

\bookmarksetup{startatroot}

\chapter{The Outside World}\label{the-outside-world}

The facilities discussed so far are fine for projects that take place
entirely on your computer's screen. But you may want to write programs
that interact with physical devices{[}{]}\{index=``devices''\}
(sensors{[}{]}\{index=``sensors''\} or robots{[}{]}\{index=``robots''\}
) or with the World Wide Web{[}{]}\{index=``World Wide Web''\} . For
these purposes Snap\emph{!} provides a
\phantomsection\label{url}{}single primitive block:

This might not seem like enough, but in fact it can be used to build the
desired capabilities.

\section{The World Wide Web}\label{the-world-wide-web}

The input to the url block{[}{]}\{index=``url block''\} is the URL
(Uniform Resource Locator{[}{]}\{index=``Uniform Resource Locator''\} )
of a web page. The block reports the body of the Web server's response
(minus HTTP header), \emph{without interpretation.} This means that in
most cases the response is a description of the page in HTML (HyperText
Markup Language){[}{]}\{index=``HTML (HyperText Markup Language)''\}
notation. Often, especially for commercial web sites, the actual
information you're trying to find on the page is actually at another URL
included in the reported HTML. The Web page is typically a very long
text string, and so the primitive split block{[}{]}\{index=``split
block''\} is useful to get the text in a manageable form, namely, as a
list of lines:

The second input to split is the character to be used to separate the
text string into a list of lines, or one of a set of common cases (such
as line, which separates on carriage return and/or newline characters.

This might be a good place for a reminder that list-view watchers scroll
through only 100 items at a time. The downarrow near the bottom right
corner of the speech balloon in the picture presents a menu of
hundred-item ranges. (This may seem unnecessary, since the scroll bar
should allow for any number of items, but doing it this way makes
Snap\emph{!} much faster.) In table view, the entire list is included.

If you include a protocol name in the input to the url block (such as
http:// or https://), that protocol will be used. If not, the block
first tries HTTPS{[}{]}\{index=``HTTPS''\} and then, if that fails,
HTTP{[}{]}\{index=``HTTP''\} .

A security restriction in JavaScript limits the ability of one web site
to initiate communication with another site. There is an official
workaround for this limitation called the CORS{[}{]}\{index=``CORS''\}
protocol (Cross-Origin Resource Sharing{[}{]}\{index=``Cross-Origin
Resource Sharing''\} ), but the target site has to allow
snap.berkeley.edu explicitly, and of course most don't. To get around
this problem, you can use third-party sites (``cors
proxies{[}{]}\{index=''cors proxies''\} '') that are not limited by
JavaScript and that forward your requests.

\section{Hardware Devices}\label{hardware-devices}

Another JavaScript security restriction prevents Snap\emph{!} from
having direct access to devices{[}{]}\{index=``devices''\} connected to
your computer, such as sensors and robots{[}{]}\{index=``robots''\} .
(Mobile devices such as smartphones may also have useful devices built
in, such as accelerometers and GPS receivers.) The url block is also
used to interface Snap\emph{!} with these external capabilities.

The idea is that you run a separate program that both interfaces with
the device and provides a local HTTP server that Snap\emph{!} can use to
make requests to the device. \emph{Unlike} Snap\emph{!} \emph{itself,
these programs have access to anything on your computer, so you have to
trust the author of the software!} Our web site, snap.berkeley.edu,
provides links to drivers for several devices, including, at this
writing, the Lego NXT{[}{]}\{index=``Lego NXT''\} ,
Finch{[}{]}\{index=``Finch''\} ,
Hummingbird{[}{]}\{index=``Hummingbird''\} , and Parallax
S2{[}{]}\{index=``Parallax S2''\} robots; the
Nintendo{[}{]}\{index=``Nintendo''\} Wiimote{[}{]}\{index=``Wiimote''\}
and Leap Motion{[}{]}\{index=``Leap Motion''\} sensors, the
Arduino{[}{]}\{index=``Arduino''\} microcomputer, and Super-Awesome
Sylvia{[}{]}\{index=``Super-Awesome Sylvia''\} 's Water Color
Bot{[}{]}\{index=``Water Color Bot''\} . The same server technique can
be used for access to third party software libraries, such as the speech
synthesis package linked on our web site.

Most of these packages require some expertise to install; the links are
to source code repositories. This situation will improve with time.

\section{Date and Time}\label{date-and-time}

The current{[}{]}\{index=``current block''\} block in the Sensing
palette can be used to find out the current date or
time{[}{]}\{index=``current date or time''\} . Each call to this block
reports one component of the date{[}{]}\{index=``date''\} or
time{[}{]}\{index=``time''\} , so you will probably combine several
calls, like this:

for Americans, or like this:

for Europeans.

\bookmarksetup{startatroot}

\chapter{Continuations}\label{continuations}

Blocks are usually used within a script. The \emph{continuation} of a
block within a particular script is the part of the computation that
remains to be completed after the block does its job. A
continuation{[}{]}\{index=``continuation''\} can be represented as a
ringed script. Continuations are always part of the interpretation of
any program in any language, but usually these continuations are
implicit in the data structures of the language interpreter or compiler.
Making continuations explicit is an advanced but versatile programming
technique that allows users to create control structures such as
nonlocal exit and multithreading.

In the simplest case, the continuation of a command block may just be
the part of the script after the block. For example, in the script

the continuation of the move 100 steps block is

But some situations are more complicated. For example, what is the
continuation of move 100 steps in the following script?

That's a trick question; the move block is run four times, and it has a
different continuation each time. The first time, its continuation is

Note that there is no repeat 3 block in the actual script, but the
continuation has to represent the fact that there are three more times
through the loop to go. The fourth time, the continuation is just

What counts is not what's physically below the block in the script, but
what computational work remains to be done.

(This is a situation in which visible code may be a little misleading.
We have to put a repeat 3 block in the \emph{picture} of the
continuation, but the actual continuation is made from the evaluator's
internal bookkeeping of where it's up to in a script. So it's really the
original script plus some extra information. But the pictures here do
correctly represent what work the process still has left to do.)\\
When a block is used inside a custom block, its continuation may include
parts of more than one script. For example, if we make a custom square
block

and then use that block in a script:

then the continuation of the first use of move 100 steps is

in which part comes from inside the square block and part comes from the
call to square. Nevertheless, ordinarily when we \emph{display} a
continuation we show only the part within the current script.

The continuation of a command block, as we've seen, is a simple script
with no input slots. But the continuation of a \emph{reporter} block has
to do something with the value reported by the block, so it takes that
value as input. For example, in the script

the continuation of the 3+4 block is

Of course the name result in that picture is arbitrary; any name could
be used, or no name at all by using the empty-slot notation for input
substitution.

\section{Continuation Passing Style}\label{continuation-passing-style}

Like all{[}{]}\{index=``continuation passing style''\} programming
languages, Snap\emph{!} evaluates compositions of nested reporters from
the inside out. For example, in the expression Snap\emph{!} first adds 4
and 5, then multiplies 3 by that sum. This often means that the order in
which the operations are done is backwards from the order in which they
appear in the expression: When reading the above expression you say
``times'' before you say ``plus.'' In English, instead of saying ``three
times four plus five,'' which actually makes the order of operations
ambiguous, you could say, ``take the sum of four and five, and then take
the product of three and that sum.'' This sounds more awkward, but it
has the virtue of putting the operations in the order in which they're
actually performed.

That may seem like overkill in a simple expression, but suppose you're
trying to convey the expression

to a friend over the phone. If you say ``factorial of three times
factorial of two plus two plus five'' you might mean any of these:

Wouldn't it be better to say, ``Add two and two, take the factorial of
that, add five to that, multiply three by that, and take the factorial
of the result''? We can do a similar reordering of an expression if we
first define versions of all the reporters that take their continuation
as an explicit input. In the following picture, notice that the new
blocks are \emph{commands}, not reporters.

We can check that these blocks give the results we want:

The original expression can now be represented as

If you read this top to bottom, don't you get ``Add two and two, take
the factorial of that, add five to that, multiply three by that, and
take the factorial of the result''? Just what we wanted! This way of
working, in which every block is a command that takes a continuation as
one of its inputs, is called \emph{continuation-passing style (CPS).}
Okay, it looks horrible, but it has subtle virtues. One of them is that
each script is just one block long (with the rest of the work buried in
the continuation given to that one block), so each block doesn't have to
remember what else to do---in the vocabulary of this section, the
(implicit) continuation of each block is empty. Instead of the usual
picture of recursion, with a bunch of little
people{[}{]}\{index=``little people''\} all waiting for each other, with
CPS{[}{]}\{index=``CPS''\} what happens is that each little person hands
off the problem to the next one and goes to the beach, so there's only
one active little person at a time. In this example, we start with
Alfred, an add specialist, who computes the value 4 and then hands off
the rest of the problem to Francine, a factorial specialist. She
computes the value 24, then hands the problem off to Anne, another add
specialist, who computes 29. And so on, until finally Sam, a say
specialist, says the value 2.107757298379527×10132, which is a very
large number!

Go back to the definitions of these blocks. The ones, such as add, that
correspond to primitive reporters are simple; they just call the
reporter and then call their continuation with its result. But the
definition of factorial is more interesting. It doesn't just call our
original factorial reporter and send the result to its continuation. CPS
is used inside factorial too! It says, ``See if my input is zero. Send
the (true or false) result to if. If the result is true, then call my
continuation with the value 1. Otherwise, subtract 1 from my input. Send
the result of that to factorial, with a continuation that multiplies the
smaller number's factorial by my original input. Finally, call my
continuation with the product.'' You can use CPS to unwind even the most
complicated branched recursions.

By the way, I cheated a bit above. The if/else block should also use
CPS; it should take one true/false input and \emph{two continuations.}
It will go to one or the other continuation depending on the value of
its input. But in fact the C-shaped blocks (or E-shaped, like if/else)
are really using CPS in the first place, because they implicitly wrap
rings around the sub-scripts within their branches. See if you can make
an explicitly CPS if/else block.

\section{Call/Run w/Continuation}\label{callrun-wcontinuation}

To use explicit continuation passing style, we had to define special
versions of all the reporters, add and so on. Snap\emph{!} provides a
primitive mechanism for capturing continuations when we need to, without
using continuation passing throughout a project.

Here's the classic example. We want to write a recursive block that
takes a list of numbers as input, and reports the product of all the
numbers:

But we can improve the efficiency of this block, in the case of a list
that includes a zero; as soon as we see the zero, we know that the
entire product is zero.

But this is not as efficient as it might seem. Consider, as an example,
the list 1,2,3,0,4,5. We find the zero on the third recursive call (the
fourth call altogether), as the first item of the sublist 0,4,5. What is
the continuation of the report 0 block? It's

Even though we already know that result is zero, we're going to do three
unnecessary multiplications while unwinding the recursive calls.

We can improve upon this by capturing the
continuation{[}{]}\{index=``call w/continuation block''\} of the
top-level call to product:

The block takes as its input a one-input script, as shown in the product
example. It calls that script with \emph{the continuation of the}
call-with-continuation \emph{block itself} as its input. In this case,
that continuation is

reporting to whichever script called product. If the input list doesn't
include a zero, then nothing is ever done with that continuation, and
this version works just like the original product. But if the input list
is 1,2,3,0,4,5, then three recursive calls are made, the zero is seen,
and product-helper \emph{runs the continuation,} with an input of 0. The
continuation immediately reports that 0 to the caller of product,
\emph{without} unwinding all the recursive calls and without the
unnecessary multiplications.

I could have written product a little more simply using a Reporter ring
instead of a Command ring:

but it's customary to use a script to represent the input to
call\textbf{~}w/continuation because very often that input takes the
form

so that the continuation is saved permanently and can be called from
anywhere in the project. That's why the input slot in call
w/continuation has a Command ring rather than a Reporter ring.

First class continuations are an experimental feature in Snap\emph{!}
and there are many known limitations in it. One is that the display of
reporter continuations shows only the single block in which the call
w/continuation is an input.

\subsection{Nonlocal exit}\label{nonlocal-exit}

Many programming{[}{]}\{index=``nonlocal exit''\} languages have a break
command{[}{]}\{index=``break command''\} that can be used inside a
looping construct such as repeat to end the repetition early. Using
first class continuations, we can generalize this mechanism to allow
nonlocal exit even within a block called from inside a loop, or through
several levels of nested loops:

The upvar break has as its value a continuation{[}{]}\{index=``run
w/continuation''\} that can be called from anywhere in the program to
jump immediately to whatever comes after the catch block in its script.
Here's an example with two nested invocations of
catch{[}{]}\{index=``catch block''\} , with the upvar renamed in the
outer one:

As shown, this will say 1, then 2, then 3, then exit both nested catches
and think ``Hmm.'' If in the run block the variable break is used
instead of outer, then the script will say 1, 2, 3, and ``Hello!''
before thinking ``Hmm.''

There are corresponding catch and throw blocks for reporters. The catch
block is a reporter that takes an expression as input instead of a
C-shaped slot. But the throw block is a command; it doesn't report a
value to its own continuation, but instead reports a value (which it
takes as an additional input, in addition to the catch tag) to \emph{the
corresponding catch block}'s continuation:

Without the throw, the inner call reports 5, the + block reports 8, so
the catch block reports 8, and the × block reports 80. With the throw,
the inner call doesn't report at all, and neither does the + block. The
throw block's input of 20 becomes the value reported by the catch block,
and the × block multiplies 10 and 20.\\
\textbf{Creating a Thread System}

Snap\emph{!} can be running several scripts at once, within a single
sprite and across many sprites. If you only have one computer, how can
it do many things at once? The answer is that only one is actually
running at any moment, but Snap\emph{!} switches its attention from one
script to another frequently. At the bottom of every looping block
(repeat, repeat until, forever), there is an implicit ``yield'' command,
which remembers where the current script is up to, and switches to some
other script, each in turn. At the end of every script is an implicit
``end thread{[}{]}\{index=''thread''\} '' command (a \emph{thread} is
the technical term for the process of running a script), which switches
to another script without remembering the old one.

Since this all happens automatically, there is generally no need for the
user to think about threads. But, just to show that this, too, is not
magic, here is an implementation of a simple thread system. It uses a
global variable named tasks that initially contains an empty list. Each
use of the C-shaped thread block{[}{]}\{index=``thread block''\} adds a
continuation (the ringed script) to the list. The yield
block{[}{]}\{index=``yield block''\} uses run w/continuation to create a
continuation for a partly done thread, adds it to the task list, and
then runs the first waiting task. The end\textbf{~}thread block (which
is automatically added at the end of every thread's script by the thread
block) just runs the next waiting task.

Here is a sample script using the thread system. One thread says
numbers; the other says letters. The number thread yields after every
prime number, while the letter thread yields after every vowel. So the
sequence of speech balloons is
1,2,a,3,b,c,d,e,4,5,f,g,h,i,6,7,j,k,l,m,n,o,8,9,10,11,
p,q,r,s,t,u,12,13,v,w,x,y,z,14,15,16,17,18,\ldots30.

If we wanted this to behave exactly like Snap\emph{!}'s own threads,
we'd define new versions of repeat and so on that run yield after each
repetition.

\bookmarksetup{startatroot}

\chapter{Metaprogramming}\label{metaprogramming}

The scripts and custom blocks that make up a program can be examined or
created by the program itself.

\section{Reading a block}\label{reading-a-block}

The definition of block{[}{]}\{index=``definition of block''\} takes a
custom block (in a ring, since it's the block itself that's the input,
not the result of calling the block) as input and reports the block's
definition, i.e., its inputs and body, in the form of a ring with named
inputs corresponding to the block's input names, so that those input
names are bound in the body.

The split by blocks block{[}{]}\{index=``split by blocks block''\} takes
any expression or script as input (ringed) and reports a list
representing a \emph{syntax tree} for the script or expression, in which
the first item is a block with no inputs and the remaining items are the
input values, which may themselves be syntax trees.

Using split by blocks to select custom blocks whose definitions contain
another block gives us this debugging aid:

Note in passing the my blocks block{[}{]}\{index=``my blocks block''\} ,
which reports a list of all visible blocks, primitive and custom.
(There's also a my categories block{[}{]}\{index=``my categories
block''\} , which reports a list of the names of the palette
categories.) Also note custom? of block{[}{]}\{index=``custom? of block
block''\} , which reports True if its input is a custom block.

\section{Writing a block}\label{writing-a-block}

The inverse function to split by blocks is provided by the join
block{[}{]}\{index=``join block''\} , which when given a syntax tree as
input reports the corresponding expression or script.

Here we are taking the definition of square, modifying the repetition
count (to 6), modifying the turning angle (to 60), using join to turn
the result back into a ringed definition, and using the define
block{[}{]}\{index=``define block''\} to create a new hexagon block.

The define block has three ``input'' slots. The quotation marks are
there because the first slot is an upvar, i.e., a way for define to
provide information to its caller, rather than the other way around. In
this case, the value of block is the new block itself (the hexagon
block, in this example). The second slot is where you give the
\emph{label} for the new block. In this example, the label is ``hexagon
\_'' in which the underscore represents an input slot. So, here are a
few examples of block label{[}{]}\{index=``block label''\} s:

set pen \_ to \_

for \_ = \_ to \_ \_

ask \_ and wait

\_ of \_

Note that the underscores are separated from the block text by spaces.
Note in the case of the for block's label that the upvar (the i) and the
C-slot both count as inputs. Note also that the label is not meant to be
a unique symbol that represents only this block. For example, and both
have the label

\_ of \_. The label does not give the input slots names (that's done in
the body, coming next) or types (that's done in the set \_ of block \_
to \_ block{[}{]}\{index=``set \_ of block \_ to \_ block''\} , coming
in two paragraphs).

The third slot is for the \emph{definition}{[}{]}\{index=``definition
(of block)''\} of the new block. This is a (ringed) script whose input
names (formal parameters) will become the formal parameters of the new
block. And the script is its script.

So far we know the block's label, parameters, and script. There are
other things to specify about the block, and one purpose of the block
upvar is to allow that. In the example on the previous page, there are
four

set \_ of block \_ to \_ blocks, reproduced below for your convenience:

The category of the block can be set to any primitive or custom
category. The default is other. The type is command, reporter, or
predicate. Command is the default, so this setting is redundant, but we
want to show all the choices in the set block. The scope is either
global or sprite, with global as the default. The last input to set
slots is a list of length less than or equal to the number of
underscores in the label. Each item of the list is a type name, like the
ones in the is (5) a (number)? block. If there is only one input, you
can use just the name instead of putting it in a list. An empty or
missing list item means type Any.

It's very important that these set blocks appear in the same script as
the define that creates the block, because the block upvar is local to
that script. You can't later say, for example,

because the copy of the hexagon block in this instruction counts as
``using'' it.

The of block reporter is useful to copy attributes from one block to
another, as we copied the definition of square, modified it, and used it
to define hexagon. Some of the values this block reports are a little
unfriendly:

``1''? Yes, this block reports \emph{numbers} instead of names for
category, type, and scope. The reason is that maybe someday we'll have
translations to other languages for custom category names, as we already
do for the built-in categories, types, and scopes; if you translate a
program using this block to another language, the numeric outputs won't
change, simplifying comparisons in your code. The set block accepts
these numbers as an alternative to the names.

There are a few more attributes of a block, less commonly used.

The list input is just like the one for set slots except for default
values instead of types. Now for a block with a menu input:

Prefer a read-only menu?

We passed too quickly over how the script turned the square block into a
hexagon block:

Those replace item blocks aren't very elegant. I had to look at foo by
hand to figure out where the numbers I wanted to change are. This
situation can be improved with a little programming:

Exercise for the reader: Implement this:

Returning to the define block, there's another reason for the block
upvar: It's helpful in defining a recursive procedure using
define{[}{]}\{index=``recursive procedure using define''\} . For a
procedure to call itself, it needs a name for itself. But in the
definition input to the define block, define itself hasn't been called
yet, so the new block isn't in the palette yet. So you do this:

Yes, you put block in the define, but it gets changed into this script
in the resulting definition. You could use this script directly in a
simple case like this, but in a complicated case with a recursive call
inside a ring inside the one giving the block definition, this script
always means the innermost ring. But the upvar means the outer ring;
note how the definition of blockify automatically creates a script
variable to hold the outer environment.

It's analogous to using explicit formal parameters when you nest calls
to higher order functions.

Note: Ordinarily, when you call a function that reports a (ringed)
procedure, that procedure was created in some specific environment, and
has access to that environment's variables. This is how instance
variables (fields) work in object oriented programming (Chapter VIII).
But the procedures made by join of a syntax tree have no associated
environment, not even the one containing global variables. That doesn't
matter if the procedure will use only its own input variables, but for
access to other variables, use

\section{Macros}\label{macros}

Users of languages in the C family have learned to think of macros as
entirely about text strings, with no relation to the syntax of the
language. So you can do things like

\#define foo baz)

with the result that you can only use the foo macro after an open
parenthesis.

In the Lisp family of languages we have a different tradition, in which
macros{[}{]}\{index=``macros''\} are syntactically just like procedure
calls, except that the ``procedure'' is a macro, with different
evaluation rules from ordinary procedures. Two things make a macro
different: its input expressions are not evaluated, so a macro can
establish its own syntax (but still delimited by parentheses, in Lisp,
or still one block, in Snap\emph{!} ); and the result of a macro call is
a new expression that is evaluated \emph{as if it appeared in the
caller} of the macro, with access to the caller's variables and,
implicitly, its continuation.

Snap\emph{!} has long had the first part of this, the ability to make
inputs unevaluated. In version 8.0 we add the ability to run code in the
context of another procedure, just as we can run code in the context of
another sprite, using the same mechanism: the of block{[}{]}\{index=``of
block (sensing)''\} . In the example on the previous page, the if \_
report \_ caller \_ block runs a report block, but not in its own
context; it causes \emph{the fizzbuzz block} to report ``fizz'' or
``buzz'' as appropriate. (Yes, we know that the rules implemented here
are simplified compared to the real game.) It doesn't just report out of
the entire toplevel script; you can see that map is able to prepend
``The answer is'' to each reported value.

This macro capability isn't fully implemented. First, we shouldn't have
to use the calling script as an explicit input to the macro. In a later
release, this will be fixed; when defining a block you'll be able to say
that it's a macro, and it will automatically get its caller's context as
an invisible input. Second, there is a possibility of confusion between
the variables of the macro and the variables of its caller. (What if the
macro wanted to refer to a variable value in its caller?) The one
substantial feature of Scheme that we don't yet implement is
\emph{hygienic macros,} which make it possible to keep the two
namespaces separate.

\bookmarksetup{startatroot}

\chapter{User Interface Elements{[}{]}}\label{user-interface-elements}

In this chapter we describe in detail the various buttons, menus, and
other clickable elements of the Snap\emph{!} user interface. Here again
is the map of the Snap\emph{!} window:

\section{Tool Bar Features}\label{tool-bar-features}

Holding down the Shift key{[}{]}\{index=``tool bar features''\} while
clicking{[}{]}\{index=``shift-clicking''\} on any of the menu buttons
gives access to an extended menu with options, shown in red, that are
experimental or for use by the developers. We're not listing those extra
options here because they change frequently and you shouldn't rely on
them. But they're not secrets{[}{]}\{index=``secrets''\} .

\subsection{\texorpdfstring{The Snap\emph{!} Logo
Menu}{The Snap! Logo Menu}}\label{the-snap-logo-menu}

The Snap\emph{!} logo{[}{]}\{index=``Snap! logo menu''\} at the left end
of the tool bar is clickable. It shows a menu of options about
Snap\emph{!} itself:

The About option{[}{]}\{index=``About option''\} displays information
about Snap\emph{!} itself, including version numbers for the source
modules, the implementors, and the license{[}{]}\{index=``license''\}
(AGPL{[}{]}\{index=``AGPL''\} : you can do anything with it except
create proprietary versions, basically).

The Reference manual option{[}{]}\{index=``Reference manual option''\}
downloads a copy of the latest revision of this manual in PDF format.

The Snap! website option{[}{]}\{index=``Snap! website option''\} opens a
browser window pointing to
snap.berkeley.edu{[}{]}\{index=``snap.berkeley.edu''\} , the web site
for Snap\emph{!}.

The Download source option{[}{]}\{index=``Download source option''\}
opens a browser window displaying the Github repository of the source
files for Snap\emph{!}.{[}{]}\{index=``source files for Snap!''\} At the
bottom of the page are links to download the latest official release. Or
you can navigate around the site to find the current development
version. You can read the code to learn how Snap\emph{!} is implemented,
host a copy on your own computer (this is one way to keep working while
on an airplane), or make a modified version with customized features.
(However, access to cloud accounts is limited to the official version
hosted at Berkeley.)

\subsection{The File Menu}\label{the-file-menu}

The file icon{[}{]}\{index=``file icon menu''\} shows a menu mostly
about saving and loading projects. You may not see all these options, if
you don't have multiple sprites, scenes, custom blocks, and custom
categories.

The Notes option{[}{]}\{index=``Project notes option''\} opens a window
in which you can type notes about the project: How to use it, what it
does, whose project you modified to create it, if any, what other
sources of ideas you used, or any other information about the project.
This text is saved with the project, and is useful if you share it with
other users.

The New option{[}{]}\{index=``New option''\} starts a new, empty
project. Any project you were working on before disappears, so you are
asked to confirm that this is really what you want. (It disappears only
from the current working Snap\emph{!} window; you should save the
current project, if you want to keep it, before using New.)

Note the \^{}N at the end of the line. This indicates that you can type
control-N as a shortcut for this menu item. Alas, this is not the case
in every browser. Some Mac browsers require command-N (⌘N) instead,
while others open a new browser window instead of a new project. You'll
have to experiment. In general, the keyboard
shortcuts{[}{]}\{index=``shortcuts:keyboard''\} in Snap\emph{!} are the
standard ones you expect in other software.

The Open\ldots{} option{[}{]}\{index=``Open\ldots{} option''\} shows a
project open dialog box in which you can choose a project to open:

In this dialog, the three large buttons at the left select a source of
projects: Cloud{[}{]}\{index=``Cloud button''\} means your Snap\emph{!}
account's cloud storage. Examples{[}{]}\{index=``Examples button''\}
means a collection of sample projects we provide. Computer is for
projects saved on your own computer; when you click it, this dialog is
replaced with your computer's system dialog for opening files. The text
box to the right of those buttons is an alphabetical listing of projects
from that source; selecting a project by clicking shows its thumbnail (a
picture of the stage when it was saved) and its project notes at the
right.

The search bar{[}{]}\{index=``search bar''\} at the top can be used to
find a project by name or text in the project notes. So in this example:

I was looking for my ice cream{[}{]}\{index=``ice cream''\} projects and
typed ``crea'' in the search bar, then wondered why ``ferris'' matched.
But then when I clicked on ferris I saw this:

My search matched the word ``re\emph{crea}te'' in the project notes.

The six buttons at the bottom select an action to perform on the
selected project. In the top row, Recover looks in your cloud account
for older versions of the chosen project. \textbf{\emph{If your project
is damaged, don't keep saving broken versions! Use Recover first
thing.}} You will see a list of saved versions; choose one to open it.
Typically, you'll see the most recent version before the last save, and
the newest version saved before today. Then come buttons Share/Unshare
and Publish/Unpublish. The labelling of the buttons depends on your
project's publication status. If a project is neither shared nor
published (the ones in lightface type in the project list), it is
private and nobody can see it except you, its owner. If it is shared
(boldface in the project list), then when you open it you'll see a URL
like this one:

https://snap.berkeley.edu/snapsource/snap.html\#present:Username=bh\&ProjectName=count\%20change

but with your username and project name. (``\%20'' in the project name
represents a space, which can't be part of a URL.) Anyone who knows this
URL can see your project. Finally, if your project is published
(\textbf{\emph{bold italic}} in the list), then your project is shown on
the Snap\emph{!} web site for all the world to see. (In all of these
cases, you are the only one who can \emph{write} to (save) your project.
If another user saves it, a separate copy will be saved in that user's
account. Projects remember the history of who created the original
version and any other ``remix'' versions along the way.

In the second row, the first button, Open, loads the project into
Snap\emph{!} and closes the dialog box. The next button (if Cloud is the
source) is Delete, and if clicked it deletes the selected project.
Finally, the Cancel button closes the dialog box without opening a
project. (It does not undo any sharing, unsharing, or deletion you've
done.)

Back to the File menu, the Save menu option{[}{]}\{index=``Save
option''\} saves the project to the same source and same name that was
used when opening the project. (If you opened another user's shared
project or an example project, the project will be saved to your own
cloud account. You must be logged in to save to the cloud.)

The Save as\ldots{} menu option{[}{]}\{index=``Save as\ldots{}
option''\} opens a dialog box in which you can specify where to save the
project:

This is much like the Open dialog, except for the horizontal text box at
the top, into which you type a name for the project. You can also
publish, unpublish, share, unshare, and delete projects from here. There
is no Recover button.

The Import\ldots{} menu option{[}{]}\{index=``Import\ldots{} option''\}
is for bringing some external resource into the current project, or it
can load an entirely separate project, from your local disk. You can
import costumes (any picture format that your browser supports), sounds
(again, any format supported by your browser), and block libraries or
sprites (XML format, previously exported from Snap\emph{!} itself).
Imported costumes and sounds will belong to the currently selected
sprite; imported blocks are global (for all sprites). Using the Import
option is equivalent to dragging the file from your desktop onto the
Snap\emph{!} window.

Depending on your browser, the Export project\ldots{} option either
directly saves to your disk or{[}{]}\{index=``Export project\ldots{}
option''\} opens a new browser tab containing your complete project in
XML notation (a plain text format). You can then use the browser's Save
feature to save the project as an XML file, which should be named
\emph{something}.xml so that Snap\emph{!} will recognize it as a project
when you later drag it onto a Snap\emph{!} window. This is an
alternative to saving the project to your cloud account: keeping it on
your own computer. It is equivalent to choosing Computer from the Save
dialog described earlier.

The Export summary\ldots{} option{[}{]}\{index=``Export project\ldots{}
option''\} creates a web page, in HTML, with all of the information
about your project: its name, its project notes, a picture of what's on
its stage, definitions of global blocks, and then per-sprite
information: name, wardrobe (list of costumes), and local variables and
block definitions. The page can be converted to PDF by the browser; it's
intended to meet the documentation requirements of the Advanced
Placement Computer Science Principles{[}{]}\{index=``Computer Science
Principles''\} create task.

The Export blocks\ldots{} option{[}{]}\{index=``Export blocks\ldots{}
option''\} is used to create a block library{[}{]}\{index=``block
library''\} . It presents a list of all the global (for all sprites)
blocks in your project, and lets you select which to export. It then
opens a browser tab with those blocks in XML format, or stores directly
to your local disk, as with the Export project option. Block libraries
can be imported with the Import option or by dragging the file onto the
Snap\emph{!} window. This option is shown only if you have defined
custom blocks.

The Unused blocks\ldots{} option{[}{]}\{index=``Unused blocks\ldots{}
option''\} presents a listing of all the global custom blocks in your
project that aren't used anywhere, and offers to delete them. As with
Export blocks, you can choose a subset to delete with checkboxes. This
option is shown only if you have defined custom blocks.

The Hide blocks\ldots{} option{[}{]}\{index=``Hide blocks\ldots{}
option''\} shows \emph{all} blocks, including primitives, with
checkboxes. This option does not remove any blocks from your project,
but it does hide selected block in your palette. The purpose of the
option is to allow teachers to present students with a simplified
Snap\emph{!} with some features effectively removed. The hiddenness of
primitives is saved with each project, so students can load a shared
project and see just the desired blocks. But users can always unhide
blocks by choosing this option and unclicking all the checkboxes.
(Right-click in the background of the dialog box to get a menu from
which you can check all boxes or uncheck all boxes.)

The New category\ldots{} option{[}{]}\{index=``New category\ldots{}
option''\} allows you to add your own categories to the palette. It
opens a dialog box in which you specify a name \emph{and a color} for
the category. (A lighter version of the same color will be used for the
zebra coloring feature.)

The Remove a category\ldots{} option{[}{]}\{index=``Remove a
category\ldots{} option''\} appears only if you've created custom
categories. It opens a very small, easy-to-miss menu of category names
just under the file icon in the menu bar. If you remove a category that
has blocks in it, all those blocks are also removed.

The next group of options concern the
\emph{scenes}{[}{]}\{index=``scenes''\} feature, new in Snap\emph{!}
7.0. A scene is a complete project, with its own stage, sprites, and
code, but several can be merged into one project, using the block to
bring another scene onscreen. The Scenes\ldots{}
option{[}{]}\{index=``Scenes\ldots{} option''\} presents a menu of all
the scenes in your project, where the File menu was before you clicked
it. The New scene option{[}{]}\{index=``New scene option''\} creates a
new, empty scene, which you can rename as you like from its context
menu. Add scene\ldots{} {[}{]}\{index=``Add scene\ldots{} option''\} is
like Import\ldots{} but for scenes. (A complete project can be imported
as a scene into another project, so you have to specify that you're
importing the project \emph{as a scene} rather than replacing the
current project.)

The Libraries\ldots{} option{[}{]}\{index=``Libraries\ldots{} option''\}
presents a menu of useful, optional block libraries:

The library menu is divided into five broad categories. The first is,
broadly, utilities: blocks that might well be primitives. They might be
useful in all kinds of projects.

The second category is blocks related to media computation: ones that
help in dealing with costumes and sounds (a/k/a Jens libraries). There
is some overlap with ``big data'' libraries, for dealing with large
lists of lists.

The third category is, roughly, specific to non-media applications
(a/k/a Brian libraries). Three of them are imports from other
programming languages: words and sentences from Logo, array functions
from APL, and streams from Scheme. Most of the others are to meet the
needs of the BJC curriculum.

The fourth category is major packages provided by users.

The fifth category provides support for hardware devices such as robots,
through general interfaces, replacing specific hardware libraries in
versions before 7.0.

When you click on the one-line description of a library, you are shown
the actual blocks in the library and a longer explanation of its
purpose. You can browse the libraries to find one that will satisfy your
needs. The libraries are described in detail in Section I.H, page
\hyperref[libraries]{25}.

The Costumes\ldots{} option{[}{]}\{index=``Costumes\ldots{} option''\}
opens a browser into the costume library:

You can import a single costume by clicking it and then clicking the
Import button. Alternatively, you can import more than one costume by
double-clicking each one, and then clicking Cancel when done. Notice
that some costumes are tagged with ``svg'' in this picture; those are
vector-format costumes that are not (yet) editable within Snap\emph{!}.

If you have the stage selected in the sprite corral, rather than a
sprite, the Costumes\ldots{} option changes to a Backgrounds\ldots{}
option{[}{]}\{index=``Backgrounds\ldots{} option''\} , with different
choices in the browser:

The costume and background libraries include both
bitmap{[}{]}\{index=``bitmap''\} (go jagged if enlarged) and
vector{[}{]}\{index=``vector''\} (enlarge smoothly) images. Thanks to
Scratch 2.0/3.0 for most of these images! Some older browsers refuse to
import a vector image, but instead convert it to bitmap.

The Sounds\ldots{} option{[}{]}\{index=``Sounds\ldots{} option''\} opens
the third kind of media browser:

The Play buttons can be used to preview the sounds.

Finally, the Undelete sprites\ldots{} option{[}{]}\{index=``Undelete
sprites\ldots{} option''\} appears only if you have deleted a sprite; it
allows you to recover a sprite that was deleted by accident (perhaps
intending to delete only a costume).

\subsection{The Cloud Menu}\label{the-cloud-menu}

The cloud icon{[}{]}\{index=``cloud icon''\} shows a menu of options
relating to your Snap\emph{!} cloud account. If you are not logged in,
you see the outline icon and get this menu:

Choose Login\ldots{[}{]}\{index=``Login\ldots{} option''\} if you have a
Snap\emph{!} account and remember your password. Choose
Signup\ldots{[}{]}\{index=``Signup\ldots{} option''\} if you don't have
an account. Choose Reset Password\ldots{[}{]}\{index=``Reset
Password\ldots{} option''\} if you've forgotten your password or just
want to change it. You will then get an email, at the address you gave
when you created your account, with a new temporary password. Use that
password to log in, then you can choose your own password, as shown
below. Choose Resend Verification Email\ldots{} if you have just created
a Snap\emph{!} account but can't find the email we sent you with the
link to verify that it's really your email. (If you still can't find it,
check your spam folder. If you are using a school email address, your
school may block incoming email from outside the school.) The Open in
Community Site option{[}{]}\{index=``Open in Community Site option''\}
appears only if you have a project open; it takes you to the community
site page about that project.

If you are already logged in, you'll see the solid icon and get this
menu:

Logout{[}{]}\{index=``Logout option''\} is obvious, but has the
additional benefit of showing you who's logged in. Change
password\ldots{[}{]}\{index=``Change password\ldots{} option''\} will
ask for your old password (the temporary one if you're resetting your
password) and the new password you want, entered twice because it
doesn't echo. Open in Community Site is the same as above.

\subsection{The Settings Menu}\label{the-settings-menu}

The settings icon{[}{]}\{index=``settings icon''\} shows a menu of
Snap\emph{!} options, either for the current project or for you
permanently, depending on the option:

The Language\ldots{} option{[}{]}\{index=``Language\ldots{} option''\}
lets you see the Snap\emph{!} user interface (blocks and messages) in a
language other than English. (Note:
Translations{[}{]}\{index=``translation''\} have been provided by
Snap\emph{!} users. If your native language is missing, send us an
email!)

The Zoom blocks\ldots{} option{[}{]}\{index=``Zoom blocks\ldots{}
option''\} lets you change the size of blocks, both in the palettes and
in scripts. The standard size is 1.0 units. The main purpose of this
option is to let you take very high-resolution pictures of scripts for
use on posters. It can also be used to improve readability when
projecting onto a screen while lecturing, but bear in mind that it
doesn't make the palette or script areas any wider, so your computer's
command-option-+ feature may be more practical. Note that a zoom of 2 is
gigantic! Don't even try 10.

The Fade blocks\ldots{} option{[}{]}\{index=``Fade blocks\ldots{}
option''\} opens a dialog in which you can change the appearance of
blocks:

Mostly this is a propaganda aid to use on people who think that text
languages are somehow better or more grown up than block languages, but
some people do prefer less saturated block colors. You can use the
pulldown menu for preselected fadings, use the slider to see the result
as you change the fading amount, or type a number into the text box once
you've determined your favorite value.

The Stage size\ldots{} option{[}{]}\{index=``Stage size\ldots{}
option''\} lets you set the size of the \emph{full-size} stage in
pixels. If the stage is in half-size or double-size (presentation mode),
the stage size values don't change; they always reflect the full-size
stage.

The Microphone resolution\ldots{} option sets the buffer size used by
the microphone block in Settings. ``Resolution'' is an accurate name if
you are getting frequency domain samples; the more samples, the narrower
the range of frequencies in each sample. In the time domain, the buffer
size determines the length of time over which samples are collected.

The remaining options let you turn various features on and off. There
are three groups of checkboxes. The first is for temporary settings not
saved in your project nor in your user preferences.

The JavaScript extensions option{[}{]}\{index=``JavaScript extensions
option''\} enables the use of the JavaScript function
block{[}{]}\{index=``JavaScript function block''\} . Because malicious
projects could use JavaScript to collect private information about you,
or to delete or modify your saved projects, you must enable JavaScript
\emph{each time} you load a project that uses it.

The Extension blocks option{[}{]}\{index=``Extension blocks option''\}
adds two blocks to the palette:

These blocks provide assorted capabilities to official libraries that
were formerly implemented with the JavaScript function block. This
allows these libraries to run without requiring the JavaScript
extensions option. Details are subject to change.

Input sliders{[}{]}\{index=``Input sliders option''\} provides an
alternate way to put values in numeric input slots; if you click in such
a slot, a slider appears that you can control with the mouse:

The range of the slider will be from 25 less than the input's current
value to 25 more than the current value. If you want to make a bigger
change than that, you can slide the slider all the way to either end,
then click on the input slot again, getting a new slider with a new
center point. But you won't want to use this technique to change the
input value from 10 to 1000, and it doesn't work at all for non-integer
input ranges. This feature was implemented because software keyboard
input on phones and tablets didn't work at all in the beginning, and
still doesn't work perfectly on Android devices, so sliders provide a
workaround. It has since found another use in providing ``lively''
response to input changes; if Input sliders is checked, reopening the
settings menu will show an additional option called Execute on slider
change{[}{]}\{index=``Execute on slider change option''\} . If this
option is also checked, then changing a slider in the scripting area
automatically runs the script in which that input appears. The project
live-tree in the Examples collection shows how this can be used; it
features a fractal tree custom block with several inputs, and you can
see how each input affects the picture by moving a slider.

Turbo mode{[}{]}\{index=``Turbo mode option''\} makes many projects run
much faster, at the cost of not keeping the stage display up to date.
(Snap\emph{!} ordinarily spends most of its time drawing sprites and
updating variable watchers, rather than actually carrying out the
instructions in your scripts.) So turbo mode isn't a good idea for a
project with glide block{[}{]}\{index=``glide block''\} s or one in
which the user interacts with animated characters, but it's great for
drawing a complicated fractal, or computing the first million digits of
𝜋, so that you don't need to see anything until the final result. While
in turbo mode, the button that normally shows a green flag instead shows
a green lightning bolt. (But when ⚑ clicked hat blocks still activate
when the button is clicked.)

Visible stepping{[}{]}\{index=``visible stepping option''\} enables the
slowed-down script evaluation described in Chapter I. Checking this
option is equivalent to clicking the footprint button above the
scripting area. You don't want this on except when you're actively
debugging, because even the fastest setting of the slider is still
slowed a lot.

Log pen vectors\phantomsection\label{logpenvectors}{} tells Snap\emph{!}
to remember lines drawn by sprites as exact vectors, rather than
remember only the pixels that the drawing leaves on the stage. This
remembered vector picture can be used in two ways: First, right-clicking
on a pen trails block gives an option to relabel it into a pen vectors
block which, when run, reports the logged lines as a vector (svg)
costume. Second, right-clicking on the stage when there are logged
vectors shows an extra option, svg\ldots, that exports a picture of the
stage in vector format. Only lines are logged, not color regions made
with the fill block.

The next group of four are user preference options, preserved when you
load a new project. Long form input dialog{[}{]}\{index=``Long form
input dialog option''\} , if checked, means that whenever a custom block
input name is created or edited, you immediately see the version of the
input name dialog that includes the type options, default value setting,
etc., instead of the short form with just the name and the choice
between input name and title text. The default (unchecked) setting is
definitely best for beginners, but more experienced Snap\emph{!}
programmers may find it more convenient always to see the long form.

Plain prototype labels{[}{]}\{index=``Plain prototype labels option''\}
eliminates the plus signs between words in the Block Editor prototype
block. This makes it harder to add an input to a custom block; you have
to hover the mouse where the plus sign would have been, until a single
plus sign appears temporarily for you to click on. It's intended for
people making pictures of scripts in the block editor for use in
documentation, such as this manual. You probably won't need it
otherwise.

Clicking sound{[}{]}\{index=``Clicking sound option''\} causes a really
annoying sound effect whenever one block snaps next to another in a
script. Certain very young children, and our colleague Dan Garcia, like
this, but if you are such a child you should bear in mind that driving
your parents or teachers crazy will result in you not being allowed to
use Snap\emph{!}. It might, however, be useful for visually impaired
users.

Flat design{[}{]}\{index=``Flat design option''\} changes the ``skin''
of the Snap\emph{!} window to a really hideous design with white and
pale-grey background, rectangular rather than rounded buttons, and
monochrome blocks (rather than the shaded, somewhat 3D-looking normal
blocks). The monochrome blocks are the reason for the ``flat'' in the
name of this option. The only thing to be said for this option is that,
because of the white background, it may blend in better with the rest of
a web page when a Snap\emph{!} project is run in a frame in a larger
page. (I confess I used it to make the picture of blocks faded all the
way to just text two pages ago, though.)

The final group of settings change the way Snap\emph{!} interprets your
program; they are saved with the project, so anyone who runs your
project will experience the same behavior. Thread safe
scripts{[}{]}\{index=``Thread safe scripts option''\} changes the way
Snap\emph{!} responds when an event (clicking the green flag, say)
starts a script, and then, while the script is still running, the same
event happens again. Ordinarily, the running process stops where it is,
ignoring the remaining commands in the script, and the entire script
starts again from the top. This behavior is inherited from Scratch, and
some converted Scratch projects depend on it; that's why it's the
default. It's also sometimes the right thing, especially in projects
that play music in response to mouse clicks or keystrokes. If a note is
still playing but you ask for another one, you want the new one to start
right then, not later after the old process finishes. But if your script
makes several changes to a database and is interrupted in the middle,
the result may be that the database is inconsistent. When you select
Thread safe scripts, the same event happening again in the middle of
running a script is simply ignored. (This is arguably still not the
right thing; the event should be remembered and the script run again as
soon as it finishes. We'll probably get around to adding that choice
eventually.) Keyboard events (when \_\_ key pressed) are always
thread-safe.

Flat line ends{[}{]}\{index=``flat line ends option''\} affects the
drawing of thick lines (large pen width). Usually the ends are rounded,
which looks best when turning corners. With this option selected, the
ends are flat. It's useful for drawing a brick wall or a filled
rectangle.

Codification support{[}{]}\{index=``codification support option''\}
enables a feature that can translate a Snap\emph{!} project to a
text-based{[}{]}\{index=``text-based language''\} (rather than
block-based) programming language. The feature doesn't know about any
particular other language; instead, you can provide a translation for
each primitive block using these special block{[}{]}\{index=``map to
code block''\} s:

Using these primitive blocks, you can build a block library to translate
into any programming language. Watch for such libraries to be added to
our library collection (or contribute one). To see some examples, open
the project ``Codification'' in the Examples project list. Edit the
blocks map to Smalltalk, map to JavaScript, etc., to see examples of how
to provide translations for blocks.

The Single palette option{[}{]}\{index=``Single palette option''\} puts
all blocks, regardless of category, into a single palette. It's intended
mainly for use by curriculum developers building \emph{Parsons
problems}{[}{]}\{index=``Parsons problems''\} \emph{:} projects in which
only a small set of blocks are provided, and the task is to arrange
those blocks to achieve a set goal. In that application, this option is
combined with the hiding of almost all primitive blocks. (See page
\hyperref[context-menus-for-palette-blocks]{119}.) When Single palette
is turned on, two additional options (initially on) appear in the
settings menu; the Show categories option{[}{]}\{index=``Show categories
option''\} controls the appearance of the palette category names such as
and , while the Show buttons option{[}{]}\{index=``Show buttons
option''\} controls the appearance of the and buttons in the palette.

The HSL pen color model option{[}{]}\{index=``HSL pen color model
option''\} changes the set pen, change pen, and pen blocks to provide
menu options hue, saturation, and lightness{[}{]}\{index=``lightness
option''\} instead of hue, saturation, and brightness (a/k/a value).
Note: the name ``saturation'' means something different in HSL from in
HSV! See Appendix A for all the information you need about
colors.{[}{]}\{index=``pen block''\}

The Disable click-to-run option tells Snap\emph{!} to ignore user mouse
clicks on blocks and scripts if it would ordinarily run the block or
script. (Right-clicking and dragging still work, and so does clicking in
an input slot to edit it.) This is another Parsons problem feature; the
idea is that there will be buttons displayed that run code only in
teacher-approved ways. But kids can uncheck the checkbox.
☺︎{[}{]}\{index=``Disable click-to-run option''\}

\subsection{Visible Stepping Controls}\label{visible-stepping-controls}

After the menu buttons you'll see the project name. After that comes the
footprint button{[}{]}\{index=``footprint button''\} used to turn on
visible stepping{[}{]}\{index=``visible stepping''\} and, when it's on,
the slider to control the speed of stepping.

\subsection{Stage Resizing Buttons}\label{stage-resizing-buttons}

Still in the tool bar, but{[}{]}\{index=``Stage resizing buttons''\}
above the left edge of the stage, are two buttons that change the size
of the stage. The first is the shrink/grow
button{[}{]}\{index=``shrink/grow button''\} . Normally it looks like
this: Clicking the button displays the stage at half-normal size
horizontally and vertically (so it takes up ¼ of its usual area). When
the stage is half size the button looks like this: and clicking it
returns the stage to normal size. The main reason you'd want a half size
stage is during the development process, when you're assembling scripts
with wide input expressions and the normal scripting area isn't wide
enough to show the complete script. You'd typically then switch back to
normal size to try out the project. The next presentation mode
button{[}{]}\{index=``presentation mode button''\} normally looks like
this: Clicking the button makes the stage double size in both dimensions
and eliminates most of the other user interface elements (the palette,
the scripting area, the sprite corral, and most of the tool bar). When
you open a shared project using a link someone has sent you, the project
starts in presentation mode. While in presentation mode, the button
looks like this: Clicking it returns to normal (project development)
mode.

\subsection{Project Control Buttons}\label{project-control-buttons}

Above{[}{]}\{index=``project control buttons''\} the right edge of the
stage are three buttons that control the running of the project.

Technically, the green flag{[}{]}\{index=``green flag button''\} is no
more a project control than anything else that can trigger a hat block:
typing on the keyboard or clicking on a sprite. But it's a convention
that clicking the flag should start the action of the project from the
beginning. It's only a convention; some projects have no flag-controlled
scripts at all, but respond to keyboard controls instead. Clicking the
green flag also deletes temporary clones.

Whenever any script is running (not necessarily in the current sprite),
the green flag is lit: .

Shift-clicking the button enters Turbo mode, and the button then looks
like a lightning bolt: . Shift-clicking again turns Turbo mode off.

Scripts can simulate clicking the green flag by broadcasting the special
message .

The pause button{[}{]}\{index=``pause button''\} suspends running all
scripts. If clicked while scripts are running, the button changes shape
to become a play button: Clicking it while in this form resumes the
suspended scripts. There is also a pause all block{[}{]}\{index=``pause
all block''\} in the Control palette that can be inserted in a script to
suspend all scripts; this provides the essence of a
breakpoint{[}{]}\{index=``breakpoint''\}
debugging{[}{]}\{index=``debugging''\} capability. The use of the pause
button is slightly different in visible stepping mode, described in
Chapter I.

The stop button{[}{]}\{index=``stop button''\} stops all scripts, like
the stop all block{[}{]}\{index=``stop all block''\} . It does
\emph{not} prevent a script from starting again in response to a click
or keystroke; the user interface is always active. There is one
exception: generic when blocks will not fire after a stop until some
non-generic event starts a script. The stop button also deletes all
temporary clones.

\section{The Palette Area}\label{the-palette-area}

At the top of the palette area{[}{]}\{index=``palette area''\} are the
eight buttons that select which palette (which block category) is shown:
Motion, Looks, Sound, Pen, Control, Sensing, Operators, and Variables
(which also includes the List and Other blocks). There are no menus
behind these buttons.

\subsection{Buttons in the Palette}\label{buttons-in-the-palette}

Under the eight palette selector buttons, at the top of the actual
palette, are two semi-transparent buttons. The first is the
\emph{search} button{[}{]}\{index=``search button''\} , which is
equivalent to typing control-F: It replaces the palette with a search
bar into which you can type part of the title text of the block you're
trying to find. To leave this search mode, click one of the eight
palette selectors, or type the Escape key.

The other button is equivalent to the ``Make a block''
button{[}{]}\{index=``Make a block button''\} , except that the dialog
window that it opens has the current palette (color) preselected.

\subsection{Context Menus for Palette
Blocks}\label{context-menus-for-palette-blocks}

Most elements{[}{]}\{index=``context menus for palette blocks''\} of the
Snap\emph{!} display can be control-clicked/right-clicked to show a
\emph{context menu}{[}{]}\{index=``context menu''\} \emph{,} with items
relevant to that element. If you control-click/right-click a
\emph{primitive} block in the palette, you see this menu:

The help\ldots{} option{[}{]}\{index=``help\ldots{} option''\} displays
a box with documentation about the block. Here's an example:

If you control-click/right-click a \emph{custom} (user-defined) block in
the palette, you see this menu:

The help\ldots{} option for a custom block{[}{]}\{index=``help\ldots{}
option for custom block''\} displays the comment, if any, attached to
the custom block's hat block in the Block Editor. Here is an example of
a block with a comment and its help display:

If the help text includes a URL, it is clickable and will open the page
in a new tab.

The delete block definition\ldots{} option{[}{]}\{index=``delete block
definition\ldots{} option''\} asks for confirmation, then deletes the
custom block and removes it from any scripts in which it appears. (The
result of this removal may not leave a sensible script; it's best to
find and correct such scripts \emph{before} deleting a block.) Note that
there is no option to \emph{hide} a custom block; this can be done in
the Block Editor by right-clicking on the hat block.

The duplicate block definition\ldots{} option{[}{]}\{index=``duplicate
block definition\ldots{} option''\} makes a \emph{copy} of the block and
opens that copy in the Block Editor. Since you can't have two custom
blocks with the same title text and input types, the copy is created
with ``(2)'' (or a higher number if necessary) at the end of the block
prototype.

The export block definition\ldots{} option{[}{]}\{index=``export block
definition\ldots{} option''\} writes a file in your browser's downloads
directory containing the definition of this block and any other custom
blocks that this block invokes, directly or indirectly. So the resulting
file can be loaded later without the risk of red Undefined! blocks
because of missing dependencies.{[}{]}\{index=``Undefined! blocks''\}

The edit\ldots{} option{[}{]}\{index=``edit\ldots{} option''\} opens a
Block Editor with the definition of the custom block.

\subsection{Context Menu for the Palette
Background}\label{context-menu-for-the-palette-background}

Right-click/control-click on{[}{]}\{index=``context menu for the palette
background''\} the grey \emph{background} of the palette area shows this
menu:

The find blocks\ldots{} option{[}{]}\{index=``find blocks\ldots{}
option''\} does the same thing as the magnifying-glass button. The hide
blocks\ldots{} option{[}{]}\{index=``hide blocks option''\} opens a
dialog box in which you can choose which blocks (custom as well as
primitive) should be hidden. (Within that dialog box, the context menu
of the background allows you to check or uncheck all the boxes at once.)

The make a category\ldots{} option{[}{]}\{index=``show primitives
option''\} , which is intended mainly for authors of snap extensions,
lets you add custom \emph{categories} to the palette. It opens a small
dialog window in which you specify a name \emph{and a color} for the new
category:

Pick a dark color, because it will be lightened for zebra coloring when
users nest blocks of the same category. Custom categories are shown
below the built-in categories in the category selector:

This example comes from Eckart{[}{]}\{index=``Modrow, Eckart''\}
Modrow's SciSnap\emph{!}{[}{]}\{index=``SciSnap!''\} library. Note that
the custom category list has its own scroll bar, which appears if you
have more than six custom categories. Note also that the buttons to
select a custom category occupy the full width of the palette area,
unlike the built-in categories, which occupy only half of the width.
Custom categories are listed in alphabetical order; this is why
Prof.~Modrow chose to start each category name with a number, so that he
could control their order.

If there are no blocks visible in a category, the category name is
dimmed in the category selector:

Here we see that category foo has blocks in it, but categories bar and
garply are empty. The built-in categories are also subject to dimming,
if all of the blocks of a category are hidden.

**\\
Palette Resizing**

At the right end of the palette area, just to the left of the scripting
area, is a resizing handle that can be dragged rightward to increase the
width of the palette area. This is useful if you write custom blocks
with very long names. You can't reduce the width of the palette below
its standard value.

\section{The Scripting Area}\label{the-scripting-area}

The scripting area{[}{]}\{index=``scripting area''\} is the middle
vertical region of the Snap\emph{!} window, containing scripts and also
some controls for the appearance and behavior of a sprite. There is
always a \emph{current sprite}{[}{]}\{index=``current sprite''\}
\emph{,} whose scripts are shown in the scripting area. A dark grey
rounded rectangle in the sprite corral shows which sprite (or the stage)
is current. Note that it's only the visible \emph{display} of the
scripting area that is ``current'' for a sprite; all scripts of all
sprites may be running at the same time. Clicking on a sprite
thumbnail{[}{]}\{index=``thumbnail''\} in the sprite corral makes it
current. The stage itself can be selected as current, in which case the
appearance is different, with some primitives not shown.

\subsection{Sprite Appearance and Behavior
Controls}\label{sprite-appearance-and-behavior-controls}

At the top of the scripting area{[}{]}\{index=``sprite appearance and
behavior controls''\} are a picture of the sprite and some controls for
it:

Note that the sprite picture reflects its rotation, if any. There are
three things that can be controlled here:

1. The three circular buttons{[}{]}\{index=``rotation buttons''\} in a
column at the left control the sprite's \emph{rotation} behavior. Sprite
costumes are designed to be right-side-up when the sprite is facing
toward the right (direction = 90). If the topmost button is lit, the
default as shown in the picture above, then the sprite's costume rotates
as the sprite changes direction. If the middle button is selected, then
the costume is reversed left-right when the sprite's direction is
roughly leftward (direction between 180 and 359, or equivalently,
between -180 and -1). If the bottom button is selected, the costume's
orientation does not change regardless of the sprite's direction.

2. The sprite's \emph{name} can be changed in the text
box{[}{]}\{index=``name box''\} that, in this picture, says ``Sprite.''

3. Finally, if the draggable checkbox{[}{]}\{index=``draggable
checkbox''\} is checked, then the user can move the sprite on the stage
by clicking and dragging it. The common use of this feature is in game
projects, in which some sprites are meant to be under the player's
control but others are not.

\subsection{Scripting Area Tabs}\label{scripting-area-tabs}

Just below the sprite controls are three \emph{tabs} that determine what
is shown in the scripting area:

\subsection{Scripts and Blocks Within
Scripts}\label{scripts-and-blocks-within-scripts}

Most of what's described in this section also applies to blocks and
scripts in a Block Editor.

Clicking on a script (which includes a single unattached block) runs it.
If the script starts with a hat block, clicking on the
script{[}{]}\{index=``clicking on a script''\} runs it even if the event
in the hat block doesn't happen. (This is a useful debugging technique
when you have a dozen sprites and they each have five scripts with
green-flag hat blocks, and you want to know what a single one of those
scripts does.) The script will have a green
``halo''{[}{]}\{index=``green halo''\} around it while it's running. If
the script is shared with clones, then while it has the green halo it
will also have a count of how many instances of the script are running.
Clicking a script with such a halo{[}{]}\{index=``halo''\} \emph{stops}
the script. (If the script includes a warp block{[}{]}\{index=``warp
block''\} , which might be inside a custom block used in the script,
then Snap\emph{!} may not respond immediately to clicks.)

If a script is shown with a \emph{red} halo{[}{]}\{index=``red halo''\}
, that means that an error was caught in that script, such as using a
list where a number was needed, or vice versa. Clicking the script will
turn off the halo.

If any blocks have been dragged into the scripting area, then in its top
right corner you'll see an \emph{undo} and/or \emph{redo}
button{[}{]}\{index=``undo button''\} that can be used to undo or redo
block and script drops. When you undo a drop into an input slot,
whatever used to be in the slot is restored. The redo button appears
once you've used undo.

The third button starts keyboard editing{[}{]}\{index=``keyboard editing
button''\} mode (Section D, page \hyperref[keyboard-editing]{130}).

Control-click/right-clicking a primitive block within a
script{[}{]}\{index=``primitive block within a script''\} shows a menu
like this one:

command block: reporter block:

The help\ldots{} option{[}{]}\{index=``help\ldots{} option''\} shows the
help screen for the block, just as in the palette. The other options
appear only when a block is right-clicked/control-clicked in the
scripting area.

Not every primitive block has a relabel\ldots{}
option{[}{]}\{index=``relabel\ldots{} option''\} . When present, it
allows the block to be replaced by another, similar block, keeping the
input expressions in place. For example, here's what happens when you
choose relabel\ldots{} for an arithmetic operator:

Note that the inputs to the existing -- block are displayed in the menu
of alternatives also. Click a block in the menu to choose it, or click
outside the menu to keep the original block. Note that the last three
choices are not available in the palette; you must use the relabel
feature to access them.

Not every reporter has a compile option{[}{]}\{index=``compile menu
option''\} ; it exists only for the higher order functions. When
selected, a lightning bolt{[}{]}\{index=``lightning bolt symbol''\}
appears before the block name: and Snap\emph{!} tries to compile the
function inside the ring to JavaScript, so it runs at primitive speed.
This works only for simple functions (but the higher order function
still works even if the compilation doesn't). The function to be
compiled must be quick, because it will be uninterruptable; in
particular, if it's an infinite loop, you may have to quit your browser
to recover. Therefore, \textbf{save your project before} you experiment
with the compilation feature. The right-click menu for a compiled higher
order function will have an uncompile option. This is an experimental
feature.

The duplicate option{[}{]}\{index=``duplicate option''\} for a command
block makes a copy of the \emph{entire script} starting from the
selected block. For a reporter, it copies only that reporter and its
inputs. The copy is attached to the mouse, and you can drag it to
another script (or even to another Block Editor window), even though you
are no longer holding down the mouse button. Click the mouse to drop the
script copy.

The block picture{[}{]}\{index=``block picture option''\} underneath the
word duplicate for a command block is another duplication option, but it
duplicates only the selected block, not everything under it in the
script. Note that if the selected block is a C-shaped control block, the
script inside its C-shaped slot is included. If the block is at the end
of its script, this option does not appear. (Use duplicate instead.)

The extract option{[}{]}\{index=``extract option''\} removes the
selected block from the script and leaves you holding it with the mouse.
In other words, it's like the block picture option, but it doesn't leave
a copy of the block in the original script. If the block is at the end
of its script, this option does not appear. (Just grab the block with
the mouse.) A shorthand for this operation is to
\emph{shift-click}{[}{]}\{index=``shift-click on block''\} and drag out
the block.

The delete option{[}{]}\{index=``delete option''\} deletes the selected
block from the script.

The add comment option{[}{]}\{index=``add comment option''\} creates a
comment, like the same option in the background of the scripting area,
but attaches it to the block you clicked.

The script pic\ldots{[}{]}\{index=``script pic\ldots{} option''\} option
saves a picture of the entire script{[}{]}\{index=``picture of
script''\} , not just from the selected block to the end, into your
download folder; or, in some browsers, opens a new browser tab
containing the picture. In the latter case, you can use the browser's
Save feature to put the picture in a file. This is a super useful
feature if you happen to be writing a Snap\emph{!}
manual{[}{]}\{index=``Snap! manual''\} ! (If you have a Retina display,
consider turning off Retina support before making script pictures; if
not, they end up huge.) For reporters not inside a script, there is an
additional result pic\ldots{} option{[}{]}\{index=``result pic\ldots{}
option''\} that calls the reporter and includes a speech
balloon{[}{]}\{index=``picture with speech balloon''\} with the result
in the picture. Note: The downloaded file is a ``smart
picture{[}{]}\{index=''smart picture''\} '': It also contains the code
of the script, as if you'd exported the project. If you later drag the
file into the costumes tab, it will be loaded as a costume. But if you
drag it into the \emph{scripts} tab, it will be loaded as a script,
which you can drop wherever you want it in the scripting area.

If the script does \emph{not} start with a hat block, or you clicked on
a reporter, then there's one more option: ringify{[}{]}\{index=``ringify
option''\} (and, if there is already a grey ring around the block or
script, unringify){[}{]}\{index=``unringify option''\} . Ringify
surrounds the block (reporter) or the entire script (command) with a
grey ring, meaning that the block(s) inside the ring are themselves
data, as an input to a higher order procedure, rather than something to
be evaluated within the script. See Chapter VI, Procedures as Data.

Clicking a \emph{custom} block in a script{[}{]}\{index=``custom block
in a script''\} gives a similar but different menu:

The relabel\ldots{} option{[}{]}\{index=``relabel\ldots{} option''\} for
custom blocks shows a menu of other same-shape custom blocks with the
same inputs. At present you can't relabel a custom block to a primitive
block or vice versa. The two options at the bottom, for custom blocks
only, are the same as in the palette. The other options are the same as
for primitive commands.

If a reporter block is in the scripting area, possibly with inputs
included, but not itself serving as input to another block, then the
menu is a little different again:

What's new here is the result pic\ldots{} option{[}{]}\{index=``result
pic\ldots{} option''\} . It's like script pic\ldots{} but it includes in
the picture a speech balloon with the result of calling the block.

Broadcast and broadcast and wait block{[}{]}\{index=``broadcast and wait
block''\} s in the scripting area have an additional option:
receivers\ldots{[}{]}\{index=``receivers\ldots{} option''\} . When
clicked, it causes a momentary (be looking for it when you click!) halo
around the picture in the sprite corral of those sprites that have a
when I receive hat block for the same message. Similarly, when I receive
blocks have a senders\ldots{} option{[}{]}\{index=``senders\ldots{}
option''\} that light up the sprite corral icons of sprites that
broadcast the same message.

\textbf{Scripting Area Background Context Menu}

Control-click/right-click on{[}{]}\{index=``scripting area background
context menu''\} the grey striped background of the scripting area gives
this menu:

The undrop option{[}{]}\{index=``undrop option''\} is a sort of ``undo''
feature for the common case of dropping a block somewhere other than
where you meant it to go. It remembers all the dragging and dropping
you've done in this sprite's scripting area (that is, other sprites have
their own separate drop memory), and undoes the most recent, returning
the block to its former position, and restoring the previous value in
the relevant input slot, if any. Once you've undropped something, the
redrop option{[}{]}\{index=``redrop option''\} appears, and allows you
to repeat the operation you just undid. These menu options are
equivalent to the and buttons described earlier.

The clean up option{[}{]}\{index=``clean up option''\} rearranges the
position of scripts so that they are in a single column, with the same
left margin, and with uniform spacing between scripts. This is a good
idea if you can't read your own project!

The add comment option{[}{]}\{index=``add comment option''\} puts a
comment box,{[}{]}\{index=``comment box''\} like the picture to the
right, in the scripting area. It's attached to the mouse, as with
duplicating scripts, so you position the mouse where you want the
comment and click to release it. You can then edit the text in the
comment as desired.

You can drag the bottom right corner of the comment box to resize it.
Clicking the arrowhead at the top left changes the box to a single-line
compact form, , so that you can have a number of collapsed comments in
the scripting area and just expand one of them when you want to read it
in full.

If you drag a comment over a block in a script, the comment will be
attached to the block with a yellow line:

Comments have their own context menu, with obvious meanings:

Back to the options in the menu for the background of the scripting area
(picture on the previous page):

The scripts pic\ldots{} option saves, or{[}{]}\{index=``scripts
pic\ldots{} option''\} opens a new browser tab with, a picture of
\emph{all} scripts in the scripting area, just as they appear, but
without the grey striped background. Note that ``all scripts in the
scripting area'' means just the top-level scripts of the current sprite,
not other sprites' scripts or custom block definitions. This is also a
``smart picture''; if you drag it into the scripting area, it will
\emph{create a new sprite} with those scripts in its scripting area.

Finally, the make a block\ldots{} option{[}{]}\{index=``make a
block\ldots{} option''\} does the same thing as the ``Make a block''
button in the palettes. It's a shortcut{[}{]}\{index=``shortcut''\} so
that you don't have to keep scrolling down the palette if you make a lot
of blocks.

\subsection{Controls in the Costumes
Tab}\label{controls-in-the-costumes-tab}

If you click{[}{]}\{index=``Costumes tab''\} on the word ``Costumes''
under the sprite controls, you'll see something like this:

The Turtle costume{[}{]}\{index=``turtle costume''\} is always present
in every sprite; it is costume number 0. Other costumes can be painted
within Snap\emph{!} or imported from files or other browser tabs if your
browser supports that. Clicking on a costume selects it; that is, the
sprite will look like the selected costume. Clicking on the paint brush
icon{[}{]}\{index=``paint brush icon''\}\\
opens the \emph{Paint Editor}{[}{]}\{index=``Paint Editor''\} \emph{,}
in which you can create a new costume. Clicking on the camera
icon{[}{]}\{index=``camera icon''\} opens a window in which you see what
your computer's camera is seeing, and you can take a picture (which will
be the full size of the stage unless you shrink it in the Paint Editor).
This works only if you give Snap\emph{!} permission to use the camera,
and maybe only if you opened Snap\emph{!} in secure
(HTTPS{[}{]}\{index=``HTTPS''\} ) mode, and then only if your browser
loves you.

\emph{Brian's bedroom when he's staying at Paul's house.}

Control-clicking/right-clicking on the turtle picture gives this menu:

In this menu, you choose the turtle's \emph{rotation
point}{[}{]}\{index=``turtle's rotation point''\} \emph{,} which is also
the point from which the turtle draws lines. The two pictures below show
what the stage looks like after drawing a square in each mode;
tip{[}{]}\{index=``tip option''\} (otherwise known as ``Jens mode'') is
on the left in the pictures below, middle{[}{]}\{index=``middle
option''\} (``Brian mode'') on the right:

As you see, ``tip'' means the front tip of the arrowhead; ``middle'' is
not the middle of the shaded region, but actually the middle of the four
vertices, the concave one. (If the shape were a simple isosceles
triangle instead of a fancier arrowhead, it would mean the midpoint of
the back edge.) The advantage of tip mode is that the sprite is less
likely to obscure the drawing. The advantage of middle mode is that the
rotation point of a sprite is rarely at a tip, and students are perhaps
less likely to be confused about just what will happen if you ask the
turtle to turn 90 degrees from the position shown. (It's also the
traditional rotation point of the Logo turtle, which originated this
style of drawing.)

Costumes other than the turtle have a different context menu:

The edit option{[}{]}\{index=``edit option''\} opens the Paint Editor on
this costume. The rename option{[}{]}\{index=``rename option''\} opens a
dialog box in which you can rename the costume. (A costume's initial
name comes from the file from which it was imported, if any, or is
something like costume5.) Duplicate{[}{]}\{index=``duplicate option''\}
makes a copy of the costume, in the same sprite. (Presumably you'd do
that because you intend to edit one of the copies.)
Delete{[}{]}\{index=``delete option''\} is obvious. The get blocks
option{[}{]}\{index=``get blocks option''\} appears only for a smart
costume, and brings its script to the scripting area. The export
option{[}{]}\{index=``export option''\} saves the costume as a file on
your computer, in your usual downloads folder.

You can drag costumes up and down in the Costumes tab in order to
renumber them, so that next costume will behave as you prefer.

If you drag a \emph{smart picture} of a script into the Costumes tab,
its icon will display the text ``\textless/\textgreater{}'' in the
corner to remind you that it includes code:

Its right-click menu will have an extra get blocks
option{[}{]}\{index=``get blocks option''\} that switches to the Scripts
tab with the script ready to be dropped there.

\subsection{The Paint Editor}\label{the-paint-editor}

Here is a picture of a Paint Editor window{[}{]}\{index=``Paint Editor
window''\} :

If you've used any painting program, most of this will be familiar to
you. Currently, costumes you import can be edited only if they are in a
bitmap format (png, jpeg, gif, etc.). There is a vector editor, but it
works only for creating a costume, not editing an imported vector (svg)
picture. Unlike the case of the Block Editor, only one Paint Editor
window can be open at a time.

The ten square buttons in two rows of five near the top left of the
window are the \emph{tools.} The top row, from left to right, are the
paintbrush tool{[}{]}\{index=``paintbrush tool''\} , the outlined
rectangle tool{[}{]}\{index=``rectangle tool''\} , the outlined ellipse
tool{[}{]}\{index=``ellipse tool''\} , the eraser
tool{[}{]}\{index=``eraser tool''\} , and the rotation point
tool{[}{]}\{index=``rotation point tool''\} . The bottom row tools are
the line drawing tool{[}{]}\{index=``line drawing tool''\} , the solid
rectangle tool{[}{]}\{index=``solid rectangle tool''\} , the solid
ellipse tool{[}{]}\{index=``solid ellipse tool''\} , the floodfill
tool,{[}{]}\{index=``floodfill tool,''\} and the eyedropper
tool{[}{]}\{index=``eyedropper tool''\} . Below the tools is a row of
four buttons that immediately change the picture. The first two change
its overall size; the next two flip the picture around horizontally or
vertically. Below these are a color palette{[}{]}\{index=``color
palette''\} , a greyscale tape, and larger buttons for black, white, and
transparent paint. Below these is a solid bar displaying the currently
selected color. Below that is a picture of a line showing the brush
width for painting and drawing, and below that, you can set the width
either with a slider or by typing a number (in pixels) into the text
box. Finally, the checkbox constrains the line tool to draw horizontally
or vertically, the rectangle tools to draw squares, and the ellipse
tools to draw circles. You can get the same effect temporarily by
holding down the shift key, which makes a check appear in the box as
long as you hold it down. (But the Caps Lock key doesn't affect it.)

You can correct errors with the undo button{[}{]}\{index=``undo
button''\} , which removes the last thing you drew, or the clear
button{[}{]}\{index=``clear button''\} , which erases the entire
picture. (Note, it does \emph{not} revert to what the costume looked
like before you started editing it! If that's what you want, click the
Cancel button{[}{]}\{index=``Cancel button''\} at the bottom of the
editor.) When you're finished editing, to keep your changes, click OK.

Note that the ellipse tool{[}{]}\{index=``ellipse tool''\} s work more
intuitively than ones in other software you may have used. Instead of
dragging between opposite corners of the rectangle circumscribing the
ellipse you want, so that the endpoints of your dragging have no obvious
connection to the actual shape, in Snap\emph{!} you start at the center
of the ellipse you want and drag out to the edge. When you let go of the
button, the mouse cursor will be on the curve. If you drag out from the
center at 45 degrees to the axes, the resulting curve will be a circle;
if you drag more horizontally or vertically, the ellipse will be more
eccentric. (Of course if you want an exact circle you can hold down the
shift key or check the checkbox.) The rectangle tools, though, work the
way you expect: You start at one corner of the desired rectangle and
drag to the opposite corner.

Using the eyedropper tool{[}{]}\{index=``eyedropper tool''\} , you can
click anywhere in the Snap\emph{!} window, even outside the Paint
Editor, and the tool will select the color at the mouse cursor for use
in the Paint Editor. You can only do this once, because the Paint Editor
automatically selects the paintbrush when you choose a color. (Of course
you can click on the eyedropper tool button again.)

The only other non-obvious tool is the rotation point
tool{[}{]}\{index=``rotation point tool''\} . It shows in the Paint
Editor where the sprite's current rotation center is (the point around
which it turns when you use a turn block); if you click or drag in the
picture, the rotation point will move where you click. (You'd want to do
this, for example, if you want a character to be able to wave its arm,
so you use two sprites connected together. You want the rotation point
of the arm sprite to be at the end where it joins the body, so it
remains attached to the shoulder while waving.)

The vector editor{[}{]}\{index=``vector editor''\} 's controls are much
like those in the bitmap editor. One point of difference is that the
bitmap editor has two buttons for solid and outline rectangles, and
similarly for ellipses, but in the vector editor there is always an edge
color{[}{]}\{index=``edge color''\} and a fill color{[}{]}\{index=``fill
color''\} , even if the latter is ``transparent
paint{[}{]}\{index=''transparent paint''\} ,'' and so only one button
per shape is needed. Since each shape that you draw is a separate layer
(like sprites on the stage), there are controls to move the selected
shape up (frontward) or down (rearward) relative to other shapes. There
is a selection tool to drag out a rectangular area and select all the
shapes within that area.

\subsection{}\label{section-3}

\subsection{Controls in the Sounds
Tab}\label{controls-in-the-sounds-tab}

There is no Sound Editor{[}{]}\{index=``controls in the Sounds tab''\}
in Snap\emph{!}, and also no current sound the way there's a current
costume for each sprite. (The sprite always has an appearance unless
hidden, but it doesn't sing unless explicitly asked.) So the context
menu for sounds has only rename, delete, and export options, and it has
a clickable button labeled Play or Stop as appropriate. There is a sound
\emph{recorder,} which appears if you click the red record button ( ):

The first, round button starts recording. The second, square button
stops recording. The third, triangular button plays back a recorded
sound. If you don't like the result, click the round button again to
re-record. When you're satisfied, push the Save button. If you need a
sound editor, consider the free (both senses)
\href{http://audacity.sourceforge.net}{https://audacity.sourceforge.net}.

\section{Keyboard Editing}\label{keyboard-editing}

An ongoing area of research is how to make visual programming languages
usable by people with visual or motoric disabilities. As a first step in
this direction, we provide a keyboard editor, so that you can create and
edit scripts without tracking the mouse. So far, not every user
interface element is controllable by keyboard, and we haven't even begun
providing \emph{output} support, such as interfacing with a speech
synthesizer. This is an area in which we know we have a long way to go!
But it's a start. The keyboard editor may also be useful to anyone who
can type faster than they can drag blocks.

\subsection{Starting and stopping the keyboard
editor}\label{starting-and-stopping-the-keyboard-editor}

There are three ways to start the keyboard
editor{[}{]}\{index=``keyboard editor''\} .
Shift-clicking{[}{]}\{index=``Shift-click (keyboard editor)''\} anywhere
in the scripting area will start the editor at that point: either
editing an existing script or, if you shift-click on the background of
the scripting area, editing a new script at the mouse position.
Alternatively, typing shift-enter{[}{]}\{index=``shift-enter (keyboard
editor)''\} will start the editor on an existing script, and you can use
the tab key to switch to another script. Or you can click the keyboard
button at the top of the scripting area.

When the script editor is running, its position is represented by a
blinking white bar:

To leave the keyboard editor, type the escape key{[}{]}\{index=``escape
key (keyboard editor)''\} , or just click on the background of the
scripting area.

\subsection{Navigating in the keyboard
editor}\label{navigating-in-the-keyboard-editor}

To move to a different script, type the tab key{[}{]}\{index=``tab key
(keyboard editor)''\} . Shift-tab{[}{]}\{index=``Shift-tab (keyboard
editor)''\} to move through the scripts in reverse order.

A script is a vertical stack of command blocks. A command block may have
input slots, and each input slot may have a reporter block in it; the
reporter may itself have input slots that may have other reporters. You
can navigate through a script quickly by using the up
arrow{[}{]}\{index=``up arrow (keyboard editor)''\} and down
arrow{[}{]}\{index=``down arrow (keyboard editor)''\} keys to move
between command blocks. Once you find the command block that you want to
edit, the left{[}{]}\{index=``left arrow (keyboard editor)''\} and right
arrow{[}{]}\{index=``right arrow (keyboard editor)''\} keys move between
editable items within that command. (Left and right arrow when there are
no more editable items within the current command block will move up or
down to another command block, respectively.) Here is a sequence of
pictures showing the results of repeated right arrow keys starting from
the position shown above:

You can rearrange scripts within the scripting area from the keyboard.
Typing shift-arrow keys{[}{]}\{index=``shift-arrow keys (keyboard
editor)''\} (left, right, up, or down) will move the current script. If
you move it onto another script, the two won't snap together; the one
you're moving will overlap the one already there. This means that you
can move across another script to get to a free space.

\subsection{Editing a script}\label{editing-a-script}

Note that the keyboard editor \emph{focus,} the point shown as a white
bar or halo, is either \emph{between} two command blocks or \emph{on} an
input slot. The editing keys do somewhat different things in each of
those two cases.

The backspace key{[}{]}\{index=``backspace key (keyboard editor)''\}
deletes a block. If the focus{[}{]}\{index=``focus (keyboard editor)''\}
is between two commands, the one \emph{before} (above) the blinking bar
is deleted. If the focus is on an input slot, the reporter in that slot
is deleted. (If that input slot has a default value, it will appear in
the slot.) If the focus is on a \emph{variadic} input (one that can
change the number of inputs by clicking on arrowheads), then \emph{one}
input slot is deleted. (When you right-arrow into a variadic input, the
focus first covers the entire thing, including the arrowheads; another
right-arrow focuses on the first slot within that input group. The focus
is ``on the variadic input'' when it covers the entire thing.)

The enter key does nothing if the focus is between commands, or on a
reporter. If the focus is on a variadic input, the enter
key{[}{]}\{index=``enter key (keyboard editor)''\} adds one more input
slot. If the focus is on a white input slot (one that doesn't have a
reporter in it), then the enter key selects that input slot for
\emph{editing;} that is, you can type into it, just as if you'd clicked
on the input slot. (Of course, if the focus is on an input slot
containing a reporter, you can use the backspace key to delete that
reporter, and then use the enter key to type a value into it.) When you
finish typing the value, type the enter key again to accept it and
return to navigation, or the escape key if you decide not to change the
value already in the slot.

The space key is used to see a menu of possibilities for the input slot
in focus. It does nothing unless the focus is on a single input slot. If
the focus is on a slot with a pulldown menu of options, then the space
key{[}{]}\{index=``space key (keyboard editor)''\} shows that menu. (If
it's a block-colored slot, meaning that only the choices in the menu can
be used, the enter key will do the same thing. But if it's a white slot
with a menu, such as in the turn blocks, then enter lets you type a
value, while space shows the menu.) Otherwise, the space key shows a
menu of variables available at this point in the script. In either case,
use the up and down arrow keys to navigate the menu, use the enter key
to accept the highlighted entry, or use the escape key to leave the menu
without choosing an option.

Typing any other character key (not special keys on fancy keyboards that
do something other than generating a character) activates the
\emph{block search palette.} This palette, which is also accessible by
typing control-F or command-F outside the keyboard editor, or by
clicking the search button floating at the top of the palette, has a
text entry field at the top, followed by blocks whose title text
includes what you type. The character key you typed to start the block
search palette is entered into the text field, so you start with a
palette of blocks containing that character. Within the palette, blocks
whose titles \emph{start} with the text you type come first, then blocks
in which \emph{a word} of the title starts with the text you type, and
finally blocks in which the text appears inside a word of the title.
Once you have typed enough text to see the block you want, use the arrow
keys to navigate to that block in the palette, then enter to insert that
block, or escape to leave the block search palette without inserting the
block. (When not in the keyboard editor, instead of navigating with the
arrow keys, you drag the block you want into the script, as you would
from any other palette.)

If you type an arithmetic operator (+-*/) or comparison operator
(\textless=\textgreater) into the block search text box, you can type an
arbitrarily complicated expression, and a collection of arithmetic
operator blocks will be constructed to match:

As the example shows, you can also use parentheses for grouping, and
non-numeric operands are treated as variables or primitive functions. (A
variable name entered in this way may or may not already exist in the
script. Only round and the ones in the pulldown menu of the sqrt block
can be used as function names.)

\subsection{Running the selected
script}\label{running-the-selected-script}

Type control-shift-enter{[}{]}\{index=``control-shift-enter (keyboard
editor)''\} to run the script with the editor focus, like clicking the
script.

\section{Controls on the Stage{[}{]}}\label{controls-on-the-stage}

The stage is the area in the top right of the Snap\emph{!} window in
which sprites move.

\subsection{Sprites}\label{sprites}

Most sprites can be moved by clicking and dragging them. (If you have
unchecked the draggable checkbox{[}{]}\{index=``draggable checkbox''\}
for a sprite, then dragging it has no effect.)
Control-clicking/right-clicking a sprite shows this context menu:

The duplicate option{[}{]}\{index=``duplicate option''\} makes another
sprite with copies of the same scripts, same costumes, etc., as this
sprite. The new sprite starts at a randomly chosen position different
from the original, so you can see quickly which is which. The new sprite
is \emph{selected:} It becomes the current sprite, the one shown in the
scripting area. The clone option makes a permanent clone of this sprite,
with some shared attributes, and selects it.

The delete option{[}{]}\{index=``delete option''\} deletes the sprite.
It's not just hidden; it's gone for good. (But you can undelete it by
clicking the wastebasket just below the right edge of the stage.) The
edit option{[}{]}\{index=``edit option''\} selects the sprite. It
doesn't actually change anything about the sprite, despite the name;
it's just that making changes in the scripting area will change this
sprite.

The move option{[}{]}\{index=``move option''\} shows a ``move handle''
inside the sprite (the diagonal striped square in the middle):

You can ordinarily just grab and move the sprite without this option,
but there are two reasons you might need it: First, it works even if the
``draggable'' checkbox above the scripting area is unchecked. Second, it
works for part sprites relative to their anchor; ordinarily, dragging a
part moves the entire nested sprite.

The rotate option displays a rotation menu:

You can choose one of the four compass directions in the lower part (the
same as in the point in direction block) or use the mouse to rotate the
handle on the dial in 15° increments.

The pivot option{[}{]}\{index=``pivot option''\} shows a crosshair
inside the sprite:

You can click and drag the crosshair anywhere onstage to set the
costume's pivot point. (If you move it outside the sprite, then turning
the sprite will revolve as well as rotate it around the pivot.) When
done, click on the stage not on the crosshair. Note that, unlike moving
the pivot point in the Paint Editor, this technique does not visibly
move the sprite on the stage. Instead, the values of x position and y
position will change.

The edit option{[}{]}\{index=``edit option''\} makes this the selected
sprite, highlighting it in the sprite corral and showing its scripting
area. If the sprite was a temporary clone{[}{]}\{index=``temporary
clone''\} , it becomes permanent.

The export\ldots{} option{[}{]}\{index=``export option''\} saves, or
opens a new browser tab containing, the XML text representation of the
sprite. (Not just its costume, but all of its costumes, scripts, local
variables and blocks, and other properties.) You can save this tab into
a file on your computer, and later import the sprite into another
project. (In some browsers, the sprite is directly saved into a file.)

\subsection{Variable watchers}\label{variable-watchers}

Right-clicking on a variable watcher shows this menu:

The first section of the menu lets you choose one of three
visualizations of the watcher:

The first (normal) {[}{]}\{index=``normal option''\} visualization is
for debugging. The second (large) {[}{]}\{index=``large option''\} is
for displaying information to the user of a project, often the score in
a game. And the third (slider) {[}{]}\{index=``slider option''\} is for
allowing the user to control the program behavior interactively. When
the watcher is displayed as a slider, the middle section of the menu
allows you to control the range of values possible in the slider. It
will take the minimum value{[}{]}\{index=``slider min\ldots{} option''\}
when the slider is all the way to the left, the maximum
value{[}{]}\{index=``slider max\ldots{} option''\} when all the way to
the right.

The third section of the menu allows data to be passed between your
computer and the variable. The import\ldots{}
option{[}{]}\{index=``import\ldots{} option''\} will read a computer
text file. Its name must end with .txt, in which case the text is read
into the variable as is, or .csv{[}{]}\{index=``.csv file''\} or
.json{[}{]}\{index=``.json file''\} , in which case the text is
converted into a list structure, which will always be a two-dimensional
array for csv (comma-separated values) data, but can be any shape for
json data. The raw data\ldots{} option{[}{]}\{index=``raw data\ldots{}
option''\} prevents that conversion to list form. The export\ldots{}
option{[}{]}\{index=``export\ldots{} option''\} does the opposite
conversion, passing a text-valued variable value into a .txt
file{[}{]}\{index=``.txt file''\} unchanged, but converting a list value
into csv format if the list is one- or two-dimensional, or into json
format if the list is more complicated. (The scalar values within the
list must be numbers and/or text; lists of blocks, sprites, costumes,
etc. cannot be exported.)

An alternative to using the import\ldots{} option is simply to drag the
file onto the Snap\emph{!} window, in which case a variable will be
created if necessary with the same name as the file (but without the
extension).

If the value of the variable is a list, then the menu will include an
additional blockify option{[}{]}\{index=``blockify option''\} ; clicking
it will generate an expression with nested list blocks that, if
evaluated, will reconstruct the list. It's useful if you imported a list
and then want to write code that will construct the same list later.

\subsection{The stage itself}\label{the-stage-itself}

Control-clicking/right-clicking on the stage background (that is,
anywhere on the stage except on a sprite or watcher) shows the stage's
own context menu:

The stage's edit option{[}{]}\{index=``edit option''\} selects the
stage, so the stage's scripts and backgrounds are seen in the scripting
area. Note that when the stage is selected, some blocks, especially the
Motion ones, are not in the palette area because the stage can't move.

The show all option{[}{]}\{index=``show all option''\} makes all sprites
visible, both in the sense of the show block and by bringing the sprite
onstage if it has moved past the edge of the stage.

The pic\ldots{} option saves, or{[}{]}\{index=``pic\ldots{} option''\}
opens a browser tab with, a picture of everything on the stage: its
background, lines drawn with the pen, and any visible sprites. What you
see is what you get. (If you want a picture of just the background,
select the stage, open its costumes tab, control-click/right-click on a
background, and export it.)

The pen trails option{[}{]}\{index=``pen trails option''\} creates a new
costume for the currently selected sprite consisting of all lines drawn
on the stage by the pen of any sprite. The costume's rotation center
will be the current position of the sprite.

If you previously turned on the log pen vectors option, and there are
logged vectors, the menu includes an extra option,
svg\ldots{[}{]}\{index=``svg\ldots{} option''\} , that exports a picture
of the stage in vector format. Only lines are logged, not color regions
made with the fill block.

\section{The Sprite Corral{[}{]}\{index=``sprite corral''\} and Sprite
Creation
Buttons{[}{]}}\label{the-sprite-corralindexsprite-corral-and-sprite-creation-buttons}

Between the stage and the sprite corral at the bottom right of the
Snap\emph{!} window is a dark grey bar containing three buttons at the
left and one at the right. The first three are used to create a new
sprite. The first button makes a sprite with just the turtle costume,
with a randomly chosen position and pen color. (If you hold down the
Shift key while clicking, the new sprite's direction will also be
random.) The second button makes a sprite and opens the Paint Editor so
that you can make your own costume for it. (Of course you could click
the first button and then click the paint button in its costumes tab;
this paint button is a shortcut{[}{]}\{index=``shortcut''\} for all
that.) Similarly, the third button uses your camera, if possible, to
make a costume for the new sprite.

The trash can button at the right has two uses. You can drag a sprite
thumbnail onto it from the sprite corral to delete that sprite, or you
can click it to undelete a sprite you deleted by accident.

In the sprite corral, you click on a sprite's ``thumbnail'' picture to
select that sprite (to make it the one whose scripts, costumes, etc. are
shown in the scripting area). You can drag sprite thumbnails (but not
the stage one) to reorder them; this has no special effect on your
project, but lets you put related ones next to each other, for example.
Double-clicking a thumbnail flashes a halo around the actual sprite on
the stage.

You can right-click/control-click a sprite's thumbnail to get this
context menu:

The show option{[}{]}\{index=``show option''\} makes the sprite visible,
if it was hidden, and also brings it onto the stage, if it had moved
past the stage boundary. The next three options are the same as in the
context menu of the actual sprite on the stage, discussed above.

The parent\ldots{} option{[}{]}\{index=``parent\ldots{} option''\}
displays a menu of all other sprites, showing which if any is this
sprite's parent, and allowing you to choose another sprite (replacing
any existing parent). The release option{[}{]}\{index=``release
option''\} is shown only if this sprite is a
(permanent{[}{]}\{index=``permanent clone''\} , or it wouldn't be in the
sprite corral) clone; it changes the sprite to a temporary clone. (The
name is supposed to mean that the sprite is released from the corral.)
The export\ldots{} option{[}{]}\{index=``export\ldots{} option''\}
exports the sprite, like the same option on the stage.

The context menu for the stage thumbnail has only one option,
pic\ldots{[}{]}\{index=``pic\ldots{} option''\} , which takes a picture
of everything on the stage, just like the same option in the context
menu of the stage background. If pen trails are being logged, there will
also be an svg\ldots{} option.

If your project includes scenes{[}{]}\{index=``scenes''\} , then under
the stage icon in the sprite corral will be the \emph{scene corral:}

Clicking on a scene will select it; right-clicking will present a menu
in which you can rename, delete, or export the scene.

\section{\texorpdfstring{Preloading a Project{[}{]}\{index=``preloading
a project''\} when Starting
Snap\emph{!}}{Preloading a Project{[}{]}\{index=``preloading a project''\} when Starting Snap!}}\label{preloading-a-projectindexpreloading-a-project-when-starting-snap}

There are several ways to include a pointer to a project in the URL when
starting Snap\emph{!}{[}{]}\{index=``starting Snap!''\} in order to load
a project automatically. You can think of such a URL as just running the
project rather than as running Snap\emph{!}, especially if the URL says
to start in presentation mode and click the green flag. The general form
is

https://snap.berkeley.edu/run\#\textbf{\emph{verb}}:\textbf{\emph{project}}\&\textbf{\emph{flag}}\&\textbf{\emph{flag}}\ldots{}

The ``verb'' above can be any of open{[}{]}\{index=``open (startup
option)''\} , run{[}{]}\{index=``run (startup option)''\} ,
cloud{[}{]}\{index=``cloud (startup option)''\} ,
present{[}{]}\{index=``present (startup option)''\} , or
dl{[}{]}\{index=``dl (startup option)''\} . The last three are for
shared projects in the Snap\emph{!} cloud; the first two are for
projects that have been exported and made available anywhere on the
Internet.

Here's an example that loads a project stored at the Snap\emph{!} web
site (not the Snap\emph{!} cloud!):

https://snap.berkeley.edu/run\#open:https://snap.berkeley.edu/snapsource/Examples/vee.xml

The project file will be opened, and Snap\emph{!} will start in edit
mode (with the program visible). Using \#run: instead of \#open: will
start in presentation mode (with only the stage visible) and will
``start'' the project by clicking the green flag. (``Start'' is in
quotation marks because there is no guarantee that the project includes
any scripts triggered by the green flag. Some projects are started by
typing on the keyboard or by clicking a sprite.)

If the verb is run, then you can also use any subset of the following
flags:

\&editMode{[}{]}\{index=``editMode (startup option)''\} Start in edit
mode, not presentation mode.

\&noRun{[}{]}\{index=``noRun (startup option)''\} Don't click the green
flag.

\&hideControls{[}{]}\{index=``hideControls (startup option)''\} Don't
show the row of buttons above the stage (edit mode, green flag, pause,
stop).

\&lang={[}{]}\{index=``lang= (startup option)''\} fr Set language to (in
this example) French.

\&noCloud Don't allow cloud operations from this project (for running
projects from unknown

sources that include JavaScript code)

\&noExitWarning{[}{]}\{index=``noExitWarning (startup option)''\} When
closing the window or loading a different URL, don't show the browser

``are you sure you want to leave this page'' message.

\&blocksZoom=n Like the Zoom blocks option in the Settings menu.

The last of these flags is intended for use on a web page in which a
Snap\emph{!} window is embedded.

Here's an example that loads a shared (public) project from the
Snap\emph{!} cloud:

https://snap.berkeley.edu/run\#present:Username=jens\&ProjectName=tree\%20animation

(Note that ``Username'' and ``ProjectName'' are TitleCased, even though
the flags such as ``noRun'' are camelCased. Note also that a space in
the project name must be represented in Unicode as \%20.) The verb
present behaves like run: it ordinarily starts the project in
presentation mode, but its behavior can be modified with the same four
flags as for run. The verb cloud (yes, we know it's not a verb in its
ordinary use) behaves like open except that it loads from the
Snap\emph{!} cloud rather than from the Internet in general. The verb dl
(short for ``download'') does not start Snap\emph{!} but just downloads
a cloud-saved project to your computer as an .xml file. This is useful
for debugging; sometimes a defective project that Snap\emph{!} won't run
can be downloaded, edited, and then re-saved to the cloud.

\section{Mirror Sites}\label{mirror-sites}

If the site snap.berkeley.edu is ever unavailable, you can load
Snap\emph{!} at the following mirror sites{[}{]}\{index=``mirror
sites''\} :

\begin{itemize}
\item
  https://bjc.edc.org{[}{]}\{index=``bjc.edc.org''\}
  /snapsource/snap.html
\item
  https://cs10.org{[}{]}\{index=``cs10.org''\} /snap
\end{itemize}

\cleardoublepage
\phantomsection
\addcontentsline{toc}{part}{Appendices}
\appendix

\chapter{\texorpdfstring{Appendix A. Snap\emph{!} color
library}{Appendix A. Snap! color library}}\label{appendix-a.-snap-color-library}

The Colors and Crayons library{[}{]}\{index=``Colors and Crayons
library''\} provides several tools for manipulating color. Although its
main purpose is controlling a sprite's pen color, it also establishes
colors as a first class data type:

For people who just want colors in their projects without having to be
color experts, we provide two simple mechanisms: a \emph{color
number}{[}{]}\{index=``color numbers''\} scale with a broad range of
continuous color variation and a set of 100 \emph{crayons} organized by
color family (ten reds, ten oranges, etc.) The
crayons{[}{]}\{index=``crayons''\} include the block colors:

For experts, we provide color selection by RGB, HSL, HSV, X11/W3C names,
and variants on those scales.

\subsection{Introduction to Color}\label{introduction-to-color}

Your computer monitor can display millions of colors, but you probably
can't distinguish that many. For example, here's red 57, green 180, blue
200: And here's red 57, green \emph{182,} blue 200: You might be able to
tell them apart if you see them side by side: \ldots{} but maybe not
even then.

Color space{[}{]}\{index=``color space''\} ---the collection of all
possible colors---is three-dimensional, but there are many ways to
choose the dimensions. RGB{[}{]}\{index=``RGB''\} (red-green-blue), the
one most commonly used in computers, matches the way TVs and displays
produce color. Behind every dot on the screen are three tiny lights: a
red one, a green one, and a blue one. But if you want to print colors on
paper, your printer probably uses a different set of three colors:
CMY{[}{]}\{index=``CMY''\} (cyan-magenta-yellow). You may have seen the
abbreviation CMYK{[}{]}\{index=``CMYK''\} , which represents the common
technique of adding black ink to the collection. (Mixing cyan, magenta,
and yellow in equal amounts is supposed to result in black ink, but
typically it comes out a muddy brown instead, because chemistry.) Other
systems that try to mimic human perception are
HSL{[}{]}\{index=``HSL''\} (hue-saturation-lightness) and
HSV{[}{]}\{index=``HSV''\} (hue-saturation-value). There are many, many
more, each designed for a particular purpose.

If you are a color professional---a printer, a web designer, a graphic
designer, an artist---then you need to understand all this. It can also
be interesting to learn about. For example, there are colors that you
can see but your computer display can't generate. If that intrigues you,
look up \href{https://en.wikipedia.org/wiki/Color_theory}{color theory}
{[}{]}\{index=``color theory''\} in Wikipedia.

\subsection{Crayons and Color Numbers}\label{crayons-and-color-numbers}

But if you just want some colors in your project, we provide a simple,
one-dimensional subset of the available colors. Two subsets, actually:
\emph{crayons} and \emph{color numbers.} Here's the difference:

The first row shows 100 distinct colors. They have names; this is
pumpkin{[}{]}\{index=``pumpkin''\} , and this is
denim{[}{]}\{index=``denim''\} . You're supposed to think of them as a
big box of 100 crayons{[}{]}\{index=``crayons''\} . They're arranged in
families: grays, pinks, reds, browns, oranges, etc. But they're not
consistently ordered within a family; you'd be unlikely to say ``next
crayon'' in a project. (But look at the crayon spiral on page
\hyperref[spirals]{145}.) Instead, you'd think ``I want this to look
like a really old-fashioned photo'' and so you'd find
sepia{[}{]}\{index=``sepia''\} as crayon number 33. You don't have to
memorize the numbers! You can find them in a menu with a submenu for
each family.{[}{]}\{index=``set pen block''\}

Or, if you know the crayon name, just .

The crayon numbers are chosen so that skipping by 10 gives a sensible
box of ten crayons{[}{]}\{index=``box of ten crayons''\} :

Alternatively, skipping by 5 gives a still-sensible set of twenty
crayons{[}{]}\{index=``box of twenty crayons''\} :

The set of \emph{color numbers} is arranged so that each color number is
visually near each of its neighbors. Bright and dark colors alternate
for each family. Color numbers{[}{]}\{index=``color numbers''\} range
from 0 to 99, like crayon numbers, but you can use fractional numbers to
get as tiny a step as you like:

(``As tiny as you like'' isn't \emph{quite} true because in the end,
your color has to be rounded to integer RGB values for display.)

Both of these scales include the range of shades of
gray{[}{]}\{index=``gray''\} , from black to white. Since black is the
initial pen color, and black isn't a hue, Scratch and Snap\emph{!} users
would traditionally try to use set color to escape from black, and it
wouldn't work. By including black in the same scale as other colors, we
eliminate the Black Hole problem{[}{]}\{index=``Black Hole problem''\}
if people use only the recommended color scales.

We are making a point of saying ``color number'' for what was sometimes
called just ``color'' in earlier versions of the library, because we now
reserve the name ``color'' for an actual color, an instance of the color
data type.\\
\textbf{How to Use the Library}

There are three library blocks specifically about controlling the pen.
They have the same names as three of the primitive Pen blocks:

The first (Pen block-colored) input slot is used to select which color
scale you want to use. (These blocks also allow reading or setting two
block properties that are not colors: the pen size and its
transparency.) The pen reporter{[}{]}\{index=``pen block''\} requires no
other inputs; it reports the state of the pen in whatever dimension you
choose.

As the last example shows, you can't ask for the pen color in a scale
incompatible with how you set it, unless the block can deduce what you
want from what it knows about the current pen color.

The change pen block{[}{]}\{index=``change pen block''\} applies only to
numeric scales (including vectors of three or four numbers). It adds its
numeric or list input to the current pen value(s), doing vector
(item-by-item) addition for vector scales.

The set pen block{[}{]}\{index=``set pen block''\} changes the pen color
to the value(s) you specify. The meaning of the white input slots
depends on which attribute of the pen you're setting:

In the last example, the number 37 sets the \emph{transparency,} on the
scale 0=opaque, 100=invisible. (All color attributes are on a 0--100
scale except for RGB components, which are 0--255.) A
transparency{[}{]}\{index=``transparency''\} value can be combined with
any of these attribute scales.

The library also includes two constructors and a selector for colors as
a data type:

The latter two are inverses of each other, translating between colors
and their attributes. The color from block's{[}{]}\{index=``color from
block''\} attribute menu has fewer choices than the similar set pen
block because you can, for example, set the Red value of the existing
pen color leaving the rest unchanged, but when creating a color out of
nothing you have to provide its entire specification, e.g., all of Red,
Green, and Blue, or the equivalent in other scales. (As you'll see on
the next page, we provide two \emph{linear} (one-dimensional) color
scales that allow you to specify a color with a single number, at the
cost of including only a small subset of the millions of colors your
computer can generate.) If you have a color and want another color
that's the same except for one number, as in the Red example, you can
use this block:

Finally, the library includes the mix block{[}{]}\{index=``mix block''\}
and a helper:

We'll have more to say about these after a detour through color theory.

That's all you have to know about colors! \emph{Crayons} for specific
interesting ones, \emph{color numbers} for gradual transformation from
one color to the next. But there's a bit more to say, if you're
interested. If not, stop here. (But look at the samples of the different
scales on page \hyperref[spirals]{145}.)\\
\textbf{More about Colors: Fair Hues and Shades}

Several of the three-dimensional arrangements of colors use the concept
of ``hue{[}{]}\{index=''hue''\} ,'' which more or less means where a
color would appear in a rainbow{[}{]}\{index=``rainbow''\}
(magenta{[}{]}\{index=``magenta''\} , near the right, is
\href{https://en.wikipedia.org/wiki/Magenta}{a long story}):

These are called ``spectral{[}{]}\{index=''spectral colors''\} ''
colors, after the \emph{spectrum} of rainbow colors. But these colors
aren't equally distributed. There's an awful lot of green, hardly any
yellow, and just a sliver of orange. And no brown at all.

And this is already a handwave, because the range of colors that can be
generated by RGB monitors doesn't include some of the \emph{true}
spectral colors. See
\href{https://en.wikipedia.org/wiki/Spectral_color}{Spectral color} in
Wikipedia for all the gory details.

This isn't a problem with the physics of rainbows. It's in the human eye
and the human brain that certain ranges of wavelength of light waves are
lumped together as named colors. The eye is just ``tuned''
{[}{]}\{index=``rods and cones''\} to recognize a wide range of colors
as green. (See
\href{https://en.wikipedia.org/w/index.php?title=Rods_and_cones}{Rods
and Cones}.) And different human cultures give names to different color
ranges. Nevertheless, in old Scratch projects, you'd say change pen
color by 1 and it'd take forever to reach a color that wasn't green.

For color professionals, there are good reasons to want to work with the
physical rainbow hue layout. But for amateurs using a simplified,
one-dimensional color model, there's no reason not to use a more
programmer-friendly hue scale:

In this scale, each of the seven rainbow colors and brown get an equal
share. (Red's looks too small, but that's because it's split between the
two ends: hue 0 is pure red, brownish reds are to its right, and
purplish reds are wrapped around to the right end.) We call this scale
``fair hue{[}{]}\{index=''fair hue''\} '' because each color family gets
a fair share of the total hue range. (By the way, you were probably
taught ``\ldots{} green, blue, indigo{[}{]}\{index=''indigo''\} ,
violet'' in school, but it turns out that color names were different in
Isaac Newton's day, and the color he called ``blue'' is more like modern
cyan, while his ``indigo'' is more like modern blue. See Wikipedia
\href{https://en.wikipedia.org/wiki/Indigo}{Indigo}.)

Our \emph{color number} scale is based on fair hues, adding a range of
grays from black (color number 0) to white (color number14) and also
adding \emph{shades} of the spectral colors. (In color terminology, a
\emph{shade} is a darker version of a color; a lighter version is called
a \emph{tint.}) Why do we add shades{[}{]}\{index=``shade''\} but not
tints{[}{]}\{index=``tint''\} ? Partly because I find shades more
exciting. A shade of red can be dark candy apple red{[}{]}\{index=``dark
candy apple red''\} or maroon{[}{]}\{index=``maroon''\} , but a tint is
just some kind of pink{[}{]}\{index=``pink''\} . This admitted prejudice
is supported by an objective fact: Most projects are made on a white
background{[}{]}\{index=``white background''\} , so dark colors stand
out better than light ones.

So, in our color number scale, color numbers 0 to 14 are kinds of
gray{[}{]}\{index=``gray''\} ; the remaining color numbers go through
the fair hues, but alternating full-strength colors with shades.

crayons by 10

crayons by 5

crayons

fair hues

color numbers

color numbers by 5

color numbers by 10

This chart shows how the color scales{[}{]}\{index=``color scales''\}
discussed so far are related. Note that all scales range from 0 to 100;
the fair hues scale has been compressed in the chart so that similar
colors line up vertically. (Its dimensions are different because it
doesn't include the grays at the left. Since there are eight color
families, the pure, named fair hues are at multiples of 100/8=12.5,
starting with red=0.)

White is crayon 14 and color number 14. This value was deliberately
chosen \emph{not} to be a multiple of 5 so that the every-fifth-crayon
and every-tenth-crayon selections don't include it, so that all of the
crayons in those smaller boxes are visible against a
white{[}{]}\{index=``white''\} stage background.

Among purples{[}{]}\{index=``purple''\} , the official spectral
violet{[}{]}\{index=``violet''\} (crayon 90) is the end of the spectrum.
Magenta{[}{]}\{index=``magenta''\} , brighter than violet, isn't a
spectral color at all. \phantomsection\label{rainbow}{}(In the picture
at the left, the top part is the spectrum of white light spread out
through a prism; the middle part is a photograph of a rainbow, and the
bottom part is a digital simulation of a rainbow.) Magenta is a mixture
of red and blue. (attribution: Wikipedia user Andys. CC BY-SA.)

The light gray at color number 10 is slightly different from crayon 10
just because of roundoff in computing crayon values. Color number 90 is
different from crayon 90 because the official RGB violet (equal parts
red and blue) is actually lighter than spectral violet. The purple
family is also unusual because magenta, crayon and color number 95, is
lighter than the violet at 90. In other families, the color numbers,
crayons, and (scaled) fair hues all agree at multiples of ten. These
multiple-of-ten positions are the standard RGB primary and secondary
colors, e.g., the yellow at color number 50 is (255, 255, 0) in RGB.
(Gray, brown, and orange don't have such simple RGB settings.)

The color numbers at odd multiples of five are generally darker shades
than the corresponding crayons. The latter are often official named
shades, e.g., teal{[}{]}\{index=``teal''\} , crayon 65, is a
half-intensity shade of cyan{[}{]}\{index=``cyan''\} . The odd-five
\emph{color numbers,} though, are often darker, since they are chosen to
be the darkest color in a given family that's visibly different from
black. The pink at color number 15, though, is quite different from
crayon 15, because the former is a pure tint of red, whereas the crayon,
to get a more interesting pink, has a little magenta mixed in. Color
numbers at multiples of five are looked up in a table; other color
values are determined by linear interpolation in RGB space.
(\emph{Crayons} are of course all found by table lookup.)

The from color block{[}{]}\{index=``from color block''\} behaves
specially when you ask for the \emph{color number} of a color. Most
colors don't exactly match a color number, and for other attributes of a
color (crayon number, X11 name) you don't get an answer unless the color
exactly matches one of the names or numbers in that attribute. But for
color number, the block tries to find the \emph{nearest color
number}{[}{]}\{index=``nearest color number''\} to the color you
specify. The result will be only approximate; you can't use the number
you get to recreate the input color. But you can start choosing nearby
color numbers as you animate the sprite.

\subsection{Perceptual Spaces: HSV and
HSL}\label{perceptual-spaces-hsv-and-hsl}

RGB is the right way to think about colors if you're building or
programming a display monitor; CMYK is the right way if you're building
or programming a color printer. But neither of those coordinate systems
is very intuitive if you're trying to understand what color \emph{you
see} if, for example, you mix 37\% red light, 52\% green, and 11\% blue.
The \emph{hue} scale is one dimension of most attempts at a perceptual
scale. The square at the right has pale blues along the top edge, dark
blues along the right edge, various shades of gray toward the left,
black at the bottom, and pure spectral blue in the top right corner.
Although no other point in the square is pure blue, you can tell at a
glance that no other spectral color is mixed with the blue.

Aside from hue, the other two dimensions of a color space have to
represent how much white and/or black is mixed with the spectral color.
(Bear in mind that ``mixing black'' is a metaphor when it comes to
monitors. There really is black paint, but there's no such thing as
black light.) One such space, HSV{[}{]}\{index=``HSV''\} , has one
dimension for the amount of color (vs. white), called \emph{saturation,}
and one for the amount of black, imaginatively called \emph{value.} HSV
stands for Hue-Saturation{[}{]}\{index=``saturation''\}
-Value{[}{]}\{index=``value''\} . (Value is also called
\emph{brightness.}) The \emph{value} is actually measured backward from
the above description; that is, if value is 0, the color is pure black;
if value is 100, then a saturation of 0 means all white, no spectral
color; a saturation of 100 means no white at all. In the square in the
previous paragraph, the \emph{x} axis is the saturation and the \emph{y}
axis is the value. The entire bottom edge is black, but only the top
left corner is white. HSV is the traditional color space used in Scratch
and Snap\emph{!.} Set pen color set the hue; set pen shade set the
value. There was originally no Pen block to set the saturation, but
there's a set brightness effect Looks block to control the saturation of
the sprite's costume. (I speculate that the Scratch designers, like me,
thought tints were less vivid than shades against a white background, so
they made it harder to control tinting.)

attribution: Wikipedia user SharkD, CC BY-SA 3.0

But if you're looking at colors on a computer display, HSV isn't really
a good match for human perception. Intuitively, black and white should
be treated symmetrically. This is the HSL{[}{]}\{index=``HSL''\}
(hue-saturation-lightness{[}{]}\{index=``lightness''\} ) color space.
\emph{Saturation,} in HSL, is a measure of the \emph{grayness} or
\emph{dullness} of a color (how close it comes to being on a
black-and-white scale) and \emph{lightness} measures \emph{spectralness}
with pure white at one end, pure black at the other end, and spectral
color in the middle. The \emph{saturation} number is actually the
opposite of grayness: 0 means pure gray, and 100 means pure spectral
color, provided that the \emph{lightness} is 50, midway between black
and white. Colors with lightness other than 50 have some black or white
mixed in, but saturation 100 means that the color is as fully saturated
as it can be, given the amount of white or black needed to achieve that
lightness. Saturation less than 100 means that \emph{both white and
black} are mixed with the spectral color. (Such mixtures are called
\emph{tones} of the spectral color. Perceptually, colors with saturation
100\% don't look gray: but colors with saturation 75\% do:

Note that HSV and HSL both have a dimension called ``saturation,'' but
\emph{they're not the same thing!} In HSV, ``saturation'' means
non-whiteness, whereas in HSL it means non-grayness (vividness).

More fine print: It's misleading to talk about the spectrum of light
wavelengths as if it were the same as perceived hue. If your computer
display is showing you a yellow area, for example, it's doing it by
turning on its red and green LEDs over that area, and what hits your
retina \emph{is still two wavelengths of light, red and green,
superimposed.} You could make what's perceptually the same yellow by
using a single intermediate wavelength. Your eye and brain don't
distinguish between those two kinds of yellow. Also, your brain
automatically adjusts perceived hue to correct for differences in
illumination. When you place a monochrome object so that it's half in
sunlight and half in the shade, you see it as one even though what's
reaching your eyes from the two regions differs a lot. And, sadly, it's
HSL whose use of ``saturation'' disagrees with the official
international color vocabulary standardization committee. I learned all
this from \href{http://www.huevaluechroma.com/011.php}{this tutorial},
which you might find more coherent than jumping around Wikipedia if
you're interested.

Although traditional Scratch and Snap\emph{!} use HSV in programs, they
use HSL in the color picker{[}{]}\{index=``color picker''\} . The
horizontal axis is hue (fair hue{[}{]}\{index=``fair hue''\} , in this
version) and the vertical axis is \emph{lightness,} the scale with black
at one end and white at the other end. It would make no sense to have
only the bottom half of this selector (HSV Value) or only the top half
(HSV Saturation). And, given that you can only fit two dimensions on a
flat screen, it makes sense to pick HSL saturation (vividness) as the
one to keep at 100\%. (In this fair-hue picker, some colors appear
twice: ``spectral'' (50\% lightness) browns as shades (≈33\% lightness)
of red or orange, and shades of those browns.)

Software that isn't primarily about colors (so, \emph{not} including
Photoshop, for example) typically use HSV or HSL, with web-based
software more likely to use HSV because that's what's built into the
JavaScript{[}{]}\{index=``JavaScript''\} programming language provided
by browsers. But if the goal is to model human color perception, neither
of these color spaces is satisfactory, because they assume that all
full-intensity spectral colors are equally bright. But if you're like
most people, you see spectral yellow as much brighter than spectral blue
. There are better perceptual color spaces with names like
L*u*v*{[}{]}\{index=``L*u*v*''\} and L*a*b*{[}{]}\{index=``L*a*b*''\}
that are based on research with human subjects to determine true
perceived brightness. Wikipedia explains all this and more at
\href{https://en.wikipedia.org/wiki/HSL_and_HSV}{HSL and HSV}, where
they recommend ditching both of these simplistic color spaces. ☺

\subsection{Mixing Colors}\label{mixing-colors}

Given first class colors, the next question is, what operations apply to
them, the way arithmetic operators apply to numbers and higher order
functions apply to lists? The equivalent to adding numbers is mixing
colors, but unfortunately there isn't a simple answer to what that
means.

The easiest kind of color mixing to understand is \emph{additive}
mixing, which is what happens when you shine two colored lights onto a
(white) wall. It's also what happens in your computer screen, where each
dot (pixel) of an image is created by a tiny red light, a tiny green
light, and a tiny blue light that can be combined at different strengths
to make different colors. Essentially, additive
mixing{[}{]}\{index=``additive mixing''\} of two colors is computed by
adding the two red components, the two green components, and the two
blue components. It's not \emph{quite} that simple only because each
component of the result must be in the range 0 to 255. So, red (255, 0,
0) mixed with green (0, 255, 0) gives (255, 255, 0), which is yellow.
But red (255, 0, 0) plus yellow (255, 255, 0) can't give (510, 255, 0).
Just limiting the red in the result to 255 would mean that red plus
yellow is yellow, which doesn't make sense. Instead, if the red value
has to be reduced by half (from 510 to 255), then \emph{all three}
values must be reduced by half, so the result is (255, 128, 0), which is
orange. (Half of 255 is 127.5, but each RGB value must be an integer.)

A different kind of color mixing based on light is done when different
colored transparent plastic sheets are held in front of a white light,
as is done in theatrical lighting. In that situation, the light that
gets through both filters is what remains after some light is filtered
out by the first one and some of what's left is filtered out by the
second one. In red-green-blue terms, a red filter filters out green and
blue; a yellow filter allows red and green through, filtering out blue.
But there isn't any green light for the yellow filter to pass; it was
filtered out by the red filter. Each filter can only remove light, not
add light, so this is called \emph{subtractive} mixing:

Perhaps confusingly, the numerical computation of subtractive
mixing{[}{]}\{index=``subtractive mixing''\} is done by
\emph{multiplying} the RGB values, taken as fractions of the maximum
255, so red (1, 0, 0) times yellow (1, 1, 0) is red again.

Those are both straightforward to compute. Much, much more complicated
is trying to simulate the result of mixing
\emph{paints}{[}{]}\{index=``mixing paints''\} \emph{.} It's not just
that we'd have to compute a more complicated function of the red, green,
and blue values; it's that RGB values (or any other three-dimensional
color space) are inadequate to describe the behavior of
paints{[}{]}\{index=``paints''\} . Two paints can look identical, and
have the same RGB values, but may still behave very differently when
mixed with other colors. The differences are mostly due to the chemistry
of the paints, but are also affected by exactly how the colors are
mixed. The mixing is mostly subtractive; red paint \emph{absorbs} most
of the colors other than red, so what's reflected off the surface is
whatever isn't absorbed by the colors being mixed. But there can be an
additive component also.

The proper mathematical abstraction to describe a paint is a
\emph{reflectance} graph{[}{]}\{index=``reflectance graph''\} , like
this:

(These aren't paints, but minerals, and one software-generated spectrum,
from the US Geological Survey's
\href{https://www.usgs.gov/labs/spec-lab/capabilities/spectral-library}{Spectral
Library}. The details don't matter, just the fact that a graph like
these gives much more information than three RGB numbers.) To mix two
paints properly, you multiply the \emph{y} values (as fractions) at each
matching \emph{x} coordinate of the two graphs.

Having said all that, the mix block takes the colors it is given as
inputs and converts them into what we hope are \emph{typical} paint
reflectance spectra that would look like those colors, and then mixes
those spectra and converts back to RGB.

But unlike the other two kinds of mixing, in this case we can't say that
these colors are ``the right answer''; what would happen with real
paints depends on their chemical composition and how they're mixed.
There are three more mixing options, but these three are the ones that
correspond to real-world color mixing.

The mix block will accept any number of colors, and will mix them in
equal proportion. If (for any kind of mixing) you want more of one color
than another, use the color at weight block{[}{]}\{index=``color at
weight block''\} to make a ``weighted color'':

This mixes four parts red paint to one part green paint. All colors in a
mixture can be weighted:

(Thanks to \href{http://scottburns.us/subtractive-color-mixture/}{Scott
Burns{[}{]}\{index=``Burns, Scott''\}} for his help in understanding
paint mixing, along with
\href{http://www.huevaluechroma.com/061.php}{David
Briggs{[}{]}\{index='' Briggs, David''\}}'s tutorial. Remaining mistakes
are bh's.)

\subsection{tl;dr}\label{tldr}

For normal people{[}{]}\{index=``normal people''\} , Snap\emph{!}
provides three simple, one-dimensional scales: \emph{crayons} for
specific interesting colors, \emph{color numbers} for a continuum of
high-contrast colors with a range of hues and shading, and \emph{fair
hues} for a continuum without shading. For color
nerds{[}{]}\{index=``color nerds''\} , it provides three-dimensional
color spaces RGB, HSL, HSV, and fair-hue variants of the latter two.
\phantomsection\label{spirals}{}We recommend ``fair
HSL{[}{]}\{index=''fair HSL''\} '' for zeroing in on a desired color.

\subsection{Subappendix: Geeky details on fair
hue}\label{subappendix-geeky-details-on-fair-hue}

Color numbers, no grays.

All color numbers.

Crayons, no grays.

Just grays.

Fair hues.

The left graph shows that, unsurprisingly, all of the brown fair
hue{[}{]}\{index=``fair hue''\} s make essentially no progress in real
hue, with the orange-brown section actually a little retrograde, since
browns are really shades of orange and so the real hues overlap between
fair browns and fair oranges. Green makes up some of the distance,
because there are too many green real hues and part of the goal of the
fair hue scale is to squeeze that part of the hue spectrum. But much of
the catching up happens very quickly, between pure magenta at fair hue
93.75 and the start of the purple-red section at fair hue 97. This
abrupt change is unfortunate, but the alternatives involve either
stealing space from red or stealing space from purple (which already has
to include both spectral violet and RGB magenta). The graph has
discontinuous derivative at the table-lookup points, of which there are
two in each color family, one at the pure-named-RGB colors at multiples
of 12.5, and the other \emph{roughly} halfway to the next color family,
except for the purple family, which has lookup points at 87.5
(approximate spectral violet), 93.75 (RGB magenta), and 97 (turning
point toward the red family). (In the color picker, blue captures cyan
and purple space in dark shades. This, too, is an artifact of human
vision.)

The right graph shows the HSV saturation and value for all the fair
hues. Saturation is at 100\%, as it should be in a hue scale, except for
a very slight drop in part of the browns. (Browns are shades of orange,
not tints, so one would expect full saturation, except that some of the
browns are actually mixtures with related hues.) But value, also as
expected, falls substantially in the browns, to a low of about 56\%
(halfway to black) for the ``pure'' brown at 45° (fair hue 12.5). But
the curve is smooth, without inflection points other than that
minimum-value pure brown.

``Fair saturation{[}{]}\{index=''fair saturation''\} '' and ``fair
value{[}{]}\{index=''fair value''\} '' are by definition 100\% for the
entire range of fair hues. This means that in the browns, the real
saturation and value are the product (in percent) of the innate shading
of the specific brown fair hue and the user's fair saturation/value
setting. When the user's previous color setting was in a real scale and
the new setting is in a fair scale, the program assumes that the
previous saturation and value were entirely user-determined; when the
previous color setting was in a brown fair hue and the new setting is
also in a fair scale, the program remembers the user's intention from
the previous setting. (Internal calculations are based on HSV, even
though we recommend HSL to users, because HSV comes to us directly from
the JavaScript color management implementation.) This is why the set pen
block includes options for ``fair saturation'' and so on.

For the extra-geeky, here are the exact table lookup points (fair
hue{[}{]}\{index=``fair hue table''\} , {[}0,100{]}):

and here are the RGB settings at those points:

\subsection{\texorpdfstring{ Subappendix: Geeky details on color
numbers}{ Subappendix: Geeky details on color numbers}}\label{subappendix-geeky-details-on-color-numbers}

Here is a picture of integer color numbers, but remember that color
numbers are continuous. (As usual, ``continuous'' values are ultimately
converted to integer RGB values, so there's really some granularity.)
Color numbers 0-14 are continuously varying grayscale, from 0=black to
14=white. Color numbers 14+ε to 20 are linearly varying shades of pink,
with RGB Red at color number 20.

Beyond that point, in each color family, the multiple of ten color
number in the middle is the RGB standard color of that family, in which
each component is either 255 or 0. (Exceptions are brown, which is of
course darker than any of those colors; orange, with its green component
half-strength: {[}255, 127, 0{]}; and violet, discussed below.) The
following multiple of five is the number of the darkest color in that
family, although not necessarily the same hue as the multiple of ten
color number. Color numbers between the multiple of ten and the
following multiple of five are shades of colors entirely within the
family. Color numbers in the four \emph{before} the multiple of ten are
mixtures of this family and the one before it. So, for example, in the
green family, we have

55 Darkest yellow.

(55, 60) shades of yellow-green mixtures. As the color number increases,
both the hue and the lightness (or value, depending on your religion)
increase, so we get brighter and greener colors.

60 Canonical green, {[}0, 255, 0{]}, whose W3C color name is ``lime,''
not ``green.''

(60, 65) Shades of green. No cyan mixed in.

65 Darkest green.

(65,70) Shades of green-cyan mixtures.

In the color number chart{[}{]}\{index=``color chart''\} , all the dark
color numbers look a lot like black, but they're quite different. Here
are the darkest colors in each color number family.

Darkest yellow doesn't look entirely yellow. You might see it as
greenish or brownish. As it turns out, the darkest color that really
looks yellow is hardly dark at all. This color was hand-tweaked to look
neither green nor brown to me, but ymmv.

In some families, the center+5 \emph{crayon} is an important named
darker version of the center color: In the red family, {[}128, 0, 0{]}
is ``maroon.'' In the cyan family, {[}0, 128, 128{]} is ``teal.'' An
early version of the color number scale used these named shades as the
center+5 color number also. But on this page we use the word ``darkest''
advisedly: You can't find a darker shade of this family anywhere in the
color number scale, but you \emph{can} find lighter shades. Teal is
color number 73.1, (\(70 + 5 \bullet \frac{255 - 128}{255 - 50}\)),
because darkest cyan, color 75, is {[}0, 50, 50{]}. The color number for
maroon is left as an exercise for the reader.

The purple family is different from the others, because it has to
include both spectral violet and extraspectral RGB magenta. Violet is
usually given as RGB {[}128, 0, 255{]}, but that's much brighter than
the violet in an actual spectrum (see page \hyperref[rainbow]{142}). We
use {[}80, 0, 90{]}, a value hand-tweaked to look as much as possible
like the violet in rainbow photos, as color number 90. (\emph{Crayon} 90
is {[}128, 0, 255{]}.) Magenta, {[}255, 0, 255{]}, is color number 95.
This means that the colors get \emph{brighter,} not darker, between 90
and 95. The darkest violet is actually color number 87.5, so it's bluer
than standard violet, but still plainly a purple and not a blue. It's
{[}39,0,76{]}. It's \emph{not} hand-tweaked; it's a linear interpolation
between darkest blue, {[}0, 0, 64{]}, and the violet at color number 90.
I determined by experiment that color number 87.5 is the darkest one
that's still unambiguously purple. (According to Wikipedia, ``violet''
names only the spectral color, while ``purple'' is the name of the whole
color family.)

Here are the reference points for color numbers that are multiples of
five, except for item 4, which is used for color 14, not color 15:

The very pale three-input list blocks are for color numbers that are odd
multiples of five, generally the ``darkest'' members of each color
family. (The block colors were adjusted in Photoshop; don't ask how to
get blocks this color in Snap\emph{!}.)

\chapter{Appendix B. APL features}\label{appendix-b.-apl-features}

The book \emph{A Programming Language}{[}{]}\{index=``A Programming
Language''\} was published by mathematician Kenneth E.
Iverson{[}{]}\{index=``Iverson, Kenneth E.''\} in 1962. He wanted a
formal language that would look like what mathematicians write on
chalkboards. The then-unnamed language would later take its name from
the first letters of the words in the book's title. It was little-known
until 1964, when a formal description of the just-announced IBM
System/360{[}{]}\{index=``IBM System/360''\} in the \emph{IBM Systems
Journal} used APL{[}{]}\{index=``APL''\} notation. (Around the same
time, Iverson's associate Adin Falkoff{[}{]}\{index=``Falkoff, Adin''\}
gave a talk on APL to a New York Association for Computing Machinery
chapter, with an excited 14-year-old Brian Harvey in the audience.) But
it wasn't until 1966 that the first public implementation of the
language for the System/360 was published by IBM. (It was called
``APL\textbackslash360{[}{]}\{index=''APL\textbackslash360''\} ''
because the normal slash character / represents the ``reduce'' operator
in APL, while backslash is ``expand.'')

The crucial idea behind APL is that
mathematicians{[}{]}\{index=``mathematicians''\} think about collections
of numbers, one-dimensional \emph{vectors}{[}{]}\{index=``vectors''\}
and two-dimensional \emph{matrices}{[}{]}\{index=``matrices''\} \emph{,}
as valid objects in themselves, what computer scientists later learned
to call ``first class data{[}{]}\{index=''first class data''\} .'' A
mathematician who wants to add two vectors writes \textbf{\emph{v}1} +
\textbf{\emph{v}2}, not ``for i = 1 to length(v1),
result{[}i{]}=v1{[}i{]}+v2{[}i{]}.'' Same for a programmer using APL.

There are three kinds of function in APL: scalar
functions{[}{]}\{index=``scalar function''\} , mixed
functions{[}{]}\{index=``mixed function''\} , and
operators{[}{]}\{index=``operator (APL)''\} . A \emph{scalar function}
is one whose natural domain is individual numbers or text characters. A
\emph{mixed function} is one whose domain includes arrays (vectors,
matrices, or higher-dimensional collections). In Snap\emph{!}, scalar
functions are generally found in the green Operators palette, while
mixed functions are in the red Lists palette. The third category,
confusingly for Snap\emph{!} users, is called \emph{operators} in APL,
but corresponds to what we call higher order
functions{[}{]}\{index=``function, higher order''\} : functions whose
domain includes functions.

Snap\emph{!} hyperblocks{[}{]}\{index=``hyperblocks''\} are scalar
functions that behave like APL scalar functions: they can be called with
arrays as inputs, and the underlying function is applied to each number
in the arrays. (If the function is \emph{monadic,} meaning that it takes
one input, then there's no complexity to this idea. Take the square root
of an array, and you are taking the square root of each number in the
array. If the function is \emph{dyadic,} taking two inputs, then the two
arrays must have the same shape. Snap\emph{!} is more forgiving than
APL; if the arrays don't agree in number of dimensions, called the
\emph{rank} of the array, the lower-rank{[}{]}\{index=``rank''\} array
is matched repeatedly with subsets of the higher-rank one; if they don't
agree in length along one dimension, the result has the shorter length
and some of the numbers in the longer-length array are ignored. An
exception in both languages is that if one of the two inputs is a
scalar, then it is matched with every number in the other array input.)

As explained in Section IV.F, this termwise
extension{[}{]}\{index=``termwise extension''\} of scalar functions is
the main APL-like feature built into Snap\emph{!} itself. We also
include an extension of the item block{[}{]}\{index=``item block''\} to
address multiple dimensions, an extension to the length
block{[}{]}\{index=``length block''\} with five list functions from APL,
and a new primitive reshape block{[}{]}\{index=``reshape block''\} . The
APL library{[}{]}\{index=``APL library''\} extends the implementation of
APL features to include a few missing scalar functions and several
missing mixed functions and operators.

Programming in APL really is \emph{very} different in style from
programming in other languages, even Snap\emph{!}. This appendix can't
hope to be a complete reference for APL, let alone a tutorial. If you're
interested, find one of those in a library or a (probably used)
bookstore, read it, and \emph{do the exercises.} Sorry to sound like a
teacher, but the notation is sufficiently weird as to take a lot of
practice before you start to think in APL.

A note on versions: There is a widely standardized APL2, several
idiosyncratic extensions, and a successor language named J. The latter
uses plain ASCII characters, unlike the ones with APL in their names,
which use the mathematician's character set, with Greek letters,
typestyles (boldface and/or italics in books; underlined, upper case, or
lower case in APL) as loose type declarations, and symbols not part of
anyone's alphabet, such as ⌊ for floor and ⌈ for ceiling. To use the
original APL, you needed expensive special computer terminals. (This was
before you could download fonts in software. Today the more unusual APL
characters{[}{]}\{index=``APL character set''\} are in
Unicode{[}{]}\{index=``Unicode''\} at U+2336 to U+2395.) The character
set was probably the main reason APL didn't take over the world.
APL2{[}{]}\{index=``APL2''\} has a lot to recommend it for Snap\emph{!}
users, mainly because it moves from the original APL idea that all
arrays must be uniform in dimension, and the elements of arrays must be
numbers or single text characters, to our idea that a list can be an
element of another list, and that such elements don't all have to have
the same dimensions. Nevertheless, its mechanism for allowing both
old-style APL arrays and more general ``nested arrays'' is complicated
and hard for an APL beginner (probably all but two or three Snap\emph{!}
users) to understand. So we are starting with plain APL. If it turns out
to be wildly popular, we may decide later to include APL2 features.

Here are some of the guiding ideas in the design of the APL library:

Goal:~ Enable interested \textbf{Snap\emph{!}} users to learn the feel
and style of APL programming. It's really worth the effort. For example,
we didn't hyperize the = block because Snap\emph{!} users expect it to
give a single yes-or-no answer about the equality of two complete
structures{[}{]}\{index=``equality of complete structures''\} , whatever
their types and shapes. In APL, = is a scalar function; it compares two
numbers or two characters. How could APL users live without the ability
to ask if two \emph{structures} are equal? Because in APL you can say
\textbf{∧}/,a=b to get that answer. Reading from right to left, a=b
reports an array of Booleans (represented in APL as 0 for False, 1 for
True); the comma operator turns the shape of the array into a simple
vector; and \textbf{∧}/ means ``reduce with and''; ``reduce'' is our
combine function. That six-character program is much less effort than
the equivalent

in Snap\emph{!}. Note in passing that if you wanted to know \emph{how
many} corresponding elements of the two arrays are equal, you'd just use
+/ instead of \textbf{∧}/. Note also that our APLish blocks are a little
verbose, because they include up to three notations for the function:
the usual Snap\emph{!} name (e.g., flatten), the name APL programmers
use when talking about it (ravel{[}{]}\{index=``ravel block''\} ), and,
in yellow type, the symbol used in actual APL code (,). We're not
consistent about it; seems self-documenting. And LCM (and) is different
even though it has two names; it turns out that if you represent Boolean
values as 0 and 1, then the algorithm to compute the least common
multiple of two integers computes the and function if the two inputs
happen to be Boolean. Including the APL symbols serves two purposes: the
two or three Snap\emph{!} users who've actually programmed in APL will
be sure what function they're using, but more importantly, the ones who
are reading an APL tutorial while building programs in Snap\emph{!} will
find the block that matches the APL they're reading.

Goal:~ Bring the best and most general APL ideas into ``mainstream''
\textbf{Snap\emph{!}} programming style. Media
computation{[}{]}\{index=``media computation''\} , in particular,
becomes much simpler when scalar functions can be applied to an entire
picture or sound. Yes, map provides essentially the same capability, but
the notation gets complicated if you want to map over columns rather
than rows. Also, Snap\emph{!} lists are fundamentally one-dimensional,
but real data often have more dimensions. A Snap\emph{!} programmer has
to be thinking all the time about the convention that we represent a
matrix as a list of rows, each of which is a list of individual cells.
That is, row 23 of a spreadsheet{[}{]}\{index=``spreadsheet''\} is item
23 of spreadsheet, but column 23 is map (item 23 of \_) over
spreadsheet. APL treats rows and columns more symmetrically.

Non-goal:~ Allow programs written originally in APL to run in
\textbf{Snap\emph{!}} essentially unchanged.~ For example, in APL the
atomic text unit is a single character, and strings of characters are
lists. We treat a text string as scalar, and that isn't going to change.
Because APL programmers rarely use conditionals, instead computing
functions involving arrays of Boolean values to achieve the same effect,
the notation they do have for conditionals is primitive (in the sense of
Paleolithic{[}{]}\{index=``Paleolithic''\} , not in the sense of built
in). We're not changing ours.

Non-goal:~ Emulate the terse APL syntax. It's too bad, in a way; as
noted above, the terseness of expressing a computation affects APL
programmers' sense of what's difficult and what isn't. But you can't say
``terse'' and ``block language'' in the same sentence. Our whole
\emph{raison d'être} is to make it possible to build a program without
having to memorize the syntax or the names of functions, and to allow
those names to be long enough to be self-documenting. And APL's syntax
has its own issues, of which the biggest is that it's hard to use
functions with more than two inputs; because most mathematical dyadic
functions use infix notation (the function symbol between the two
inputs), the notion of ``left argument'' and ``right argument'' is
universal in APL documentation. The thing people most complain about,
that there is no operator precedence (like the
multiplication-before-addition rule in normal arithmetic notation),
really doesn't turn out to be a problem. Function grouping is strictly
right to left, so 2×3+4 means two times seven, not six plus four. That
takes some getting used to, but it really doesn't take long if you
immerse yourself in APL. The reason is that there are too many infix
operators for people to memorize a precedence table. But in any case,
block notation eliminates the problem, especially with Snap\emph{!}'s
zebra coloring. You can see and control the grouping by which block is
inside which other block's input slot. Another problem with APL's syntax
is that it bends over backward not to have reserved words, as opposed to
Fortran, its main competition back then. So the dyadic \textbf{○}
``circular functions'' function uses the left argument to select a trig
function. 1\textbf{○}x is sin(x), 2\textbf{○}x is cos(x), and so on.
\textbf{‾}1\textbf{○}x is arcsin(x). What's 0\textbf{○}x? Glad you
asked; it's\(\\\sqrt{1 - x^{2}}\).

\subsection{Boolean values}\label{boolean-values}

Snap\emph{!} uses distinct Boolean values true and false that are
different from other data types. APL uses 1 and 0, respectively. The APL
style of programming depends heavily on doing arithmetic on Booleans,
although their conditionals insist on only 0 or 1 in a Boolean input
slot, not other numbers. Snap\emph{!} \emph{arithmetic} functions treat
false as 0 and true as 1, so our APL library tries to report
Snap\emph{!} Boolean values from predicate functions.

\subsection{Scalar functions}\label{scalar-functions}

These are the scalar functions{[}{]}\{index=``scalar function''\} in the
APL library. Most of them are straightforward to figure out. The scalar
= block{[}{]}\{index=``scalar = block''\} provides an APL-style version
of = (and other exceptions) as a hyperblock that extends termwise to
arrays. Join, the only non-predicate non-hyper scalar primitive, has its
own scalar join block{[}{]}\{index=``scalar join block''\} . 7
deal{[}{]}\{index=``deal block''\} 52 reports a random vector of seven
numbers from 1 to 52 with no repetitions, as in dealing a hand of cards.
Signum{[}{]}\{index=``signum block''\} of a number reports 1 if the
number is positive, 0 if it's zero, or -1 if it's negative.
Roll{[}{]}\{index=``roll block''\} 6 reports a random roll of a
six-sided die. To roll 8 dice, use , which would look much more pleasant
as ?8⍴6. But perhaps our version is more instantly readable by someone
who didn't grow up with APL. All the library functions have help
messages available.

\subsection{Mixed functions}\label{mixed-functions}

Mixed functions include lists in their natural domain or range. That is,
one or both of its inputs \emph{must} be a list, or it always reports a
list. Sometimes both inputs are naturally lists; sometimes one input of
a dyadic mixed function is naturally a scalar, and the function treats a
list in that input slot as an implicit map, as for scalar functions.
This means you have to learn the rule for each mixed
function{[}{]}\{index=``function, mixed''\} individually.

The shape of function{[}{]}\{index=``shape of block''\} takes any input
and reports a vector of the maximum size of the structure along each
dimension. For a vector, it returns a list of length 1 containing the
length of the input. For a matrix, it returns a two-item list of the
number of rows and number of columns of the input. And so on for higher
dimensions. If the input isn't a list at all, then it has zero
dimensions, and shape of reports an empty vector.

Equivalent to the dimensions of primitive, as of 6.6.

Rank of isn{[}{]}\{index=``rank of block''\} 't an actual APL primitive,
but the composition ⍴⍴ (shape of shape of a structure), which reports
the number of dimensions of the structure (the length of its shape
vector), is too useful to omit. (It's very easy to type the same
character twice on the APL keyboard, but less easy to drag blocks
together.) Equivalent to the rank of primitive, as of 6.6.

Reshape{[}{]}\{index=``reshape block''\} takes a shape vector (such as
shape might report) on the left and any structure on the right. It
ignores the shape of the right input, stringing the atomic elements into
a vector in row-major order (that is, all of the first row left to
right, then all of the second row, etc.). (The primitive reshape takes
the inputs in the other order.) It then reports an array with the shape
specified by the first input containing the items of the second:

If the right input has more atomic elements than are required by the
left-input shape vector, the excess are ignored without reporting an
error. If the right input has too \emph{few} atomic elements, the
process of filling the reported array starts again from the first
element. This is most useful in the specific case of an atomic right
input, which produces an array of any desired shape all of whose atomic
elements are equal. But other cases are sometimes useful too:

Flatten{[}{]}\{index=``flatten block''\} takes an arbitrary structure as
input and reports a vector of its atomic elements in row-major order.
Lispians call this flattening the structure, but APLers call it
``ravel'' because of the metaphor of pulling on a ball of yarn, so what
they really mean is ``unravel.'' (But the snarky sound of that is
uncalled-for, because a more advanced version that we might implement
someday is more like raveling.) One APL idiom is to apply this to a
scalar in order to turn it into a one-element vector, but we can't use
it that way because you can't type a scalar value into the List-type
input slot. Equivalent to the primitive flatten of block.

ID ← \{(⍵,⍵)⍴1,⍵⍴0\}

Catenate{[}{]}\{index=``catenate block''\} is like our primitive append,
with two differences: First, if either input is a scalar, it is treated
like a one-item vector. Second, if the two inputs are of different rank,
the catenate function is recursively mapped over the higher-rank input:

Catenate vertically{[}{]}\{index=``catenate vertically block''\} is
similar, but it adds new rows instead of adding new columns.

Integers{[}{]}\{index=``integers block''\} (I think that's what it
stands for, although APLers just say ``iota'') takes a positive integer
input and reports a vector of the integers from 1 to the input. This is
an example of a function classed as ``mixed'' not because of its domain
but because of its range. The difference between this block and the
primitive numbers from block is in its treatment of lists as inputs.
Numbers from is a hyperblock, applying itself to each item of its input
list:

Iota{[}{]}\{index=``iota block''\} has a special meaning for list
inputs: The input must be a shape vector; the result is an array with
that shape in which each item is a list of the indices of the cell along
each dimension. A picture is worth 103 words, but Snap\emph{!} isn't so
good at displaying arrays with more than two dimensions, so here we
reduce each cell's index list to a string:

Dyadic iota is like the index of{[}{]}\{index=``index of block (APL)''\}
primitive except for its handling of multi-dimensional arrays. It looks
only for atomic elements, so a vector in the second input doesn't mean
to search for that vector as a row of a matrix, which is what it means
to index of, but rather to look separately for each item of the vector,
and report a list of the locations of each item. If the first input is a
multi-dimensional array, then the location of an item is a vector with
the indices along each row.

In this example, the 4 is in the second row, second column. (This is
actually an extension of APL iota, which is more like a hyperized index
of.) Generalizing, if the rank of the second input is less than the rank
of the first input by two or more, then iota looks for the entire second
input in the first input. The reported position is a vector whose length
is equal to the difference between the two ranks. If the rank of the
second input is one less than the rank of the first, the reported value
is a scalar, the index of the entire second input in the first.

However, if the two ranks are equal, then the block is hyperized; each
item of the second input is located in the first input. As the next
example shows, only the first instance of each item is found (e.g., the
1 in position 2, not the 1 in position 4); if an item does not occur in
the left input, what is reported is one more than the length of the left
input (here, 8).

Why the strange design decision to report length+1 when something isn't
found, instead of a more obvious flag value such as 0 or false? Here's
why:

Note that code has 27 items, not 26. The asterisk at the end is the
ciphertext is the translation of all non-alphabet characters (spaces and
the apostrophe in ``doesn't''). This is a silly example, because it
makes up a random cipher every time it's called, and it doesn't report
the cipher, so the recipient can't decipher the message. And you
wouldn't want to make the spaces in the message so obvious. But despite
being silly, the example shows the benefit of reporting length+1 as the
position of items not found.

The contained in block{[}{]}\{index=``contained in block''\} is like a
hyperized contains with the input order reversed. It reports an array of
Booleans the same shape as the left input. The shape of the right input
doesn't matter; the block looks only for atomic elements.

The blocks grade up{[}{]}\{index=``grade up block''\} and grade
down{[}{]}\{index=``grade down block''\} are used for sorting data.
Given an array as input, it reports a vector of the indices in which the
items (the rows, if a matrix) should be rearranged in order to be
sorted. This will be clearer with an example:

The result from grade up tells us that item 3 of \textbf{foo} comes
first in sorted order, then item 4, then 2, then 1. When we actually
select items of \textbf{foo} based on this ordering, we get the desired
sorted version. The result reported by grade down is almost the reverse
of that from grade up, but not quite, if there are equal items in the
list. (The sort is stable, so if there are equal items, then whichever
comes first in the input list will also be first in the sorted list.)

Why this two-step process? Why not just have a sort primitive in APL?
One answer is that in a database application you might want to sort one
array based on the order of another array:

This is the list of employees of a small company. (Taken from
\emph{Structure and Interpretation of Computer Programs} by Abelson and
Sussman. Creative Commons licensed.) Each of the smaller lists contains
a person's name, job title, and yearly salary. We would like to sort the
employees' names in big-to-small order of salary. First we extract
column 3 of the database, the salaries:

\strut \\
Then we use grade down to get the reordering indices:

At this point we \emph{could} use the index vector to sort the salaries:

But what we actually want is a list of \emph{names,} sorted by salary:

By taking the index vector from grade down of column 3 and telling item
to apply it to column 1, we get what we set out to find. As usual the
code is more elegant in APL: database{[}⍒database{[};3{]};1{]}.

In case you've forgotten, or would select the third \emph{row} of the
database; we need the list 3 in the second input slot of the outer list
to select by columns rather than by rows.

Select (if take{[}{]}\{index=``take block''\} ) or select all but (if
drop{[}{]}\{index=``drop block''\} ) the first (if
\emph{n}\textgreater0) or last (if \emph{n}\textless0)
\textbar{}\emph{n}\textbar{} items from a vector, or rows from a matrix.
Alternatively, if the left input is a two-item vector, select rows with
the first item and columns with the second.

The compress block{[}{]}\{index=``compress block''\} selects a subset of
its right input based on the Boolean values in its left input, which
must be a vector of Booleans whose length equals the length of the array
(the number of rows, for a matrix) in the right input. The block reports
an array of the same rank as the right input, but containing only those
rows whose corresponding Boolean value is true. The columns version
\textbf{⌿} is the same but selecting columns rather than selecting rows.

A word about the possibly confusing names of these blocks: There are two
ways to think about what they do. Take the standard / version, to avoid
talking about both at once. One way to think about it is that it selects
some of the rows. The other way is that it shortens the columns. For
Lispians, which includes you since you've learned about keep, the
natural way to think about / is that it keeps some of the rows. Since we
represent a matrix as a list of rows, that also fits with how this
function is implemented. (Read the code; you'll find a keep inside.) But
APL people think about it the other way, so when you read APL
documentation, / is described as operating on the last dimension (the
columns), while \textbf{⌿} is described as operating on rows. We were
more than a month into this project before I understood all this. You
get long block names so it won't take you a month!

Don't confuse this block with the reduce block{[}{]}\{index=``reduce
block''\} , whose APL symbol is also a slash. In that block, what comes
to the left of the slash is a dyadic combining function; it's the APL
equivalent of combine. This block is more nearly equivalent to keep. But
keep takes a predicate function as input, and calls the function for
each item of the second input. With compress, the predicate function, if
any, has already been called on all the items of the right input in
parallel, resulting in a vector of Boolean values. This is a typical APL
move; since hyperblocks are equivalent to an implicit map, it's easy to
make the vector of Booleans, because any scalar function, including
predicates, can be applied to a list instead of to a scalar. The reason
both blocks use the / character is that both of them reduce the size of
the input array, although in different ways.

The reverse row order{[}{]}\{index=``reverse block''\} , reverse column
order{[}{]}\{index=``reverse columns block''\} , and transpose
blocks{[}{]}\{index=``transpose block''\} form a group: the group of
reflections of a matrix. The APL symbols are all a circle with a line
through it; the lines are the different axes of reflection. So the
reverse row order block reverses which row is where; the reverse column
order block reverses which column is where; and the transpose block
turns rows into columns and vice versa:

Except for reverse row order, these work only on full arrays, not
ragged-right lists of lists, because the result of the other two would
be an array in which some rows had ``holes'': items 1 and 3 exist, but
not item 2. We don't have a representation for that. (In APL, all arrays
are full, so it's even more restrictive.)

\subsection{Higher order functions}\label{higher-order-functions}

The final category of function is operators{[}{]}\{index=``operator
(APL)''\} ---APL higher order functions{[}{]}\{index=``higher order
function''\} . APL has no explicit map function, because the hyperblock
capability serves much the same need. But APL2 did add an explicit map,
which we might get around to adding to the library next time around. Its
symbol is \textbf{¨} (diaeresis or umlaut).

The APL equivalent of keep is compress, but it's not a higher order
function. You create a vector of Booleans (0s and 1s, in APL) before
applying the function to the array you want to compress.

But APL does have a higher order version of combine:

The reduce block{[}{]}\{index=``reduce block''\} works just like
combine, taking a dyadic function and a list. The / version translates
each row to a single value; the \textbf{⌿} version translates each
column to a single value. That's the only way to think about it from the
perspective of combining individual elements: you are adding up, or
whatever the function is, the numbers in a single row (/) or in a single
column (\textbf{⌿}). But APLers think of a matrix as made up of vectors,
either row vectors or column vectors. And if you think of what these
blocks do as adding vectors, rather than adding individual numbers, it's
clear that in

the \emph{vector} (10, 26, 42) is the sum of \emph{column vectors} (1,
5, 9)+(2, 6, 10)+(3, 7, 11)+(4, 8, 12). In pre-6.0 Snap\emph{!}, we'd
get the same result this way:

mapping over the \emph{rows} of the matrix, applying combine to each
row. Combining rows, reducing column vectors.\\
The outer product block{[}{]}\{index=``outer product block''\} takes two
arrays (vectors, typically) and a dyadic scalar function as inputs. It
reports an array whose rank is the sum of the ranks of the inputs (so,
typically a matrix), in which each item is the result of applying the
function to an atomic element of each array. The third element of the
second row of the result is the value reported by the function with the
second element of the left input and the third element of the right
input. (The APL symbol ◦. is pronounced ``jot dot.'') The way to think
about this block is ``multiplication table{[}{]}\{index=''table''\} ''
from elementary school:

The inner product block{[}{]}\{index=``inner product block''\} takes two
matrices and two operations as input. The number of columns in the left
matrix must equal the number of rows in the right matrix. When the two
operations are + and ×, this is the matrix
multiplication{[}{]}\{index=``multiplication, matrix''\} familiar to
mathematicians:

But other operations can be used. One common inner product is ∨.∧ (``or
dot and'') applied to Boolean matrices, to find rows and columns that
have corresponding items in common.

The printable block{[}{]}\{index=``printable block''\} isn't an APL
function; it's an aid to exploring APL-in-Snap\emph{!}. It transforms
arrays to a compact representation that still makes the structure clear:

Experts will recognize this as the Lisp representation of list
structure,

Index

! block · 32

.csv file · 134

.json file · 134

.txt file · 134

\# variable · 25

\#1 · 69

+ block · 22

× block · 22

≠ block · 20

≤ block · 20

≥ block · 20

⚡ (lightning bolt) · 123

A

a new clone of block · 77

\emph{A Programming Language} · 148

Abelson, Hal · 4

About option · 107

add comment option · 124, 125

Add scene\ldots{} option · 111

additive mixing · 144

Advanced Placement Computer Science Principles · 110

AGPL · 107

all but first blocks · 27

all but first of block · 49

all but first of stream block · 26

all but last blocks · 27

all of block · 28

Alonzo · 9, 55

anchor · 10

anchor (in my block) · 78

animate block · 33

animation · 12

animation library · 33

anonymous list · 46

Any (unevaluated) type · 72

any of block · 28

Any type · 60

APL · 4, 58, 148

APL character set · 149

APL library · 35, 148

APL2 · 149

APL\textbackslash360 · 148

Arduino · 92

arithmetic · 11

array, dynamic · 49

arrow, upward-pointing · 63

arrowheads · 46, 63, 69

ask and wait block · 24

ask block · 86

assoc block · 25

association list · 88

associative function · 51

at block · 19

atan2 block · 20

atomic data · 57

attribute · 76

attributes, list of · 78

audio comp library · 34

B

background blocks · 19

Backgrounds\ldots{} option · 112

backspace key (keyboard editor) · 131

Ball, Michael · 4

bar chart block · 28

bar charts library · 28

base case · 44

BIGNUMS block · 32

binary tree · 47

bitmap · 79, 112

bitwise library · 36

bjc.edc.org · 137

Black Hole problem · 139

block · 6; command · 6; C-shaped · 7; hat · 6; predicate · 12; reporter
· 10; sprite-local · 75

Block Editor · 41, 42, 59

block label · 102

block library · 45, 110

block picture option · 124

block shapes · 40, 60

block variable · 43

block with no name · 32

blockify option · 134

blocks, color of · 40

Boole, George · 12

Boolean · 12

Boolean (unevaluated) type · 72

Boolean constant · 12

box of ten crayons · 139

box of twenty crayons · 139

break command · 99

breakpoint · 17, 118

Briggs, David · 145

broadcast and wait block · 9, 125

broadcast block · 21, 23, 73, 125

brown dot · 9

Build Your Own Blocks · 40

Burns, Scott · 145

button: pause · 17; recover · 39; visible stepping · 18

C

C programming language · 68

call block · 65, 68

call w/continuation block · 97

camera icon · 126

Cancel button · 129

carriage return character · 20

cascade blocks · 26

case-independent comparisons block · 33

cases block · 28

catch block · 26, 99

catch errors library · 31

catenate block · 152

catenate vertically block · 152

center of the stage · 22

center x (in my block) · 78

center y (in my block) · 78

Chandra, Kartik · 4

change background block · 22

Change password\ldots{} option · 113

change pen block · 24, 29, 117, 140

child class · 87

children (in my block) · 78

Church, Alonzo · 9

class · 85

class/instance · 76

clean up option · 125

clear button · 129

clicking on a script · 122

Clicking sound option · 116

clone: permanent · 74; temporary · 74

clone of block · 89

clones (in my block) · 78

cloud (startup option) · 136

Cloud button · 37, 108

cloud icon · 113

cloud storage · 37

CMY · 138

CMYK · 138

codification support option · 117

color at weight block · 145

color block · 140

color chart · 147

color from block · 29, 140

color nerds · 145

color numbers · 29, 138, 139

color of blocks · 40

color palette · 128

color picker · 143

color scales · 141

color space · 138

color theory · 138

Colors and Crayons library · 138

colors library · 29

columns of block · 57

combine block · 50

combine block (APL) · 157

command block · 6

comment box · 125

compile menu option · 123

compose block · 26

compress block · 156

Computer Science Principles · 110

cond in Lisp · 28

conditional library: multiple-branch · 28

constant functions · 71

constructors · 47

contained in block · 153

context menu · 119

context menu for the palette background · 120

context menus for palette blocks · 119

continuation · 93

continuation passing style · 94

Control palette · 7

controls in the Costumes tab · 126

controls in the Sounds tab · 130

controls on the stage · 132

control-shift-enter (keyboard editor) · 132

copy of a list · 50

CORS · 92

cors proxies · 92

costume · 6, 8

costume from text block · 31

costume with background block · 31

costumes (in my block) · 78

Costumes tab · 9, 126

costumes, first class · 79

Costumes\ldots{} option · 112

counter class · 85

CPS · 96

crayon library · 31

crayons · 29, 138, 139

create var block · 32

create variables library · 32

Cross-Origin Resource Sharing · 92

crossproduct · 70

cs10.org · 137

C-shaped block · 7, 67

C-shaped slot · 72

CSV (comma-separated values) · 54

CSV format · 20

csv of block · 57

current block · 92

current date or time · 92

current location block · 34

current sprite · 122

custom block in a script · 124

custom? of block block · 102

cyan · 142

D

dangling rotation · 10

dangling? (in my block) · 78

dark candy apple red · 141

data hiding · 73

data structure · 47

data table · 88

data type · 19, 59

database library · 34

date · 92

Dave, Achal · 4

deal block · 150

debugging · 118

Debugging · 17

deep copy of a list · 50

default value · 63

define block · 102

define of recursive procedure · 104

\emph{definition (of block)} · 102

definition of block · 101

delegation · 87

Delete a variable · 14

delete block definition\ldots{} option · 120

delete option · 124, 128, 133

delete var block · 32

denim · 139

design principle · 46, 77

devices · 91, 92

dialog, input name · 42

dimensions of block · 57

Dinsmore, Nathan · 4

direction to block · 22

Disable click-to-run option · 117

dispatch procedure · 85, 86, 88

distance to block · 22

dl (startup option) · 136

do in parallel block · 31

does var exist block · 32

down arrow (keyboard editor) · 131

Download source option · 108

drag from prototype · 43

draggable checkbox · 122, 132

dragging onto the arrowheads · 69

drop block · 155

duplicate block definition\ldots{} option · 120

duplicate option · 124, 128, 132

dynamic array · 49

E

easing block · 33

easing function · 33

edge color · 129

edit option · 128, 133, 135

edit\ldots{} option · 120

editMode (startup option) · 137

effect block · 19

ellipse tool · 128, 129

ellipsis · 63

else block · 28

else if block · 28

empty input slots, filling · 66, 68, 70

enter key (keyboard editor) · 131

equality of complete structures · 149

eraser tool · 128

error block · 31

error catching library · 31

escape key (keyboard editor) · 130

Examples button · 108

Execute on slider change option · 115

export block definition\ldots{} option · 120

Export blocks\ldots{} option · 110

export option · 128, 133

Export project\ldots{} option · 110

export\ldots{} option · 134, 136

expression · 11

Extension blocks option · 115

extract option · 124

eyedropper tool · 128, 129

F

factorial · 44, 71

factorial · 32

Fade blocks\ldots{} option · 114

fair HSL · 145

fair hue · 29, 141, 143, 146

fair hue table · 146

fair saturation · 146

fair value · 146

Falkoff, Adin · 148

false block · 19

file icon menu · 108

fill color · 129

Finch · 92

find blocks\ldots{} option · 120

find first · 50

first class data · 148

first class data type · 46

first class procedures · 65

first class sprites · 73

first word block · 27

flag, green · 6

Flat design option · 116

flat line ends option · 117

flatten block · 152

flatten of block · 57

floodfill tool, · 128

focus (keyboard editor) · 131

footprint button · 117

for block · 13, 19, 26, 64, 65

for each block · 20

for each item block · 25

For this sprite only · 15

formal parameters · 69

frequency distribution analysis library · 34

from color block · 29, 140, 142

function, associative · 51

function, higher order · 49, 148

function, mixed · 148, 151

function, scalar · 55, 148

functional programming style · 48

G

generic hat block · 6

generic when · 6

get blocks option · 128

getter · 76

getter/setter library · 32

glide block · 115

global variable · 14, 15

go to block · 22

grade down block · 154

grade up block · 154

graphics effect · 19

gray · 139, 141

green flag · 6

green flag button · 118

green halo · 123

Guillén i Pelegay, Joan · 4

H

halo · 11, 123; red · 69

hat block · 6, 41; generic · 6

help\ldots{} option · 119, 123

help\ldots{} option for custom block · 119

hexagonal blocks · 41, 60

hexagonal shape · 12

hide and show primitives · 17

hide blocks option · 120

Hide blocks\ldots{} option · 111

hide var block · 32

hide variable block · 17

hideControls (startup option) · 137

higher order function · 49, 70, 148, 157

higher order procedure · 66

histogram · 34

Hotchkiss. Kyle · 4

HSL · 138, 143

HSL color · 29

HSL pen color model option · 117

HSV · 138, 142

HTML (HyperText Markup Language) · 91

HTTP · 92

HTTPS · 92, 126

Hudson, Connor · 4

hue · 141

Huegle, Jadga · 4

Hummingbird · 92

hyperblocks · 148

Hyperblocks · 55

Hz for block · 34

I

IBM System/360 · 148

ice cream · 109

icons in title text · 64

id block · 71

id option · 22

identical to · 20

identity function · 71

if block · 12

if do and pause all block · 26

if else block · 71

if else reporter block · 19

ignore block · 26

imperative programming style · 48

import\ldots{} option · 134

Import\ldots{} option · 110

in front of block · 49

in front of stream block · 26

index of block (APL) · 152

index variable · 19

indigo · 141

infinite precision integer library · 32

Ingalls, Dan · 4

inherit block · 77

inheritance · 73, 87

inner product block · 158

input · 6

input list · 68, 69

input name · 69

input name dialog · 42, 59

Input sliders option · 115

input-type shapes · 59

instance · 85

integers block · 152

interaction · 15

internal variable · 63

iota block · 152

is \_ a \_ ? block · 19

is flag block · 20

is identical to · 20

item 1 of block · 49

item 1 of stream block · 26

item block · 148

item of block · 56

iteration library · 26

Iverson, Kenneth E. · 4, 148

J

jaggies · 79

Java programming language · 68

JavaScript · 19, 143

JavaScript extensions option · 115

JavaScript function block · 115

jigsaw-piece blocks · 40, 60

join block · 102

JSON (JavaScript Object Notation) file · 54

JSON format · 20

json of block · 57

jukebox · 9

K

Kay, Alan · 4

key:value: block · 34

keyboard editing button · 123

keyboard editor · 130

keyboard shortcuts · 108

key-value pair · 88

L

L*a*b* · 143

L*u*v* · 143

label, block · 102

lambda · 67

lang= (startup option) · 137

Language\ldots{} option · 114

large option · 134

last blocks · 27

layout, window · 5

Leap Motion · 92

left arrow (keyboard editor) · 131

Lego NXT · 92

length block · 148

length of block · 57

length of text block · 22

letter (1) of (world) block · 27

lexical scope · 85

lg option · 22

Libraries\ldots{} option · 25, 111

library: block · 45

license · 107

Lieberman, Henry · 77

Lifelong Kindergarten Group · 4

lightness · 143

lightness option · 117

lightning bolt symbol · 25, 123

line break in block · 64

line drawing tool · 128

lines of block · 57

linked list · 49

Lisp · 58

list ➔ sentence block · 27

list ➔ word block · 27

list block · 46

list comprehension library · 35

list copy · 50

list library · 25

list of procedures · 70

List type · 60

list view · 51

list, linked · 49

list, multi-dimensional · 55

listify block · 34

lists of lists · 47

little people · 44, 96

loading saved projects · 38

local state · 73

local variables · 19

location-pin · 15

Login\ldots{} option · 113

Logo tradition · 27

Logout option · 113

Long form input dialog option · 116

long input name dialog · 59

M

macros · 105

magenta · 141, 142

Make a block · 40

Make a block button · 119

make a block\ldots{} option · 126

Make a list · 46

Make a variable · 14

make internal variable visible · 63

Maloney, John · 4

map block · 50, 65

map library · 35

map over stream block · 26

map to code block · 117

map-pin symbol · 75

maroon · 141

Massachusetts Institute of Technology · 4

mathematicians · 148

matrices · 148

matrix multiplication · 158

max block · 20

McCarthy, John · 4

media computation · 55, 149

Media Lab · 4

memory · 16

menus library · 36

message · 73

message passing · 73, 86

method · 73, 75, 86

methods table · 88

microphone · 82

microphone block · 82

middle option · 127

min block · 20

mirror sites · 137

MIT Artificial Intelligence Lab · 4

MIT Media Lab · 4

mix block · 140

mix colors block · 29

mixed function · 148, 151

mixing paints · 144

Modrow, Eckart · 121

monadic negation operator · 22

Morphic · 4

Motyashov, Ivan · 4

mouse position block · 21

move option · 133

MQTT library · 36

multiline block · 33

multimap block · 25

multiple input · 63

multiple-branch conditional library · 28

multiplication table · 158

multiplication, matrix · 158

mutation · 48

mutators · 47

my block · 73, 76

my blocks block · 102

my categories block · 102

N

name (in my block) · 78

name box · 122

name, input · 69

nearest color number · 142

neg option · 22

negation operator · 22

neighbors (in my block) · 78

nested calls · 70

Nesting Sprites · 10

New category\ldots{} option · 111

new costume block · 80

new line character · 64

New option · 108

New scene option · 111

new sound block · 84

new sprite button · 8

newline character · 20

Nintendo · 92

noExitWarning (startup option) · 137

nonlocal exit · 99

normal option · 134

normal people · 145

noRun (startup option) · 137

Number type · 60

numbers from block · 20

O

object block · 73

Object Logo · 77

object oriented programming · 73, 85

Object type · 60

objects, building explicitly · 85

of block (operators) · 22

of block (sensing) · 24, 106

of costume block · 79

open (startup option) · 136

Open in Community Site option · 113

Open\ldots{} option · 108

operator (APL) · 148, 157

orange oval · 13

other clones (in my block) · 78

other sprites (in my block) · 78

outer product block · 158

outlined ellipse tool · 128

outlined rectangle tool · 128

oval blocks · 40, 60

P

paint brush icon · 126

Paint Editor · 126

Paint Editor window · 128

paintbrush tool · 128

paints · 144

Paleolithic · 150

palette · 6

palette area · 119

palette background · 120

Parallax S2 · 92

parallelism · 8, 48

parallelization library · 31

parent (in my block) · 78

parent attribute · 77

parent class · 87

parent\ldots{} option · 136

\emph{Parsons problems} · 117

parts (in my block) · 78

parts (of nested sprite) · 10

pause all block · 17, 118

pause button · 17, 118

pen block · 24, 29, 117, 140

pen down? block · 19

pen trails block · 18

pen trails option · 135

pen vectors block · 18

permanent clone · 74, 136

physical devices · 91

pic\ldots{} option · 135, 136

picture of script · 124

picture with speech balloon · 124

picture, smart · 124

pink · 141

pivot option · 133

pixel · 79

pixel, screen · 19

pixels library · 27

Plain prototype labels option · 116

play block · 34

play sound block · 9

playing sounds · 9

plot bar chart block · 28

plot sound block · 34

point towards block · 22

points as inputs · 22

polymorphism · 75

position block · 21, 33

Predicate block · 12

preloading a project · 136

present (startup option) · 136

presentation mode button · 118

primitive block within a script · 123

printable block · 27, 158

procedure · 12, 66

Procedure type · 72

procedures as data · 9

product block · 22, 28

project control buttons · 118

Project notes option · 108

Prolog · 58

prototype · 41

prototyping · 76, 88

pulldown input · 61

pumpkin · 139

purple · 142

R

rainbow · 141

rank · 148

rank of block · 57, 151

ravel block · 149

raw data\ldots{} option · 134

ray length block · 22

read-only pulldown input · 61

receivers\ldots{} option · 125

recover button · 39

rectangle tool · 128

recursion · 43

recursive call · 68

recursive operator · 71

recursive procedure using define · 104

red halo · 68, 69, 123

redo button · 123

redrop option · 125

reduce block · 156, 157

Reference manual option · 108

reflectance graph · 144

relabel option · 20

relabel\ldots{} option · 123, 124

release option · 136

Remove a category\ldots{} option · 111

remove duplicates from block · 25

rename option · 128

renaming variables · 15

\textbf{repeat} block · 7, 67

repeat blocks · 26

repeat until block · 12

report block · 44

Reporter block · 10

reporter \textbf{if} block · 12

reporter if else block · 19

reporters, recursive · 44

Reset Password\ldots{} option · 113

reshape block · 56, 148, 151

Restore unsaved project option · 39

result pic\ldots{} option · 124, 125

reverse block · 156

reverse columns block · 156

Reynolds, Ian · 4

RGB · 138

RGBA option · 19

right arrow (keyboard editor) · 131

ring, gray · 49, 66, 68

ringify · 66

ringify option · 124

Roberts, Eric · 44

robots · 91, 92

rods and cones · 141

roll block · 150

Romagosa, Bernat · 4

rotation buttons · 122

rotation point tool · 128, 129

rotation x (in my block) · 78

rotation y (in my block) · 78

run (startup option) · 136

run block · 65, 68

run w/continuation · 99

S

safely try block · 31

sample · 82

saturation · 143

Save as\ldots{} option · 110

Save option · 110

save your project in the cloud · 37

scalar = block · 150

scalar function · 55, 148, 150

scalar join block · 150

scenes · 111, 136

Scenes\ldots{} option · 111

Scheme · 4

Scheme number block · 32

SciSnap\emph{!} · 121

SciSnap\emph{!} library · 36

scope: lexical · 85

Scratch · 5, 9, 40, 46, 47, 48, 59

Scratch Team · 4

screen pixel · 19

script · 5

script pic · 43

script pic\ldots{} option · 124

\textbf{script variables} block · 15, 19, 86

scripting area · 6, 122

scripting area background context menu · 125

scripts pic\ldots{} option · 126

search bar · 109

search button · 119

secrets · 107

select block · 156

selectors · 47

self (in my block) · 78

senders\ldots{} option · 125

sensors · 91

sentence ➔ list block · 27

sentence block · 25

sentence library · 27

sentence➔list block · 25

separator: menu · 62

sepia · 139

serial-ports library · 33

Servilla, Deborah · 4

set \_ of block \_ to \_ block · 102

set background block · 22

\textbf{set} block · 15

set flag block · 20, 32

set pen block · 24, 29, 117, 139, 140

set pen to crayon block · 30, 139

set value block · 32

set var block · 32

setter · 76

setting block · 32

settings icon · 114

shade · 141

shallow copy of a list · 50

shape of block · 151

shapes of blocks · 40

shift-arrow keys (keyboard editor) · 131

Shift-click (keyboard editor) · 130

shift-click on block · 124

shift-clicking · 107

shift-enter (keyboard editor) · 130

Shift-tab (keyboard editor) · 130

shortcut · 126, 135

shortcuts: keyboard · 108

show all option · 135

Show buttons option · 117

Show categories option · 117

show option · 136

show primitives option · 121

show stream block · 26

show var block · 32

show variable block · 17

shown? block · 19

shrink/grow button · 118

sieve block · 26

sign option · 22

Signada library · 36

signum block · 150

Signup\ldots{} option · 113

simulation · 73

sine wave · 83

Single palette option · 117

single stepping · 18

slider: stepping speed · 18

slider max\ldots{} option · 134

slider min\ldots{} option · 134

slider option · 134

Smalltalk · 58

smart picture · 124

snap block · 27

snap option · 22

Snap\emph{!} logo menu · 107

Snap\emph{!} manual · 124

Snap\emph{!} program · 5

Snap! website option · 108

snap.berkeley.edu · 108

solid ellipse tool · 128

solid rectangle tool · 128

sophistication · 72

sort block · 25

sound · 82

sound manipulation library · 34

sounds (in my block) · 78

sounds, first class · 79

Sounds\ldots{} option · 113

source files for Snap\emph{!} · 108

space key (keyboard editor) · 131

speak block · 31

special form · 72

spectral colors · 141

speech balloon · 124

speech synthesis library · 31

split block · 20, 91

split by blocks block · 101

split by line block · 57

spreadsheet · 149

sprite · 6, 73

sprite appearance and behavior controls · 122

sprite corral · 8, 135

sprite creation buttons · 135

sprite nesting · 10

sprite-local block · 75

sprite-local variable · 14, 15

square stop sign · 6

squiral · 13

stack of blocks · 6

stage · 6, 73

stage (in my block) · 78

stage blocks · 19

Stage resizing buttons · 118

Stage size\ldots{} option · 114

Stanford Artificial Intelligence Lab · 4

starting Snap\emph{!} · 136

Steele, Guy · 4

stop all block · 118

stop block · 22

stop block block · 44

stop button · 118

stop script block · 44

stop sign · 8

stop sign, square · 6

Stream block · 26

stream library · 26

Stream with numbers from block · 26

stretch block · 80

string processing library · 33

\emph{Structure and Interpretation of Computer Programs} · 4

submenu · 62

substring block · 33

subtractive mixing · 144

sum block · 22, 28

Super-Awesome Sylvia · 92

Sussman, Gerald J. · 4

Sussman, Julie · 4

svg\ldots{} option · 135

switch in C · 28

symbols in title text · 64

synchronous rotation · 10

system getter/setter library · 32

T

tab character · 20

tab key (keyboard editor) · 130

table · 158

table view · 51

take block · 155

teal · 142

temporary clone · 74, 133

Terms of Service · 38

termwise extension · 148

text costume library · 31

text input · 9

Text type · 60

text-based language · 117

text-to-speech library · 31

\emph{Thinking Recursively} · 44

thread · 100

thread block · 100

Thread safe scripts option · 116

throw block · 26

thumbnail · 122

time · 92

tint · 141

tip option · 127

title text · 42

to block · 22

tool bar · 6

tool bar features · 107

touching block · 22

transient variable · 16

translation · 114

translations option · 43

transparency · 30, 79, 140

transparent paint · 129

transpose block · 156

true block · 19

TuneScope library · 36

Turbo mode option · 115

turtle costume · 126

Turtle costume · 9

turtle's rotation point · 127

two-item (x,y) lists · 22

type · 19

U

Undefined! blocks · 120

Undelete sprites\ldots{} option · 113

undo button · 123, 129

undrop option · 125

unevaluated procedure types · 61

unevaluated type · 72

Unicode · 149

Uniform Resource Locator · 91

unringify · 66, 86

unringify option · 124

Unused blocks\ldots{} option · 111

up arrow (keyboard editor) · 131

upvar · 64

upward-pointing arrow · 63

url block · 34, 91

USE BIGNUMS block · 32

use case-independent comparisons block · 33

user interface elements · 107

user name · 37

V

value · 143

value at key block · 34

var block · 32

variable · 13, 76; block · 43; global · 14; renaming · 15; script-local
· 15; sprite-local · 14, 15; transient · 16

variable watcher · 14

variable-input slot · 68

variables in ring slots · 66

variables library · 32

variables, local · 19

variadic · 22

variadic input · 46, 63

variadic library · 28

vector · 112

vector editor · 129

vectors · 148

video block · 22

video on block · 80

violet · 142

visible stepping · 45, 117

visible stepping button · 18

visible stepping option · 115

visual representation of a sentence · 27

W

wardrobe · 9

warp block · 19, 123

watcher · 15

Water Color Bot · 92

web services library · 34

when I am block · 23

when I am stopped script · 23

when I receive block · 23

when, generic · 6

white · 142

white background · 141

whitespace · 20

Wiimote · 92

window layout · 5

with inputs · 66

word ➔ list block · 27

word and sentence library · 27

world map library · 35

World Wide Web · 91

write block · 18

writeable pulldown inputs · 61

X

X position · 11

X11/W3C color names · 29

Xerox PARC · 4

Y

Y position · 11

yield block · 100

Yuan, Yuan · 4

Z

zebra coloring · 11

Zoom blocks\ldots{} option · 114

{[}1{]} One of the hat blocks, the generic{[}{]}\{index=``hat
block:generic''\} ``when anything'' block , is subtly different from the
others. When the stop sign is clicked, or when a project or sprite is
loaded, this block doesn't test whether the condition in its hexagonal
input slot is true, so the script beneath it will not run, until some
\emph{other} script in the project runs (because, for example, you click
the green flag). When generic when{[}{]}\{index=``generic when''\}
blocks are disabled, the stop sign{[}{]}\{index=``stop sign, square''\}
will be square{[}{]}\{index=``square stop sign''\} instead of octagonal.

{[}2{]} The hide variable and{[}{]}\{index=``hide variable block''\}
show variable block{[}{]}\{index=``show variable block''\} s can also be
used to hide and show primitives{[}{]}\{index=``hide and show
primitives''\} in the palette. The pulldown menu doesn't include
primitive blocks, but there's a generally useful technique to give a
block input values it wasn't expecting using run or call:

In order to use a block as an input this way, you must explicitly put a
ring around it, by right-clicking on it and choosing ringify. More about
rings in Chapter VI.

{[}3{]} This use of the word ``prototype'' is unrelated to the
\emph{prototyping object oriented programming} discussed later.

{[}4{]} Note to users of earlier versions: From the beginning, there has
been a tension in our work between the desire to provide tools such as
for (used in this example) and the higher order functions introduced on
the next page as primitives, to be used as easily as other primitives,
and the desire to show how readily such tools can be implemented in
Snap\emph{!} itself. This is one instance of our general pedagogic
understanding that learners should both use abstractions and be
permitted to see beneath the abstraction barrier. Until version 5.0, we
used the uneasy compromise of a library of tools written in Snap\emph{!}
and easily, but not easily enough, loaded into a project. By \emph{not}
loading the tools, users or teachers could explore how to program them.
In 5.0 we made them true primitives, partly because that's what some of
us wanted all along and partly because of the increasing importance of
fast performance as we explore ``big data'' and media computation. But
this is not the end of the story for us. In a later version, after we
get the design firmed up, we intend to introduce ``hybrid'' primitives,
implemented in high speed Javascript but with an ``Edit'' option that
will open, not the primitive implementation, but the version written in
Snap\emph{!}. The trick is to ensure that this can be done without
dramatically slowing users' projects.

{[}5{]} In Scratch, every block that takes a Text-type input has a
default value that makes the rectangles for text wider than tall. The
blocks that aren't specifically about text either are of Number
type{[}{]}\{index=``Number type''\} or have no default value, so those
rectangles are taller than wide. At first some of us (bh) thought that
Text was a separate type that always had a wide input slot; it turns out
that this isn't true in Scratch (delete the default text and the
rectangle narrows), but we thought it a good idea anyway, so we allow
Text-shaped boxes even for empty input slots. (This is why Text comes
just above Any in the input type selection box.)

{[}6{]} There is a primitive id function in the menu of the sqrt of
block, but we think seeing its (very simple) implementation will make
this example easier to understand.

{[}7{]} Some languages popular in the ``real world'' today, such as
JavaScript, claim to use prototyping, but their object system is much
more complicated than what we are describing (we're guessing it's
because they were designed by people too familiar with class/instance
programming); that has, in some circles, given prototyping a bad name.
Our prototyping design comes from Object Logo{[}{]}\{index=``Object
Logo''\} , and before that, from Henry
Lieberman{[}{]}\{index=``Lieberman, Henry''\} . {[}Lieberman, H., Using
Prototypical Objects to Implement Shared Behavior in Object-Oriented
Systems, First Conference on Object-Oriented Programming Languages,
Systems, and Applications {[}OOPSLA-86{]}, ACM SigCHI, Portland, OR,
September, 1986. Also in \emph{Object-Oriented Computing,} Gerald
Peterson, Ed., IEEE Computer Society Press, 1987.{]}

{[}8{]} \emph{Neighbors} are all other sprites whose bounding boxes
intersect the doubled dimensions of the requesting sprite's bounds.

\chapter{}\label{section-4}

\chapter{Example Test Title}\label{example-test-title}

\chapter{Quarto Book Examples and
Tests}\label{quarto-book-examples-and-tests}

This is a document designed for testing / reference.

Some text {in PDF.}

You can also mark content as visible for all formats except a specified
format. For example:

This will only appear in PDFs.

\chapter{References}\label{sec-test-referencekey}

You can link to references~\ref{sec-test-referencekey}.

\phantomsection\label{refs}
\begin{CSLReferences}{0}{1}
\end{CSLReferences}

\chapter*{Index}\label{index}
\addcontentsline{toc}{chapter}{Index}

\markboth{Index}{Index}

\printindex


\backmatter

\printindex


\end{document}
